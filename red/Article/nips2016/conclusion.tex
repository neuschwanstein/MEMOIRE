\section{Discussion}\label{sec:conc}

As a conclusion, we would like to review the main messages we hope to deliver from this
paper. First off, it is possible to use side information from the market, such as market
news, financial indicators, economic variables and so on in order to build a portfolio
with actual performance guarantees on the out-of-sample. Second, it is also possible to
obtain performance bounds on the suboptimality of the empirical decision in comparison to
what might have been the best decision, given full knowledge of the market. Third, there
might be a cost to pay for increasing the number of side information treated by the model
when the sample size on which the decision is based on is not large enough.

That said, for ``small-data'' situations where $p=o(\sqrt{n})$, we believe our framework
can be particularly well suited for aggregating and treating market side information in
order to make a sound investment decision that is guaranteed to appeal to the investor's
perception of risk.

% % Finally, we would like to review once again the main message from Section
% % \ref{sec:bigdata}. In particular, it is important to understand that Corollary 1 serves as
% % a worst case scenario and that we don't necessarily expect to observe downgrading
% % performances as soon as $n=o(p^2)$. Still, no matter what, there is a great cost to pay in
% % pouring more and more features into this a portfolio problem, and this cost is directly
% % exhibited through $\xi^2$ and the loosening of the guarantees bound. One might therefore
% % wish to be prudent when facing such high-risk regimes.

% However, for situation where $p=o(\sqrt{n})$, we believe our framework can be particularly
% well-suited in analyzing portfolio selection based on side information of the market,
% since performance on the investor certainty equivalent are known to be bound and to shrink
% at a $O(p/\sqrt{n})$ rate.  \Erick{La conclusion doit être retravaillée. Noramelement, on
%   resume le message important du papier ce qui n'est pas fait. La discussion que tu fait
%   pourrait être raccourcie et insérée à la fin de la section 5. }

%%% Local Variables:
%%% mode: latex
%%% TeX-master: "big_data_portfolio_optimization"
%%% End:
