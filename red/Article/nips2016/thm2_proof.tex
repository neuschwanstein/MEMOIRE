\subsection{Proof of Theorem \ref{thm_truopt}}

First, in order to tidy up the proof, let us define $\EU(q) := \E_F(u(R\cdot q^T X))$ and $\EU_\lambda(q) := \E_F(u(R\cdot q^T X)) - \lambda\|q\|^2$,
with $q^\star := \argmin_q \EU(q)$ and $q^\star_\lambda := \argmin_q \EU_\lambda(q)$.

\begin{thm}[Theorem 1 and surrounding text  in \cite{sridharan2009fast}]\label{thm:shai}
  Given that assumptions \ref{ass:R}, \ref{ass:X}, and \ref{ass:u} are satisfied, and
  since the function $EU_\lambda$ is $2\lambda$-strongly convex, then one has with confidence of
  $1-\delta$ that
  \[
    -\lambda\|\hat q - q^\star_\lambda\| \geq \EU_\lambda(\hat q) - \EU_\lambda(q^\star_\lambda) \geq -\omega,
  \]
  where
  \[
    \omega = \frac{4\gamma^2\xi^2(32+\log(1/\delta))}{\lambda n}.
  \]
\end{thm}

Notice that Theorem \ref{thm:shai} implies with confidence $1-\delta$ that
\[
  \EU(\hat q) - \EU(q^\star_\lambda) \geq \lambda\bigl(\|\hat q\|^2 -
  \|q^\star_\lambda\|^2\bigr) -\omega
  \geq -\lambda\bigl(\|\hat q -
  q^\star_\lambda\|^2 +2\|\hat q\|\|\hat q - q^\star_\lambda\|\bigr) - \omega.
\]
As shown in Section \ref{sec:thm1_proof}, $\|\hat q\| \leq \bar r\xi/(2\lambda)$. Theorem \ref{thm:shai} further implies concerning the same $1-\delta$ probability outcomes that
$\|\hat q - q^\star_\lambda\|^2 \leq \omega/\lambda$, and therefore
$\|\hat q - q^\star_\lambda\| \leq \sqrt{\omega/\lambda}$, so that we end up with
\[
  \EU(\hat q) - \EU(q^\star_\lambda) \geq -2\omega - \bar r\xi\sqrt{\frac{\omega}{\lambda}}.
\]
with probability $1-\delta$. Finally, note that since by the  definition of $q^\star_\lambda$, we have that
\[
  \EU(q^\star) - \lambda\|q^\star\|^2 \leq \EU(q^\star_\lambda) - \lambda\|q^\star_\lambda\|^2\;,
\]
it follows that
\[
  \EU(q^\star) - \EU(q^\star_\lambda) \leq \lambda\bigl(\|q^\star\|^2 -
  \|q^\star_\lambda\|^2\bigr) \leq \lambda\|q^\star\|^2,
\]
so that we can bound the suboptimality of the policy $\hat q$ with probability $1-\delta$
in the following fashion:
\begin{align*}
  \EU(\hat q) &= \EU(q^\star) + \EU(\hat q) - \EU(q^\star_\lambda) + \EU(q^\star_\lambda)
                - \EU(q^\star)\\
              &\geq \EU(q^\star) - \lambda\|q^\star\|^2 - \bar r\xi\sqrt{\frac{\omega}{\lambda}} - 2\omega.
\end{align*}
This relation can be exploited in a similar way as in the proof of Theorem \ref{thm:outsampleBound1} (see Section \ref{sec:thm1_proof}) to derive the relation between certainty equivalents that is presented in our theorem.
the same trick as in Section .

Next, we show that $\CE(q^\star;F)-\CE(\qhat;F)$ is bounded by proving that $\|q^\star\|$ is
finite. Since the other terms of the upper bound established above are also finite, the second part of Theorem
\ref{thm_truopt} follows.


Instead of optimizing $q$ directly, as was done previously, we can reformulate problem \eqref{EUF} in terms of both  an orientation vector and a scale decision variable. This gives us
  \begin{eqnarray*}
    \maximize_{s\in\Re,v\in\Re^p} && \E[ u(s R\,X^T v)] \\
    \st&& s\geq0\;,\;\|v\|=1\;.
  \end{eqnarray*}

Based on assumption \ref{ass:arbitrage}, since no feature induce arbitrage, it follows that,
there exists a $\delta>0$ such that  $\pp\{R\,X^T v<-\delta\}=\varrho>0$ for all $v$ with a norm of one. Now,
let $B$ be a discrete random variable with two states such that
$\pp\{B=-\delta\}=1-\pp\{B=\bar r\xi\}=\varrho$. Since $|R\,X^T v|<\bar r\xi$, we have
that $\pp\{B\geq r\} \geq \pp\{R\,X^T v\geq r\}$ for all $r\in\Re$, i.e. that $B$ stochastically dominates $R\,X^T v$, so that 
it must necessarily follow that $\E[u(sB)]\geq \E[u(sR\,X^T v)]$. But, by the sublinearity asumption on
$u$,
\begin{align*}
  \lim_{s\to\infty}\E[u(sR\,X^T v)] &\leq   \lim_{s\to\infty}\E[u(sB)]  = \lim_{s\to\infty}\big(\varrho  u(-s\delta)+(1-\varrho)u(s\bar r\xi)\big) \\
  &\leq   \lim_{s\to\infty} -\varrho s\delta + (1-\varrho) o(s) =  -\infty\;
 \end{align*} 
%\[
%  \lim_{s\to\infty}\E[u(sB)] = \lim_{s\to\infty}\big(\varrho
%  u(-s\delta)+(1-\varrho)u(s\bar r\xi)\big) \leq   \lim_{s\to\infty} -\varrho s\delta + (1-\varrho) o(s) =  -\infty\;
%\]
for all $v$ of norm one, which shows that
$s^\star$, and therefore $\|q^\star\|$, is bounded.






%%% Local Variables:
%%% mode: latex
%%% TeX-master: "big_data_portfolio_optimization"
%%% End:
