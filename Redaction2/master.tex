\documentclass[11pt]{article}

\newcommand{\ts}{\textsuperscript}
\newcommand{\figref}[1]{Fig.~\ref{#1}}
\usepackage{amsmath}
\usepackage{amsthm}
\usepackage{graphicx}
\usepackage{geometry}
\usepackage{subcaption}
\usepackage{bm}
\usepackage{hyperref}
\usepackage[retainorgcmds]{IEEEtrantools}
\usepackage{mathtools}
\usepackage{color}
\usepackage{marginnote}
\usepackage[utf8]{inputenc}


\DeclareMathOperator*{\argmax}{arg\,max}
\DeclareMathOperator*{\argmin}{arg\,min}
\DeclareMathOperator{\st}{s.t.}
\DeclareMathOperator{\epi}{epi}
\DeclareMathOperator{\diag}{diag}
\DeclareMathOperator{\dom}{dom}
\DeclareMathOperator{\tr}{tr}
\DeclareMathOperator*{\minimize}{minimize}
\DeclarePairedDelimiter\floor{\lfloor}{\rfloor}

\newcommand{\iso}{\simeq}
\newcommand{\dd}{\partial}
\newcommand{\real}{\bm R}
\newcommand{\trueRisk}{R_{\mathrm{true}}}

\newcommand{\starsection}{\vspace{1em}\begin{center}$\star\quad\star\quad\star$\end{center}\vspace{1em}}


\newcommand{\hilight}[1]{\colorbox{yellow}{#1}}
\let\oldmarginnote\marginnote
\renewcommand{\marginnote}[1]{\oldmarginnote{\footnotesize\emph{#1}}[0cm]}

\theoremstyle{plain}
\newtheorem{prop}{Proposition}
\newtheorem{thm}{Theorem}

\theoremstyle{definition}
\newtheorem*{deff}{Definition}
\newtheorem*{rem}{Remark}


\geometry{letterpaper}
\IEEEeqnarraydefcolsep{0}{\leftmargini}


\title{Portfolio Optimization in a Big Data Context}
\author{Thierry \textsc{Bazier-Matte}}
\date{Summer 2015}

\begin{document}
\maketitle


The goal of this document is twofold. First, given a risk averse utility function and a
sample of the market observations (returns and states), we want to prescribe an optimal
linear investment policy depending on the risk aversion of the investor. Second, we also
wish to return to the investor probabilist bounds on the certainty equivalent of the
prescribed investment policy.

\section{Assumptions and Definitions}

In the following, $\bm A$ (capital boldface) are assumed to represent a real subset of any
dimension, $A$ (capital case) represents random variables (or distributions) and $a$
(lower case) represents deterministic variables or realizations. $\real$ represents the
real set.

Let $M=(X,R)$ the \textsl{market} be a random variable with support
$\bm M = \bm X\times \bm R \subseteq\real^{p+1}$, ie. numerically qualifiable, with
$(x,r) = m\sim M$ a \textsl{market observation}, consisting in one part \textsl{state}
$x\in\real^p$ and another part \textsl{outcome} $r\in\real$. Typically $x$ is a vector of
observations from various variable of interests, such as financial or economical news,
etc. Scalar $r$ in this article shall represent the return from a financial asset of
interest. Finally, let $M_n = \{M,\dots,M\}$ be a \textsl{random set} of $n$ (unrealized)
observations (with support $\bm M^n$). Therefore $\mu_n \sim M_n$ represents an iid sample
of $n$ market observations.

We shall study linear investment decisions $q$ such that $p=q^Tx$ part of the portfolio is
allocated to the risky asset with return $r$, and $1-p$ part of the portfolio is allocated
to the riskless asset, providing a total return $rp + R_f(1-p)$.
\begin{assumption}
  We suppose that observed returns $r$ are constrained by $|r| \leq \bar r$ and that
  observed states $x$ are constrained by $\|x\|_2 \leq \xMax$.
\end{assumption}

\begin{assumption}
  We also suppose that the investor's utility function $u$ is such that $u$ can be
  rescaled to a standard utility $\bar u$ with affine transformation
  $u(r) = k\bar u(r) + l$, where $k>0$ and $\bar u$ has the following properties:
  \begin{itemize}
  \item $\bar u$ is concave and non-decreasing;
  \item Let $X$ be the random returns of the portfolio. Then
    $CE[X] \leq E[X] - \epsilon Var(X)$;
  \item $\bar u$ is Lipschitz continuous, with constant $\gamma$, ie.
    $|\bar u(r_1) - \bar u(r_2)| \leq \gamma|r_1-r_2|$;
  \item $\bar u$ is bounded above, ie. there exists a constant $M$ such that $\bar
    u(r)\leq M$ for any $r$;
  \item $\bar u(0) = 0$;
  \item $\lim_{r\to 0^{+}} \bar u(r) = 1$.
  \end{itemize}
  Examples of standard utility functions are $\bar u(r) = \min(\beta x,x)$ with
  $\beta > 1$ and $\bar u(r) = \beta^{-1}(e^{\beta x}-1)$ with $\beta>0$. [Pente non
  contrôlée!]
\end{assumption}

\begin{deff}
  Let $\ell:\bm M\times \bm Q\to\real$ be a \textsl{loss function} defined by
  \begin{equation*}
    \ell(m,q) = \ell(x,r,q) = -u(r\,q^{T}x + R_f (1 - q^{T}x)),
  \end{equation*}
  where $R_f$ is the risk free return rate. We also define the \textsl{cost function}
  $c:\real\times\bm R\to\real$ as
  \begin{equation*}
    c(p,r) = -u(pr + (1-p)R_f),
  \end{equation*}
  so that $\ell(x,r,q) = c(q^Tx,r)$. 
\end{deff}

\begin{deff}
  The \textsl{empirical risk} $\hat R: \bm M^n\times \bm Q \to \real$ associated with decision
  $q$ and market sample $\mu_n$ is given by
 \begin{equation*}
   \hat R_{\mu_n}(q) = n^{-1} \sum_{i=1}^n \ell(m_i,q).
 \end{equation*}
\end{deff}

\begin{deff}
  The \textsl{empirical decision algorithm} $\hat A_n:\bm M^n \to \bm Q$ associated with
  market sample $\mu_n$ is the optimal value of the problem
  \begin{equation*}
    \text{minimize}\quad\hat R_{\mu_n}(q) + \lambda\|q\|_2^2.
  \end{equation*}
\end{deff}

From now on, $\hat q_n := \hat A_n(\mu_n)$ the empirical decision associated with market sample
$\mu_n$ and $\hat Q_n := A_n(S_n)$ the random empirical decision, ie. $\hat q_n \sim \hat Q_n$.

\begin{deff}
  The \textsl{true risk} $\trueRisk:\bm Q\to\real$ associated with decision $q$ is given by
  \begin{equation*}
    \trueRisk(q) = E_M[\ell(m,q)].
  \end{equation*}
\end{deff}

\begin{deff}
  The \textsl{optimal decision} $\qStar$ is the optimal value of the problem
  \begin{equation*}
    \text{minimize}\quad \trueRisk(q).
  \end{equation*}
\end{deff}

\begin{deff}
  The \textsl{optimal regularized decision} $\qStar_\lambda$ is the optimal value of the
  regularized problem 
  \begin{equation*}
        \text{minimize}\quad \trueRisk(q) + \lambda\|q\|_2^2.
  \end{equation*}
  % In particular, $\qStar = \qStar_{\lambda=0}$. 
\end{deff}


\section{[Stability Definitions and Theorems]}

We adapt in this section a number of theorems and definitions from Bousquet 2002 [Add
reference].

\subsection{Algorithmic Stability}

\begin{deff}
  Let $\hat q_n=\hat A_n(\mu_n)$ and
  $\hat q_{n\backslash i}=\hat A_n(\mu_{n\backslash i})$, where $\mu_n$ and
  $\mu_{n\backslash i}$ only differs in their $i$\textsuperscript{th} observation, which
  has been redrawn from $M$ in the case of $\mu_{n\backslash i}$. The algorithm $\hat A_n$
  is said to have \textsl{uniform stability} $\alpha_n$ if, for any $m\sim M$,
  \begin{equation*}
    |\ell(m,\hat q_n) - \ell(m,\hat q_{n\backslash i})| \leq \alpha_n. 
  \end{equation*}
\end{deff}

\begin{deff}
  A loss function $\ell$ is $\sigma$-admissible if its cost function $c$ is convex with
  respect to $p$ the investment decision and the following holds for any $p_1,p_2$ and r:
  \begin{equation*}
    |c(p_1,r) - c(p_2,r)| \leq \sigma|p_1-p_2|.
  \end{equation*}
\end{deff}

\begin{rem}
  The loss function as defined above is $\sigma$-admissible with $\sigma=k\gamma(r+R_f)$. See
  Claim \ref{claim0}.
\end{rem}

\begin{thm}
  If $\ell$ is $\sigma$-admissible and if, for any $x\in\bm X$, $\|x\|_2^2\leq\xMax^2$,
  then $\hat A_n$ has uniform stability with
  \begin{equation*}
    \alpha_n = \frac{\sigma^2 \xMax^2}{2\lambda n}.
  \end{equation*}
\end{thm}

\begin{proof}
  See Bousquet, Theorem 22. 
\end{proof}

We therefore conclude that $\hat A_n$ has uniform stability with
\begin{equation*}
  \alpha_n = \frac{k^2\gamma^2(\bar r+R_f)^2\xMax^2}{2\lambda n}.
\end{equation*}

\subsection{Out of Sample Bound}

\begin{thm}
  \label{thm2}
  If $\hat A_n$ has uniform stability $\alpha_n$ and the loss function is such that for
  any $m\sim M$ and any $\hat q_n=\hat A_n(\mu_n)$, $0\leq \ell(m,\hat q_n)\leq B_n$, then
  for any $\delta\in(0,1)$, the following bound holds with probability at least $1-\delta$
  over the random sample draw $\mu_n\sim M_n$:
  \begin{equation*}
    |\trueRisk(\qHat_n) - \hat R(\qHat_n)| \leq 2\alpha_n + (4n\alpha_n + B_n)\sqrt{\frac{\log(2/\delta)}{2n}}.
  \end{equation*}
\end{thm}

The bound $B_n$ from Theorem \ref{thm2}, can be explicitly stated: the highest loss occurs
when investment decision is at its highest and the return at its lowest. With the
definition of $\bar p$ from Claim \ref{claim_p_bound}, we find that
\begin{align*}
  c(p,r) &\leq c(\bar p,-\bar r)\\
         &= -k\bar u(-\bar p\bar r+(1-\bar p)R_f) - l\\
         &\leq |l| + k|\bar u(-\bar p\bar r+(1-\bar p)R_f)|\\
         &\leq |l| + k\gamma|R_f - \bar p(\bar r + R_f)|.
\end{align*}
[Can we do better? Here, we've taken the maximum slope of $\bar u$, ie. $\gamma$ to derive
the results, where as we should check the slope at this particular point.]

\subsection{True Optimal Bound}

We now derive a measure of performance where we compare the empiric risk $\hat R(\hat q)$
with the optimal risk $\trueRisk(\qStar)$.

We have the following inequality:
\begin{equation*}
  |\trueRisk(\qStar) - \hat R(\qHat_n)| \leq |\trueRisk(\hat q) - \hat R(\hat q)| +
  |\trueRisk(\qStar) - \trueRisk(\hat q)|.
\end{equation*}
The first term has already been identified in the previous equation (REFERENCE), while the
second is bounded by
\begin{equation*}
  |\trueRisk(\qStar) - \trueRisk(\hat q)| \leq (\bar r +R_f)\xMax\|\qStar - \qHat_n\|_2 + R_f.
\end{equation*}
See Claim \ref{claim1} in appendix for proof. 

Therefore, what we're looking for is a probabilistic bound on the variance of
$\hat q_n = \hat A_n(\mu_n)$ [because, using CLT, $\qHat_n$ converges in distribution to
$\qStar_\lambda$ and therefore its variance can tell us how close we are from its expected
value $\qStar_\lambda$].

\newpage

\section*{Appendix}
\begin{claim}
  \label{claim0}
  The following inequality holds:
  \begin{equation*}
    |c(p_1,r) - c(p_2,r)| \leq k\gamma(\bar r+R_f)|p_1-p_2|.
  \end{equation*}
\end{claim}
\begin{proof}
  Using the Lipschitz property of $\bar u$ we have:
  \begin{align*}
    |c(p_1,r) - c(p_2,r)| &= |u(p_1r + (1-p_1)R_f) - u(p_2r + (1-p_2)R_f)|\\
    &= k|\bar u(p_1 r + (1-p_1)R_f) - \bar u(p_2 r + (1-p_2)R_f)|\\
    &\leq k c |p_1 r + (1-p_1)R_f - p_2r - (1-p_2)R_f|\\
    &= k\gamma |r-R_f||p_1-p_2|\\
    &\leq k\gamma(\bar r+R_f)|p_1-p_2|.\qedhere
  \end{align*}
\end{proof}


\begin{claim}
  \label{claim_p_bound}
  The $\ell_2$ norm of decision vectors $\qHat_n$ obtained from algorithm $\hat A_n$ are
  bounded by $\bar p = k\gamma\xMax(\bar r-R_f)/(2\lambda)$.
\end{claim}
\begin{proof}
  Let $\mu_n$ be a sample of the market. The empirical decision algorithm 
  \begin{equation*}
    \minimizeEquation{n^{-1}\sum_{i=1}^n\ell(m_i,q) + \lambda\|q\|^2_2}
  \end{equation*}
  is equivalent to 
  \begin{align*}
    \minimizeEquationSt{n^{-1}\sum_{i=1}^n \ell(m_i,sq) + \lambda s^2}[s\geq0\\,\|q\|_2=1],
  \end{align*}
  where the optimization variables are now on the direction ($q$) and the scale
  ($s$). Therefore, for any direction $q$, we can define a convex function $g(s)$ which
  becomes the objective:
  \begin{align*}
    \minimizeEquationSt{g(s)}[s\geq 0],
  \end{align*}
  where
  \begin{equation*}
    g(s) = n^{-1}\sum_{i=1}^n \ell(m_i, sq) + \lambda s^2.
  \end{equation*}
  Convexity of $g$ follows from convexity of $\ell$ (see [Claim ??]).

  
  Because $g$ is convex, we can consider two cases: either the minimum is realized at the
  boundary, ie. $s^\star=0$, or there exists an optimal value $s^\star > 0$ such that
  $g'(s^\star)=0$. To derive a bound on $s^\star$, we can seek a value $\bar s$ such
  that for any $q$, $g'(\bar s)>0$ and therefore $\bar s > s^\star$.

  To do so, we first note that 
  \begin{align*}
    g'(s) &= \grad_s \left[n^{-1}\sum_{i=1}^n\ell(m_i,sq) + \lambda s^2\right]\\
          &= 2\lambda s - kn^{-1}\sum_{i=1}^n \grad_s \bar u(sr_iq^Tx_i+R_f(1-sq^Tx_i))\\
          &= 2\lambda s - kn^{-1}\sum_{i=1}^n (r_i-R_f)q^Tx_i \, \bar u'(sr_iq^Tx_i+R_f(1-sq^Tx_i)).
  \end{align*}
  Now, because $\|q\|_2=1$, we have $q^Tx_i\leq\|x_i\|_2\leq\xMax$. We also have
  $r_i-R_f\leq \bar r-R_f$ and $\bar u'\leq\gamma$, so that
  \begin{equation*}
    k\gamma\xMax(\bar r-R_f) \geq n^{-1}\sum_{i=1}^n (r_i-R_f)q^Tx_i \, \bar u'(sr_iq^Tx_i+R_f(1-sq^Tx_i)).
  \end{equation*}
  Therefore, with
  \begin{equation*}
    \bar s := \frac{k\gamma}{2\lambda}\xMax(\bar r-R_f),
  \end{equation*}
  for any $s>\bar s$,
  \begin{equation*}
    g'(s) \geq 0.
  \end{equation*}
  The previous inequality therefore implies that the norm $\|\qHat_n\|_2$ is at most $\bar
  s$. 
\end{proof}


\begin{claim}
\label{claim1}
 The following inequality holds:
\begin{equation*}
  |\trueRisk(\qStar) - \trueRisk(\hat q)| \leq (\bar r +R_f)\xMax\|\qStar - \qHat_n\|_2 + R_f.
\end{equation*}
\end{claim}

\begin{proof}
The following development holds:
\begin{align*}
  |\trueRisk(\qStar) - \trueRisk(\qHat_n)| &= |E_M[\ell(m,\qStar)] + E_M[-\ell(m,\qHat_n)]\\
                                           & = |E_M[\ell(m,\qStar) - \ell(m,\qHat_n)]|\\
                                           & \leq E_M[|\ell(m,\qStar) - \ell(m,\qHat_n)|].\\
  \intertext{Now, if we let $m=(x,r)$, using the Lipschitz property of $u$, with Lipschitz constant 
  $\gamma$, and the Hölder's inequality, we also have the following identity:}
  |\ell(m,\qStar) - \ell(m,\qHat_n)| &= |u(r\,\qStar^Tx + R_f(1-\qStar^{T}x)) - u(r\,\qHat_n^Tx + R_f(1-\qHat_n^{T}x))| \\
                                           & \leq \gamma|r\,\qStar^Tx + R_f(1-\qStar^Tx) - r\,\qHat_n^Tx - R_f(1-\qHat_n^Tx)|\\
                                           & = \gamma|r(\qStar-\qHat_n)^Tx + R_f(1-(\qStar-\qHat_n)^Tx)|\\
                                           & \leq \gamma|r(\qStar-\qHat_n)^Tx| + R_f|1-(\qStar-\qHat_n)^Tx|\\
                                           & \leq \gamma(\bar r\xMax \|\qStar-\qHat_n\|_2 + R_f(1+\xMax \|\qStar-\qHat_n\|_2))\\
                                           & = \gamma((\bar r + R_f)\xMax \|\qStar - \qHat_n\|_2 + R_f).
\end{align*}
Where we used the bounds on outcomes $x$ and $r$. And so we are left with the claimed
inequality. 
\end{proof}

\begin{claim}
  If the utility $u(r)$ is unbounded as $r\to\infty$ and the random return $R$ is a linear
  transformation of the random market state, ie. $R = t^T X$, then $\|\qStar\|_2$ is
  unbounded.
\end{claim}
\begin{proof}
  We first observe that, by definition, $\qStar$ is the minimal value of $E_M[\ell(m,q)]$, ie.
  \begin{align*}
    \qStar &= \argmax_q E_M[u(r\,q^{T}x + R_f(1-q^{T}x)]\\
    &\leq \argmax_q E_M[r\,q^Tx + R_f(1-q^Tx)]\\
    &= \argmax_q E_M[t^Txq^Tx] + R_fE_M[1-q^Tx]\\
    &= \argmax_q t^T E_M[xx^T] q + R_fE_M[1-q^Tx]\\
    &= \argmax_q t^T\Sigma q + R_fE_M[1-q^Tx],
  \end{align*}
with $\Sigma$ the covariance of $X$. It is therefore easy to see that if $(t^T\Sigma)_i$
is positive, we set $q_i=\infty$ and if $(t^T\Sigma)_i$ is negative, we set $q_i=-\infty$,
so that the total expression tends to infinity. [The general idea is here, but the proof
is far from perfect.]
\end{proof}
\end{document}


%%% Local Variables:
%%% mode: latex
%%% TeX-master: t
%%% End:
