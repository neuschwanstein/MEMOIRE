\documentclass[11pt]{article}

\newcommand{\ts}{\textsuperscript}
\newcommand{\figref}[1]{Fig.~\ref{#1}}
\usepackage{amsmath}
\usepackage{amsthm}
\usepackage{graphicx}
\usepackage{geometry}
\usepackage{subcaption}
\usepackage{bm}
\usepackage{hyperref}
\usepackage[retainorgcmds]{IEEEtrantools}
\usepackage{mathtools}
\usepackage{color}
\usepackage{marginnote}
\usepackage[utf8]{inputenc}


\DeclareMathOperator*{\argmax}{arg\,max}
\DeclareMathOperator*{\argmin}{arg\,min}
\DeclareMathOperator{\st}{s.t.}
\DeclareMathOperator{\epi}{epi}
\DeclareMathOperator{\diag}{diag}
\DeclareMathOperator{\dom}{dom}
\DeclareMathOperator{\tr}{tr}
\DeclareMathOperator*{\minimize}{minimize}
\DeclarePairedDelimiter\floor{\lfloor}{\rfloor}

\newcommand{\iso}{\simeq}
\newcommand{\dd}{\partial}
\newcommand{\real}{\bm R}
\newcommand{\trueRisk}{R_{\mathrm{true}}}

\newcommand{\starsection}{\vspace{1em}\begin{center}$\star\quad\star\quad\star$\end{center}\vspace{1em}}


\newcommand{\hilight}[1]{\colorbox{yellow}{#1}}
\let\oldmarginnote\marginnote
\renewcommand{\marginnote}[1]{\oldmarginnote{\footnotesize\emph{#1}}[0cm]}

\theoremstyle{plain}
\newtheorem{prop}{Proposition}
\newtheorem{thm}{Theorem}

\theoremstyle{definition}
\newtheorem*{deff}{Definition}
\newtheorem*{rem}{Remark}


\geometry{letterpaper}
\IEEEeqnarraydefcolsep{0}{\leftmargini}


\title{Portfolio Optimization in a Big Data Context}
\author{Thierry \textsc{Bazier-Matte}}
\date{Summer 2015}

\begin{document}
\maketitle

\paragraph{Notation.}
In the following, $\bm A$ (capital boldface) are assumed to represent a real subset of any
dimension, $A$ (capital case) represents random variables (or distributions) and $a$
(lower case) represents deterministic variables or realizations. $\real$ represents the
real set.

Let $M=(X,R)$ the \textsl{market} be an unknown distribution with support
$\bm M = \bm X\times \bm R \subseteq\real^{p+1}$, ie. numerically qualifiable, with
$(x,r) = m\sim M$ a \textsl{market observation}, consisting in one part \textsl{state}
$x\in\real^p$ and another part \textsl{outcome} $r\in\real$. Typically $x$ is a vector of
observations from various variable of interests, such as financial or economical news,
etc. Scalar $r$ in this article shall represent the return from a financial asset of
interest. Finally, let $M_n = \{M,\dots,M\}$ be a \textsl{random set} of $n$ (unrealized)
observations (with support $\bm M^n$). Therefore $\mu_n \sim M_n$ represents an iid sample
of $n$ market observations.

This article studies \textsl{linear investment decisions} $q^Tx$, with
$q\in \bm Q\subseteq\real^p$.

\begin{assumption}
  We suppose that observed returns $r$ are constrained by $|r| \leq \bar r$ with
  probability $1-\delta_r$ and that observed states $x$ are constrained by
  $\|x\|_2 \leq \xMax$ with probability $1-\delta_x$.
\end{assumption}

\begin{deff}
  Let $\ell:\bm M\times \bm Q\to\real$ be a \textsl{loss function} defined by
  \begin{equation*}
    \ell(m,q) = \ell(x,r,q) = -u(r\,q^{T}x + R_f (1 - q^{T}x)),
  \end{equation*}
  where $R_f$ is the risk free return rate and $u(r) = \min(r,\beta r)$, with $0<\beta<1$
  the risk aversion parameter. We also define the \textsl{cost function}
  $c:\real\times\bm R\to\real$ as
  \begin{equation*}
    c(p,r) = -u(pr + (1-p)R_f),
  \end{equation*}
  so that $\ell(x,r,q) = c(q^Tx,r)$. 
\end{deff}

\begin{deff}
  The \textsl{empirical risk} $\hat R: \bm M^n\times \bm Q \to \real$ associated with decision
  $q$ and market sample $\mu_n$ is given by
 \begin{equation*}
   \hat R_{\mu_n}(q) = n^{-1} \sum_{i=1}^n \ell(m_i,q).
 \end{equation*}
\end{deff}

\begin{deff}
  The \textsl{empirical decision algorithm} $\hat A_n:\bm M^n \to \bm Q$ associated with
  market sample $\mu_n$ is the optimal value of the problem
  \begin{equation*}
    \text{minimize}\quad\hat R_{\mu_n}(q) + \lambda\|q\|_2^2.
  \end{equation*}
\end{deff}

From now on, $\hat q_n := \hat A_n(\mu_n)$ the empirical decision associated with market sample
$\mu_n$ and $\hat Q_n := A_n(S_n)$ the random empirical decision, ie. $\hat q_n \sim \hat Q_n$.

\begin{deff}
  The \textsl{true risk} $\trueRisk:\bm Q\to\real$ associated with decision $q$ is given by
  \begin{equation*}
    \trueRisk(q) = E_M[\ell(m,q)].
  \end{equation*}
\end{deff}

\begin{deff}
  The \textsl{optimal decision} $\qStar$ is the optimal value of the problem
  \begin{equation*}
    \text{minimize}\quad \trueRisk(q) + \lambda\|q\|_2^2.
  \end{equation*}
\end{deff}


\section{Stability Definitions and Theorems}

\begin{deff}
  Let $\hat q_n=\hat A_n(\mu_n)$ and
  $\hat q_{n\backslash i}=\hat A_n(\mu_{n\backslash i})$, where $\mu_n$ and
  $\mu_{n\backslash i}$ only differs in their $i$\textsuperscript{th} observation, which
  has been redrawn from $M$ in the case of $\mu_{n\backslash i}$. The algorithm $\hat A_n$
  is said to have \textsl{uniform stability} $\alpha_n$ if, for any $m\sim M$,
  \begin{equation*}
    |\ell(m,\hat q_n) - \ell(m,\hat q_{n\backslash i})| \leq \alpha_n. 
  \end{equation*}
\end{deff}

\begin{deff}
  A loss function $\ell$ is $\sigma$-admissible if its cost function $c$ is convex with
  respect to $p$ the investment decision and the following holds for any $p_1,p_2$ and r:
  \begin{equation*}
    |c(p_1,r) - c(p_2,r)| \leq \sigma|p_1-p_2|.
  \end{equation*}
\end{deff}

\begin{rem}
  The loss function as defined above is $\sigma$-admissible with $\sigma=\bar r+R_f$.
\end{rem}

\begin{thm}
  If $\ell$ is $\sigma$-admissible and if, for any $x\in\bm X$, $\|x\|_2^2\leq\xMax^2$,
  then $\hat A_n$ has uniform stability with
  \begin{equation*}
    \alpha_n = \frac{\sigma^2 \xMax^2}{2\lambda n}.
  \end{equation*}
\end{thm}

\begin{proof}
  See Bousquet, Theorem 22. 
\end{proof}

We therefore conclude that $\hat A_n$ has uniform stability with
\begin{equation*}
  \alpha_n = \frac{(\bar r+R_f)^2\xMax^2}{2\lambda n}.
\end{equation*}

\begin{thm}
  If $\hat A_n$ has uniform stability $\alpha_n$ and the loss function is such that for
  any $m\sim M$ and any $\hat q_n=\hat A_n(\mu_n)$, $0\leq \ell(m,\hat q_n)\leq B_n$, then
  for any $\delta\in(0,1)$, the following bound holds with probability at least $1-\delta$
  over the random sample draw $\mu_n\sim M_n$:
  \begin{equation*}
    |\trueRisk(\qHat_n) - \hat R(\qHat_n)| \leq 2\alpha_n + (4n\alpha_n + B_n)\sqrt{\frac{\log(2/\delta)}{2n}}.
  \end{equation*}
\end{thm}

% \begin{proof}
%   See Bousquet, Theorem
% \end{proof}
\end{document}


%%% Local Variables:
%%% mode: latex
%%% TeX-master: t
%%% End:
