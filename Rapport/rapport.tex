\documentclass[11pt,fleqn]{article}

\newcommand{\ts}{\textsuperscript}
\usepackage{amsmath}
\usepackage{graphicx}
\usepackage{geometry}
\usepackage[parfill]{parskip}
\usepackage{bm}

\geometry{letterpaper}

\title{COMP 652 \\ End of Term Project}
\author{Thierry \textsc{Bazier-Matte}}
\date{April 2015}

\begin{document}
\maketitle

\section{Formal Problem}

\subsection{The Utility Function}

Let $U$ be the two-pieces linear utility function of an investor. If we suppose our investor
to be `rational', then there is a critical return rate $r_c$ at which lower returns on
investments have a higher effect on utility, and higher returns yield a lower effect on
utility. 

Mathematically, such a function would be of the form 
\begin{align*}
  U(r) &= \min[\beta(r-r_c), r-r_c]\\
       &= (r-r_c)^+ - \beta(r_c - r)^+
\end{align*}
$\beta>1$ represents the higher rate at which the utility decrease when exposed to a
return lower than $r_c$. We can, for simplification purposes let
\begin{equation*}
  r_c = R_f = \log(1.05)/365
\end{equation*}
the market risk free rate, taken here to be of 5\% per annum, but expressed as a daily
continuous return.

The optimization problem will therefore take the form
\begin{equation*}
  \max U(p)
\end{equation*}
where $p$ is the return on a portfolio.

\begin{figure}
  \centering
  \includegraphics[width=0.7\textwidth]{Utility}
  \caption{Utility function, with $\beta=2$ and $r_c=5\%$}
  \label{fig:utility}
\end{figure}

\subsection{Portfolio Composition}

We now consider a fictitous portfolio consisting of only two assets, the risk-free asset
(money market) and risky asset, such as a stock. At day $t$ we run a machine learning
algorithm to determine the optimal quantity $q^Tx_t$ to be invested in the risky asset. 

The portfolio return at time $t+1$  will then be of the form:
\begin{equation*}
  p_{t+1}(x_t,r_{t+1}) = r_{t+1}q^Tx_t + R_f(1-q^Tx_t).
\end{equation*}
Where $x_t$ is the feature vector at time $t$ for a given stock $s$ and $r_{t+1}$ is the
return of the stock at end of day $t$.

\subsection{Linear Program}


To setup the learning program of the portfolio compostion vector $\bm q$, we try to
optimize over $\bm q$ the average utility of the individual returns over all days and all
stocks considered by the problem:
\begin{align*}
  &\max_{\bm q} \frac{1}{N} \sum_{\text{stocks}}\sum_{\text{days}} U(p(\bm q))\\
  &\qquad=\max_{\bm q} \sum_{s,t} U(p(\bm q)-R_f)^+ - \beta (R_f - p(\bm q))^+\\
  &\qquad = \max_{\bm q} \sum_{s,t} (r_{s,t+1} \bm q^T \bm x_{s,t} - \bm q^T \bm
    x_{s,t})^+ -\beta(-r_{s,t+1}\bm q^T \bm x_{s,t}+\bm q^T\bm x_{s,t})^+\\
  &\qquad = \max_{\bm q} \sum_{s,t} ((r_{s,t+1}-1)\bm q^T\bm x_{s,t})^+
    -\beta((1-r_{s,t+1})\bm q^T\bm x_{s,t})^+
\end{align*}

This is a linear program optimization problem which can be reformulated as follows:
\begin{align*}
  \max_{\bm q,\bm\lambda,\bm\mu} &\hspace{0.5em} \sum_{s,t} \lambda_{s,t} - \beta\mu_{s,t}\\
  \mathrm{s.t.} &\hspace{0.5em} \lambda_{s,t} \geq (r_{s,t+1}-1)\bm q^T\bm x_{s,t}\\
  &\hspace{0.5em} \mu_{s,t} \geq (1-r_{s,t+1})\bm q^T\bm x_{s,t}\\
  &\hspace{0.5em} \lambda_{s,t}, \mu_{s,t} \geq 0
\end{align*}
Or, alternatively,
\begin{align*}
  \max_{\bm q,\bm \lambda,\bm \mu}  &\hspace{0.5em} \sum_{s,t} \lambda_{s,t} - \beta \mu_{s,t}\\
  \mathrm{s.t.} &\hspace{0.5em} \lambda_{s,t} - \mu_{s,t} + 2(1-r_{s,t+1})\bm q^T\bm x_{s,t} \geq 0\\
  &\hspace{0.5em} \lambda_{s,t}, \mu_{s,t} \geq 0
\end{align*}


\end{document}
