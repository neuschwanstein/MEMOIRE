\usepackage{amsmath}
%\usepackage{pdfcomment}
\usepackage{amsthm}
\usepackage{graphicx}
\usepackage{geometry}
\usepackage{subcaption}
\usepackage{bm}
\usepackage{hyperref}
\usepackage[retainorgcmds]{IEEEtrantools}
\usepackage{xparse}
\usepackage{mathtools}
\usepackage{color}
\usepackage{marginnote}
\usepackage[utf8]{inputenc}

% \setlength{\parindent}{0em}
\setlength{\parskip}{0.8em}

\DeclareMathOperator*{\argmax}{arg\,max}
\DeclareMathOperator*{\argmin}{arg\,min}
\DeclareMathOperator{\st}{s.t.}
\DeclareMathOperator{\epi}{epi}
\DeclareMathOperator{\diag}{diag}
\DeclareMathOperator{\dom}{dom}
\DeclareMathOperator{\tr}{tr}
\DeclareMathOperator*{\minimize}{minimize}
\DeclarePairedDelimiter\floor{\lfloor}{\rfloor}

\newcommand{\ts}{\textsuperscript}
\newcommand{\figref}[1]{Fig.~\ref{#1}}

\newcommand{\iso}{\simeq}
\newcommand{\dd}{\partial}
\newcommand{\real}{\bm R}
\newcommand{\trueRisk}{R_{\mathrm{true}}}
\newcommand{\uInv}{u^{-1}}
\newcommand{\qHat}{{\hat q}}
\newcommand{\qStar}{{q^\star}}

\newcommand{\starsection}{\vspace{1em}\begin{center}$\star\quad\star\quad\star$\end{center}\vspace{1em}}

% http://tex.stackexchange.com/questions/111551/recursive-multiple-subscript-and-superscript-with-xparse
\NewDocumentCommand{\minimizeEquation}{m >{\SplitList{,}}O{}}
{\begin{align}\text{minimize} & \quad #1\\ \text{subject to} \ProcessList{#2}{\stCommand}\end{align}}

\NewDocumentCommand{\stCommand}{m}
{& \quad #1}

\theoremstyle{plain}
\newtheorem{prop}{Proposition}
\newtheorem{thm}{Theorem}

\theoremstyle{definition}
\newtheorem*{deff}{Definition}
\newtheorem*{rem}{Remark}


\geometry{letterpaper}
\IEEEeqnarraydefcolsep{0}{\leftmargini}
%%% Local Variables:
%%% mode: latex
%%% TeX-master: "master"
%%% End:
