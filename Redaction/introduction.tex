\section{Introduction}
\label{sec:intro}

This document considers a two-asset portfolio, of which one is the risk-free asset,
yielding a constant return rate $R_f$, and the other being a risky asset $s$, typically a
stock, yielding a random return rate $r_{st}$ for each period $t$. The distribution
underlying this return is assumed to be unknown. We suppose that each risky asset $s$ can
be decribed daily by an \emph{information vector} $x_{st}$ containing potentially useful
information, such as technical, fundamental or news-related information. Furthermore, we
assume that the allocation of each asset of the portfolio $p_{st}$ can be fully determined
using a \emph{decision vector} $q$ based on past information of the previous returns and
of the previous information vectors. The allocation rule is the folowing: $q^Tx_{st}$ is
allocated to the risky asset and $1-q^Tx_{st}$ is allocated to the risk-free asset. Over
the period $t$, the portfolio $p_{st}$ consisting of asset $s$ will therefore yield a
return rate of:
\begin{equation}
  p_{st}(q) = r_{st}q^Tx_{st} + (1-q^Tx_{st})R_f.
\end{equation}

The question we now wish to ask is how the decision vector $q$ should be chosen. We assume
we have access to a training dataset $S_n$, comprising of $s$ different assets over
$t$% \marginnote{What's the difference between having $n$ points with having $s\times t$
  % points? For example, what if $s\gg t$ or the reverse?} 
periods, such that
$n = s\times t$. 
%%% Local Variables:
%%% mode: latex
%%% TeX-master: "master"
%%% End:
