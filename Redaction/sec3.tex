\section{Multi-asset Portfolio}

We now consider a multi-asset portfolio, made of $m$ risky assets and a risk-free
asset. Each of theses risky assets have information vector $x_{st}$, where
$s\in\{1,\ldots,m\}$ and $t\in\{1,\ldots,n\}$. Therefore, we suppose having access to a
sample dataset $S_n = \{X_1,\ldots,X_n\}$ where $X_i\in\real^{m\times p}$.

There are two ways to approach the multi-asset scenario. 


\subsection{First approach}

On one hand, we can define a single decision vector $q$ such that the return on a single
holdinq period would be defined by
\begin{equation}
  r^{T}Xq + (1 - 1^{T}Xq)R_f,
\end{equation}
where $r\in\real^{m}$ is the return vector of the $m$ stocks and $1^T$ is a contant vector
of ones in $\real^m$ whose role is simply to sum over the components of $Xq$, ie. sum over
the risky allocations of the portfolio. However, this method has the disadvantage that the
decision vector is applied for each stock in the same way, and so all their information
vectors are ultimately `averaged' over. Therefore, if some asset is more sensitive to an
information component than other assets, this knowledge might be lost if using this
method. 

Nevertheless, we can again define two different cost functions, which our optimal decision
vector would minimize when applied to our sample dataset $S_n$. Again, the cost can be
simply defined as
\begin{equation}
  \label{multiCost}
  c(p,r) = -U(r^{T}p + (1 - 1^{T}p)R_f).
\end{equation}
As discussed previously, this definition might have the disadvantage of providing an
unconstrained optimization problem when determining the optimal decision vector.

We are not limited to a single cost definition, and in fact, the multi-asset scenario
provides many legitimate cost definitions. Here the most `aggressive' one:
\begin{equation}
  c(p,r) =
  \begin{cases}
    \floor{U(\max r) - U(r^{T}p + (1-1^{T}p)R_f)} & \text{if } \max r>R_f\\
    \floor{U(R_f) - U(r^{T}p + (1-1^{T}p)R_f)} &\text{if } \max r\leq R_f.
  \end{cases}  
\end{equation}
This means that we expect (or conversely, that we suffer a lesser loss) when the total
return of our portfolio is near the best return achieved by our considered stocks. Instead
of $\max r$, we also could have chosen the average returns $\bar r$ during the period as
the target that we wish to achieve (provided that the average is superior to the risk-free
return). We could in fact chose any function mapping from a vector of returns $r$ to a
scalar as our global return objective.

In any case, the algorithm determining the decision vector $q$ would once be defined as
previously, ie. as \eqref{algo}.


\subsection{Second approach}

Under this second approach, instead of averaging over all information vector at the risk
of losing asset-specific information, we now use a decision matrix $Q$ of size $p\times
m$, whose $m$ columns are actually trained independently one to each other so that each
asset reacts differently from the rest. 

The allocation of each vector would then be $\diag(XQ)$, ie. the diagonal elements of the
$m\times{}m$ matrix $XQ$. The one-period return is therefore
\begin{equation}
  r^{T}\diag(XQ) + (1 - 1^{T}\diag(XQ))R_f.
\end{equation}

This method still has a disadvantage though, in that it won't consider the eventual
influence of information $x_j$ of asset $j$ to asset $j$, ie. each stock is trained
independantly one to each other.

\subsection{Third Approach}

Finally, we introduce a third approach with the objective of using all available data of
the market, available through $X\in\real^{m\times sp}$. However, we must be clear about
what information $X$ carries with it; its information is only bounded to different
assets, and therefore does not include `global' information with it.

Following a simple idea introduced by \cite{wsj}, we can add an unknown linear
transformation $W$ to $X$ so that $q$ can benefit from the whole information matrix, and
not only from the average of its columns. The total return using a transformation $W$ and
decision vector $q$ would therefore be expressed as
\begin{equation}
  r^{T}XWq + (1 - 1^{T}XWq)R_f.
\end{equation}

\subsection{Fourth Approach}

We now introduce a more general framework that actually encompasses all other methods
we've seen so far. In addition, we also add regularization terms so that the following
rules are respected :
\begin{itemize}
\item Given an asset $i$, we insist that the allocation scalar for $i$ relies more
  strongly on feature vector $x_i$ than on feature vectors from other assets in the
  market.
\item The market should be segmented in different industries, and our algorithm should
  partition the $m$ stocks in $k$ groups of similar nature. This would also impact how
  cross-influence of feature affect very weakly assets of other groups. 
\item Each asset $i$ has its own decision vectors $q_{ij}$, where $q_{ij}$ is the
  influence of asset $j$ on asset $i$. 
\end{itemize}

With this in mind we can now define a general allocation rule given feature vectors
$\{x_i\}$ for each asset in the market. As before, we consider a general inner product
$x_i^Tq_j$ as a mapping from the feature space of $i$ to the allocation scalar
space. Therefore, the total influence of the $m$ assets on $i$ can be expressed as 
\begin{equation}
  x_i^{T}q_{ii} + \sum_{j\neq i}x_j^{T}q_{ij}.
\end{equation}
We can reduce this expression by introducing a decision matrix $Q_i \in \real^{p\times{}m}$ defined by
\begin{equation}
  Q_i = \begin{pmatrix}[Q_i]_1 & \cdots & [Q_i]_m\end{pmatrix}.
\end{equation}
By using the same feature matrix $X$, the above expression is therefore equivalent to
\begin{equation}
  \tr(XQ_i),
\end{equation}
so that the total return will be given by
\begin{equation}
  \sum_{i=1}^m r_i \tr(XQ_i) + \left(1 - \sum_{i=1}^m\tr(X Q_i)\right)R_f.
\end{equation}

Now, as before the $Q_i$ matrices are unknown at first, but can be learned using an
optimization problem, where the objective is the average cost $\bar{c}$, defined by
\eqref{multiCost}. Let us, just for a moment, neglect the regularization term. Then, the
optimal matrices are $Q_i^\star$ are the solution to the following optimization problem:
\begin{align}
  \label{optProblemMulti}
  \minimize_{Q_1^\star,\dots,Q_m^\star} \quad & \frac{1}{n} \sum_{i=1}^{n}c(p_i,r_i)\\
  \st \quad & [p_i]_j = \tr(X_iQ_j).\nonumber
\end{align}

Now, there are some fundamental questions concerning this optimization problem which we'll
report for later. Indeed, before we start analyzing the theoretical guanrantees we can get
out of this problem, we would first like to add regularization terms to the objective,
with the hope of being compliants with the rules we've proposed above.

The first rule we proposed is certainly the most easy to implement. What it says is
basically that in the allocation scalar for asset $i$, the contribution from feature $i$
should be larger than contribution coming from feature of other assets. 

Mathematically, this would translate in this `soft' constraint:
\begin{equation}
  x_i^{T}[Q_i]_i \gg \sum_{j\neq i}x_j^{T}[Q_i]_j.
\end{equation}

In an optimization context, we can add the following expression to the objective:
\begin{equation}
%  +\lambda (\tr(XQ_i) - x_i^{T}[Q_i]_i)^2
  +\tr(XQ_i)^2 - \lambda(x_i^T[Q_i]_i)^2,
\end{equation}
with $\lambda \geq0$. In plain words, this means that we expect the whole allocation
scalar close to the contribution from $x_i^T[Q_i]_i$, with a regularization parameter
$\lambda$ to avoid all other terms $x_j^{T}[Q_i]_j$ falling to zero. 

Now, let's tackle the second constraint, where we expect the cross-contribution of a
feature vector to only affect a close number of assets. For example, we'd expect
technology assets to affect each other, but with little impact on energy assets. This
evidently reminds of $L1$ regularization, where the activation is non-smooth, and either
you're part of a group, or you're not. 

We can first start by defining the number of groups, or segments, we exepct the market to
have. We shall note this number of groups by $k$. 

Let us then state the problem formally. Let $\Xi \in \real^{m\times{}m}$ be the allocation
matrix, such that
\begin{equation}
  [\Xi]_{ij} = x_{j}^{T}[Q_i]_j.
\end{equation}
Then, the allocation vector $p$ for each of the $m$ assets corresponds to the sum of each
of the rows of $\Xi$. Therefore, $\Xi$ is invariant to reordering of individual rows. In
the process of optimization, we want to have to turn $\Xi$ in a $k$-block matrix, with
each block representing a certain sector. However, this \textit{block} requirement should
not be too hard, as in certain event, we could have assets with overlapping
sectors. Anyway, $L1$ sounds like a good idea, since this kind of regularization often
allows for features to be either `in' or `out'. In regularization, there's always two
opposing forces, and its role is to find balance between those two forces. 

So what are the two forces at play here? Let us consider for one moment a certain asset
$i$. Then, its allocation $p_i$, given a feature matrix $X$ whose rows can be expressed by
$x_j^{T}$, will be of the form
\begin{equation}
  p_i = \tr(XQ_i) = \sum_{j=1}^m x_j^{T}[Q_i]_j.
\end{equation}
The optimization algorithm, in order to find the optimal $Q_i$, no matter what the shape
of the utility is, will try to increase $p_i$ as much as possible if $r_i>0$, and
conversely, will try to decrease it if $r_i<0$. Let's pretend for a moment that $r_i>0$,
so that we're trying to drive $p_i$ as positive as possible. 

\starsection

With $\Xi$ defined as above, than we can state the optimal $\Xi$ as an optimization
problem of the following form:
\begin{align}
  \minimize\quad & -U(1^{T}\Xi r),
\end{align}
where $\Xi$ is a function of $m^2$ independant vectors
$q^1_1,\dots,q^1_m,\dots,q^m_1,\dots,q^m_m$, such that $\Xi$ can be expressed by
\begin{equation}
  \Xi = \begin{pmatrix}
    x_1^{T}q^1_1 & x_2^{T}q^1_2 & \cdots & x_m^{T}q^1_m \\\\
    x_1^{T}q^2_1 & x_2^{T}q^2_2 & \cdots & x_m^{T}q^2_m \\
    \vdots & \vdots & \ddots & \vdots\\
    x_1^{T}q^m_1 & x_2^{T}q^m_2 & \cdots & x_m^{T}q^m_m
    \end{pmatrix}.
\end{equation}

We can further decompose this expression. First we need to introduce a decision matrix
$Q\in\real^{m\times{}mp}$:
\begin{equation}
  Q =
  \begin{pmatrix}
    {q^1_1}^T & {q^1_2}^T & \cdots & {q^1_m}^T\\\\
    {q^2_1}^T & {q^2_2}^T & \cdots & {q^2_m}^T\\
    \vdots & \vdots &\ddots & \vdots\\
    {q^m_1}^{T} & {q^m_2}^{T} & \cdots & {q^m_m}^{T}
  \end{pmatrix},
\end{equation}
as well as a feature matrix $X$:
\begin{equation}
  X=\begin{pmatrix}
    x_1 & 0 & 0 & 0\\
    0 & x_2 & 0 & 0\\
    0 & 0   &\ddots & 0\\
    0 & 0 & 0 &x_m
  \end{pmatrix}
\end{equation}
where each $x_i\in\real^p$ should be understood as column vectors, so that
$X\in\real^{mp\times m}$. 

With these definitions, we have
\begin{equation}
  \Xi = QX.
\end{equation}
The allocation vector $p$ can also be expressed as 
\begin{equation}
  p^T = 1^T\Xi = 1^TQX,
\end{equation}
so that the total return with asset returns vector $r$ is 
\begin{equation}
  1^{T}QXr.
\end{equation}

This is the quantity we seek to maximize, or rather with which we seek to cost-maximize
(More on that later).

\starsection

Now what we want to do is to solve $Q$. For now, we can try to simply optimize $1^TQXr$,
with $Q$ the free variable. 

This problem is not bounded, for we can simply target positive $r_i$, and assign them
positive (infinite) $p_i$ weight, and conversely for negative $r_i$ to obtain an infinite
return. Therefore, we want to regularize the weight from $Q$, and we can do so by adding a
regularization of the form
\begin{equation}
  +\lambda \|QX\|
\end{equation}
for some appropriate matrix norm $\|\cdot\|$.