\section{Definitions and Bounds}

\subsection{Definitions Notation}

Most of the following notation and defintions follow directly from \cite{bousquet2002}.

Let $S_n$ be a set of $n$ vectors of $\real^p\times\real$ of the form:
\begin{equation*}
  S_n = \{(x_1,r_1),\ldots,(x_n,r_n)\}.
\end{equation*}
Each component of $S_n$ is a tuple $(x,r)$, where $x$ is the information vector and $r$ is
the observed return rate.

Using $S_n$, we wish to create a decision vector $q_{S_n}\in\real^p$ from which we can
make an investment decision $p$  when confronted with a random draw $d=(x,r)$.

\paragraph{Loss, cost and regret.}
We introduce the loss $\ell$ as being a function mapping from the space of the decision
vectors and full observations (including market observations and the return of the asset)
to a numerical quantity called the cost. Formally, given $q$ a decision vector and a
random draw $d=(x,r)$, the loss will be
\begin{equation*}
\ell(q,d) = c(q(x),r) = c(q^Tx,r).
\end{equation*}

Supposing an utility $U$, there are two different cost functions we can use. The first
one, and perhaps the most obvious is defined by
\begin{equation*}
  c(p,r) = -u(pr + (1-p)R_f).
\end{equation*}
If the utility is piecewise linear, then the cost is not bounded, and so an infinite
return would yield a negatively infinite cost. This means that risk-taking, ie. positions
with $|p|>1$, will be encouraged since they can in fact yield a negative cost. This also
means that with an unbounded utility, the algorithm minimizing the cost over the sample
must have a non-zero regularization term, otherwise the problem might be
unbounded. \textbf{Unless we add a constant piece if $r>\bar r$.}

We can also define another cost function that is always non-negative, that we shall call the
regret $\rho$ of the decision:
\begin{equation*}
  \rho(p,r) =
  \begin{cases}
    \lfloor u(r) - u(pr + (1-p)R_f)\rfloor & \text{if } r>R_f\\
    \lfloor u(R_f) - u(pr + (1-p)R_f)\rfloor &\text{if } r\leq R_f,
  \end{cases}
\end{equation*}
where by $\lfloor \cdot \rfloor$ we mean $\max(.,0)$. In the case of exponential
utilty, this cost function is actually equivalent to the one previously defined, but is
quite different in the case of an unbounded utility function, for example the piece-wise
linear one. Such a cost does not encourage risky position, since the cost is at least
0. This means that if $r>0$, then there's no point having $p>1$ instead of $p=1$ since
both will yield a zero cost position. 

\paragraph{Utility.}
There are two ways we can model our utility, and both are concave function, to represent a
risk-averse approach. Let $0<k<1$. The first utility is the \emph{linear utility} of the form
\begin{equation}
  \label{linearUtility}
  u(r) = k\min(r, \beta r)
\end{equation}
where $0\leq\beta\leq1$ is the slope of the utility for $r>0$. 

The other utility is the \emph{exponential utility}:
\begin{equation}
  \label{expUtility}
  u(r) = -k\exp(-\mu r),
\end{equation}
with $\mu > 0$. Scaling factor $k$ here is a reminder that utility is a fundamentally
unitless value, and that the utility can therefore be scaled to our convenience. We shall
however see later how we can remove $k$ from our framework to obtain `real' values.

\begin{figure}
  \centering
  \begin{subfigure}{.4\textwidth}
  \includegraphics[width=1.1\textwidth]{ExpULossAboveZero.pdf}
\end{subfigure}%
\begin{minipage}{.4\textwidth}
  \includegraphics[width=1.1\textwidth]{ExpULossBelowZero.pdf}
\end{minipage}
\end{figure}

\paragraph{Algorithm.}
We will be concerned with probabilistic confidence bounds on results produced using the
following algorithm, using dataset $S_n$. We shall denote this algorithm as $A_{S_n}$.
\begin{equation}
  \label{algo}
  q^\star = \argmin_{q\in\real^p}\frac{1}{n} \sum_{i=1}^{n} c(q^Tx_i,r_i) + \lambda\|q\|^2_2.
\end{equation}

\paragraph{Assumptions.}
We will assume that information vectors have been pre-processed and lie in a $X^2_{\max}$
radius ball. We also assume that the return rates observed are comprised within $[-\bar r,
\bar r]$. This last assumption will be relaxed. 


\subsection{Theoretical Optimal Value [Relevant?]}

Before we go on to derive an algorithm for the in-sample optimal allocation, we first
derive some theoretical results for an optimal allocation scalar $p^\star$, provided the
distribution of the returns is priorly known. 

Formally, we suppose that this distribution is characterized by its CDF $F$. We then wish
to minimize the expected cost $c(p,r)$, where $p$ is a variable and $r$ a random return
rate. Since,
\begin{equation*}
  E[c(p,r)] = \int_{-\infty}^{\infty} c(p,r)\,dF(r),
\end{equation*}
the optimal allocation $p^\star$ is therefore the solution to
\begin{equation*}
  \frac{d}{dp} \int_{-\infty}^{\infty} c(p,r)\,dF(r) = 0.
\end{equation*}

However, our theoretical assumptions are fundamentally different from this one, since we
consider a joint distribution $\bm D= \bm X \times \bm R$, so that returns are dependant
of a draw from $\bm X$, whereas it's assumed here that decision can only be made on the
basis of the returns. 

\subsection{Definitions, Theorems and Tools}
We now state a certain number of definitions and theorems that will be of great importance
in the analysis on the results produced by the algorithm mentioned above. These
definitions and theorems are taken verbatim from \cite{bousquet2002}.

\begin{deff}
  An algorithm $A$ has \textit{uniform stability} $\beta$ with respect to the loss
  function $\ell$ if, for all $S_n\in\bm D^n$ and $i\in\{1,\ldots,n\}$, the following
  holds:
  \begin{equation*}
    \|\ell(A_{S_n},.) - \ell(A_{S^{\backslash i}_n},.)\|_{\infty} \leq \beta_n,
  \end{equation*}
  or, equivalently,
  \begin{equation*}
    \sup_{d\in\bm D}|\ell(A_{S_n},d) - \ell(A_{S^{\backslash i}_n},d)| \leq \beta_n.
  \end{equation*}
  Here, $S^{\backslash i}_n$ means the set $S_n$ with the $i$th data point removed.

  Furthermore, $A$ is \textit{stable} when $\beta_n = O(1/n)$.
\end{deff}

\begin{deff}
  A loss function $\ell$ is \textit{$\sigma$-admissible} if the associated cost function
  $c$ is convex with respect to its first argument and the following condition holds for
  any $p_1,p_2$ and $r$:
  \begin{equation*}
    |c(p_1,r) - c(p_2,r)| \leq \sigma |p_1 - p_2|.
  \end{equation*}
  It is easy to see that if $\dom c(\cdot,r)$ is bounded and $c(\cdot,r)$ does not reach
  infinity, then its loss function is $\sigma$-admissible.
\end{deff}

\begin{thm}
  \label{thm1}
  Let $F$ be a reproducing kernel Hilbert space with kernel $\kappa$ such that
  $\forall x\in X$, $\kappa(x,x) \leq \kappa^2 <\infty$. If $\ell$ is $\sigma$-admissible
  with respect to $F$, then the learning algorithm defined by
  \begin{equation}
    \label{eq:above3}
    A_S = \argmin_{g\in F}\frac{1}{n}\sum_{i=1}^n \ell(g,d_i) + \lambda\|g\|^2_k
  \end{equation}
  has uniform stability $\alpha_n$ with respect to $\ell$ with
  \begin{equation*}
    \alpha_n \leq \frac{\sigma^2 \kappa^2}{2\lambda n}.
  \end{equation*}
\end{thm}

\begin{deff}
  The \emph{true risk} with respect to algorithm $A$ and set $S_n$ is defined as
  \begin{equation*}
    R_{\text{true}}(A,S_n) = E_d[\ell(A_{S_n},d)],
  \end{equation*}
  which is, in plain words, the expected loss incured when applying the algorithm created
  from training set $S_n$ in the wild, ie. out of sample.
\end{deff}

\begin{deff}
  The \emph{empirical risk} with respect to algorithm $A$ and set $S_n$ is defined as
  \begin{equation*}
    \hat R(A,S_n) = \frac{1}{n} \sum_{i=1}^n \ell(A_{S_n},d_i),
  \end{equation*}
  which is, in plain words, the average cost incured by our model over all the training
  set.
\end{deff}

\begin{thm}
  \label{thm2}
  Let $A$ be an algorithm with uniform stability $\alpha_n$ with respect to a loss
  function $\ell$ such that $0\leq\ell(A_{S_n},d)\leq M$ for all $d=(x,r)\sim D$ and all sets
  $S_n$ of size $n$. Then for any $n\geq1$ and any $\delta\in(0,1)$, the following bound
  holds with probability at least $1-\delta$ over the random draw of the sample $S_n$:
  \begin{equation*}
    |\trueRisk(A,S_n) - \hat R(A,S_n)| \leq 2\alpha_n + (4n\alpha_n + M) \sqrt{\frac{\log(2/\delta)}{2n}}.
  \end{equation*}
\end{thm}

\subsection{Remarks and applications to our algorithm}

In this section we shall derive bounds on the proposed algorithms concerned with the
difference between the in-sample average cost using the decision vector determined by our
algorithm with the (theoretical) expected cost over the whole distribution when using the
same decision vector. 

The analysis will be done in two parts, one for each of our cost functions: first with
$c=-U$ (the canonical cost) and second with the cost function as the regret at each
decision. Both are quite similar in their results. 

\subsubsection{Canonical cost.}
We first consider the case where 
\begin{equation*}
  c(p,r) = -k u(pr + (1-p)R_f).
\end{equation*}

\paragraph{$\sigma$-admissibility.}  The canonical cost is $\sigma$-adimissible, with 
\begin{equation}
  \label{linearSigmaAdmissibility}
  \sigma = k(\bar r+R_f)
\end{equation}
for the linear utility and 
\begin{equation}
  \label{expSigmaAdmissibility}
  \sigma = k (\bar r+ R_f) \mu\exp\mu\bar r
\end{equation}
for the exponential utility.

\begin{proof}We first remark that for both forms of $U$, $c$ is a convex function of $p$
  with $r$ fixed, since $U$ is concave.

Next, the expression $|c(p_1,r)-c(p_2,r)|$ reduces to
\begin{equation}
  \label{eq:above1} 
  k|u(p_1r + (1-p_1)R_f) - u(p_2r + (1-p_2)R_f|.
\end{equation} Now because $r\in[-\bar r,\bar r]$, $U$ is Lipschitz continuous on its
domain, and so \eqref{eq:above1} is bounded by
\begin{equation}
  \label{eq:above2} 
  \alpha k|p_1r + (1-p_1)R_f - (p_2r + (1-p_2)R_f)| = \alpha k|p_1-p_2||r-R_f|
\end{equation} 
where
\begin{equation*} 
  \alpha = \sup_{r\in[-\bar r,\bar r]} |U'(r)|.
\end{equation*}

In the linear case, the derivative is piecewise constant, and is set to 1 on for returns
below $r_c$, so that $\alpha=1$. In the exponential case, $U'(r) = \mu\exp\mu r$, and $\alpha
= \mu\exp \mu \bar r$.

The bound \eqref{eq:above2} must hold for any $r$. The expression $|r-R_f|$ will reach
its largest value at $r=-\bar r$, since $R_f$ is assumed to be non-negative, and therefore
\begin{equation*} 
  |c(p_1,r) - c(p_2,r)| \leq \alpha k(\bar r + R_f)|p_1-p_2|,
\end{equation*} 
which shows that $\ell$ is $\sigma$-admissible with the canonical cost,
with $\sigma=\alpha k(\bar r+ R_f)$.
\end{proof}

We also remark that the utility is actually invariant to scaling, ie. $\tilde{U} = kU$
where $k$ is positive yields the same relative utility. The $\sigma$ bound can therefore
be as tight as desired, since it can be scaled by an arbitrary positive constant, and the
algorithm will yield the same result.

\paragraph{Algorithmic stability.}
Because the loss function is $\sigma$-admissible, the conditions for Theorem \ref{thm1}
are respected: if we take the kernel on the space $\real^p$ as the regular inner product,
then $\kappa(x,x)\leq X^2_{\max}$. Therefore, using results
\eqref{linearSigmaAdmissibility} and \eqref{expSigmaAdmissibility}, our algorithm has
algorithmic stability with:
\begin{equation}
  \label{algoStabilityLinear}
  \alpha_n \leq \frac{k^2(\bar r+R_f)^2X^2_{\max}}{2\lambda n}
\end{equation}
 with linear utility and
\begin{equation*}
   \alpha_n \leq \frac{k^2 \mu^2 \exp(2\mu\bar r)(\bar r + R_f)^2 X^2_{\max}}{2\lambda n}
\end{equation*}
in the case of exponential utility.

% \marginnote{How can we leverage $k$?}
% As discussed above, $k>0$ is a scaling term we can add to the utility function, with
% which we can actually have algorithmic stability as tight as needed.

\paragraph{Bounds on the allocation scalar.}
We now wish to derive a bound on the value the allocation scalar $p$ can take when
applying algorithm \eqref{algo}.

To do so, let us first rewrite algorithm \eqref{algo} where instead of optimising over the
decision vector $q$, we optimise on its direction $q$ and its scale $s\geq0$. Algorithm
\eqref{algo} therefore becomes \minimizeEquation%
{\frac{1}{n}\sum_{i=1}^n c(s\,q^T x_i, r_i) + \lambda s^2}%
[s\geq0\\,\|q\|_2=1]

The optimization variables are $q$ and $s$, and we can solve each variable independantly
of the other one. Therefore, for any $q$ we can define a convex function $g(s)$ which
becomes the objective: \minimizeEquation{g(s)}[s\geq0] where
\begin{equation*}
  g(s) = \frac{1}{n} \sum_{i=1}^n c(s\,q^T x_i, r_i) + \lambda s^2.
\end{equation*}

Because $g$ is convex, we can consider two cases: either the minimum is realized at the
boundary $s^\star=0$, or there exists an optimal value $s^\star > 0$ such that
$g'(s^\star) > 0$. To derive a bound on $s^\star$, we can seek a value $s_{\max}$ such
that for any $q$, $s_{\max} > s^\star$, and therefore $g'(s_{\max}) > 0$.

To do so, we first note that
\begin{align*}
    g'(s) &= \frac{\dd}{\dd s}\left[\frac{1}{n} \sum_{i=1}^n c(s\, q^{T}x_i, r_i) + \lambda s^2\right]\\
          &= 2\lambda s + \frac{1}{n} \sum_{i=1}^n q^{T}x_i \left. \frac{\dd c(p_i,r_i)}{\dd p_i} \right|_{p_i =
            s\,q^{T}x_i} \\
          &= 2\lambda s - \frac{k}{n} \sum_{i=1}^n q^{T}x_i \left. \frac{\dd}{\dd p_i} u(p_i r_i +
            (1-p_i)R_f) \right|_{p_i = s\,q^{T}x_i}\\
          &= 2\lambda s - \frac{k}{n} \sum_{i=1}^n q^{T}x_i (r_i-R_f) \left. U'(p_i r_i +
            (1-p_i)R_f) \right|_{p_i = s\,q^{T}x_i}.
\end{align*}

In particular, in the case of the linear utility \eqref{linearUtility}, because
$q^Tx_i\leq X_{\max}$, $r_i-R_f\leq \bar r - R_f$ and $U'\leq 1$, we have
\begin{equation*}
 k X_{\max}(\bar{r}-R_f) \geq \frac{k}{n} \sum_{i=1}^n q^{T}x_i (r_i-R_f) \left. U'(p_i r_i +
    (1-p_i)R_f) \right|_{p_i = s\,q^{T}x_i}.
\end{equation*}

Therefore, with
\begin{equation}
  \label{scalarBound}
  \bar s := \frac{k}{2\lambda}X_{\max}(\bar r-R_f),
\end{equation}
we have for any $s>\bar s$
\begin{equation*}
  g(s) > 0.
\end{equation*}
Equation \eqref{scalarBound} therefore gives a bound on the scale of the optimal
decision vector, ie. $\|q^\star\|_2 \leq \bar s$. From this bound, we can finally derive a
bound on the optimal allocation scalar
\begin{equation}
  \label{allocationBound}
  |p_i| = |q^{T}x_i|\leq \frac{k}{2\lambda} X^2_{\max}(\bar r -R_f).
\end{equation}

\paragraph{Generalization bound for linear utility.}
Using Theorem \ref{thm2}, we can derive an out-of-sample bound on the result produced by
our algorithm for each utility forms that holds with probability $1-\delta$, for
$0<\delta<1$. The $M$ value stated by Theorem \ref{thm2} corresponds to the highest loss
we can incur on the domain, which happens when $p$ is at its largest and $r_i$ at its
lowest.

With linear utility, using result \eqref{allocationBound}, we have
\begin{align*}
  M &\leq c(k X^2_{\max}(\bar r - R_f)/2\lambda, -\bar r)\\
    &=-k u\left(\frac{k}{2\lambda} X^2_{\max}(R_f+\bar r)(R_f-\bar r) + R_f\right)\\
    &=\frac{k^2}{2\lambda} X^2_{\max}(\bar r + R_f)(\bar r - R_f) - k R_f,
\end{align*}
which leads, using result \eqref{algoStabilityLinear}, to
\begin{equation*}
  M \leq n\alpha_n \frac{\bar r - R_f}{\bar r + R_f} - kR_f.
\end{equation*}
Therefore, we have the following out of sample generalization bound for linear utility:
\begin{equation}
  \label{boundWithK}
  |\trueRisk(\hat{q}) - \hat{R}(\hat{q})| \leq 
  2\alpha_n + \left(4n\alpha_n + n\alpha_n\frac{\bar r - R_f}{\bar r +
      R_f} - kR_f\right)
  \sqrt{\frac{\log(2/\delta)}{2n}}.
\end{equation}
  % Using Theorem \ref{thm2}, we can derive an out-of-sample bound on the result produced by
  % our algorithm for each utility forms that holds with probability $1-\delta$, for
  % $0<\delta<1$. The $M$ value stated by Theorem \ref{thm2} corresponds to the highest loss
  % we can incur on the domain, which happens when $p=1$ and $r=-\bar r$, ie.
  % \begin{equation}
  %   M = c(1,-\bar r) = -u(\bar r).
  % \end{equation}

  % In the linear utility case, 
  % \begin{align}
  %   &|\trueRisk(A,S_n) - \hat{R}(A,S_n)| \leq 2\alpha_n + (4n\alpha_n +
  %     M)\sqrt{\frac{\log(2/\delta)}{2n}}\\
  %   &\qquad \leq \frac{k^2(\bar r+R_f)^2X^2_{\max}}{\lambda n} + \left(\frac{2k^2(\bar r+R_f)^2X^2_{\max}}{\lambda} + M\right)\sqrt{\frac{\log(2/\delta)}{2n}},
  % \end{align}
  % and in the exponential utility case, 
  % \begin{align}
  %   &|\trueRisk(A,S_n) - \hat{R}(A,S_n)| \leq 2\alpha_n + (4n\alpha_n +
  %     M)\sqrt{\frac{\log(2/\delta)}{2n}}\\
  %   &\qquad \leq \frac{k^2 \exp(2\mu\bar r)(\bar r + R_f)^2 X^2_{\max}}{\lambda n} + \left(\frac{2k^2 \exp(2\mu\bar r)(\bar r + R_f)^2 X^2_{\max}}{\lambda} + M\right)\sqrt{\frac{\log(2/\delta)}{2n}}.
  % \end{align}

\paragraph{Bound characterization in terms of returns.}
While useful, the bound \eqref{boundWithK} does have the disadvantage of being expressed
in terms of $k$ (since $\alpha_n \sim k^2$), wich is in itself an artificial variable
representing the invariance of the utility to scaling, ie. $\tilde u \simeq k u$. We would
therefore like to get rid of this quantity to obtain a bound on
\begin{equation*}
  u^{-1}(|\trueRisk(\hat q) - \hat R(\hat q)|),
\end{equation*}
which can be interpreted in `units of returns'. 

% \subparagraph{Assumption.} In the following discussion, we will assume that $0<k<1$, so
% that $k^2<k$. 

To do so, we can first write down the form of $u^{-1}$ from \eqref{linearUtility}:
\begin{equation*}
  u^{-1}(t) = \frac{1}{k}\max(t,t/\beta).
\end{equation*}
Let us also define 
\begin{equation*}
  \gamma_n := \frac{\alpha_n}{k^2}. 
\end{equation*}

Then, \eqref{boundWithK} can be rewritten as
\begin{align}
  |\trueRisk(\hat{q}) - \hat{R}(\hat{q})| &\leq k^2\left(
  2\gamma_n + \left(4n\gamma_n + n\gamma_n\frac{\bar r - R_f}{\bar r +
      R_f}\right) \sqrt{\frac{\log(2/\delta)}{2n}}\right) \nonumber \\
  \label{newKBound}
  &\leq k\left(
  2\gamma_n + \left(4n\gamma_n + n\gamma_n\frac{\bar r - R_f}{\bar r +
      R_f}\right) \sqrt{\frac{\log(2/\delta)}{2n}}\right).
\end{align}
None of the expression within the outher brackets depend on $k$. 

Now, because $\uInv$ is convex and non decreasing, if $a<b$, then $\uInv(a)\leq\uInv(b)$,
from which it follows that
\begin{equation*}
  \uInv(|\trueRisk(\hat{q}) - \hat{R}(\hat{q})|) \leq \uInv\left(k\left(
  2\gamma_n + \left(4n\gamma_n + n\gamma_n\frac{\bar r - R_f}{\bar r +
      R_f}\right) \sqrt{\frac{\log(2/\delta)}{2n}}\right)\right).
\end{equation*}
Therefore, by the definition of $\uInv$, 
\begin{equation*}
  \uInv(|\trueRisk(\hat{q}) - \hat{R}(\hat{q})|) \leq 
  \frac{\gamma_n}{\beta}\left[2+ n\left(4 + \frac{\bar r - R_f}{\bar r +
      R_f}\right) \sqrt{\frac{\log(2/\delta)}{2n}}\right].
\end{equation*}

% Next, we derive some results needed to transform the bound \eqref{newKBound}. First,
% because $u^{-1}$ is convex and non decreasing, if $a\leq b$, then 
% \begin{equation*}
%   u^{-1}(a)\leq u^{-1}(b).
% \end{equation*}
% Also, if $\lambda$ is a supergradient to $u$ at $0$, ie. $u(r) \leq \lambda
% r$, then $r \leq u^{-1}(\lambda r) \leq \lambda u^{-1}(r)$, ie.
% \begin{equation*}
%   \frac{r}{\lambda} \leq u^{-1}(r).
% \end{equation*}
% Which follows from the convexity of $u^{-1}$, ie. we have
% $u^{-1}(\lambda t) \leq \lambda u^{-1}(t)$.

% For simplicity, let's rewrite \eqref{newKBound} as
% \begin{equation*}
%   Q \leq kB. 
% \end{equation*}
% We then have
% \begin{equation*}
%   \uInv(Q) \leq \uInv(kB)
% \end{equation*}
% and
% \begin{equation*}
%   B \leq \uInv(kB).
% \end{equation*}
% \textsl{
% We want to put $Q$ and $\uInv(B)$ in relation. Ideally we would have a bound of the form
% \begin{equation*}
%   Q \leq A + C\uInv(B).
% \end{equation*}
% }

















\paragraph{Absolute bound.}
Finally, we derive a measure of performance when we compare $\hat q=A_{S_n}$ with the true
optimal $q^\star$ over the distribution, and how each perfom when applied on the true risk
of the distribution\ldots

\subsubsection{Regret cost.}

We now turn our attention toward the regret cost, ie.
\begin{equation*}
  c(p,r) = \floor{u(\max(r,R_f)) - u(pr + (1-p)R_f)}.
\end{equation*}
This type cost is different from the canonical cost in that to does not encourage `risky'
position for which $0<p<1$, since there's no gain in over-investing our portfolio in a
winning asset. Conversely, there's no gain in shorting a losing asset. Such a cost could
therefore be more suitable in a conservative environment, even if the results produced by
the algorithm $A$ might very well yield a position $0<p<1$.

\begin{rem}[$\sigma$-admissibility]
  The $\sigma$-admissibility of the regret cost follows closely from the
  $\sigma$-admissibility of the canonical cost derived above. Indeed, we remark that,
  provided that $c(p_1,r), c(p_2,r) > 0$, then the expression $|c(p_1,r) - c(p_2,r)|$
  reduces to 
  \begin{equation*}
    |u(p_1r + (1-p_1)R_f) - u(p_2r + (1-p_2)R_f)|,
  \end{equation*}
  wich has exactly the same form as \eqref{eq:above1}, and therefore yields the same
  $\sigma$ bound. Now for the case where, without loss of generality $c(p_2,r) = 0$. Let
  $\tilde c$ be defined as $c$, but without $\floor{\cdot}$, so that
  $\floor{\tilde{c}(p,r)} = c(p,r)$. In particular $\tilde c(p_2,r)<0$ and so
  \begin{align*}
    |c(p_1,r) - c(p_2,r)| &= |\floor{\tilde c(p_1,r)} - \floor{\tilde c(p_2,r)}|\\
    &\leq |\tilde c(p_1,r) - \tilde c(p_2,r)|\\
    &= |u(p_1r + (1-p_1)R_f) - u(p_2r + (1-p_2)R_f)|,
  \end{align*}
  which has again the the same form as \eqref{eq:above1}.

  This means that the loss defined with the regret cost has the same
  $\sigma$-admissibility as the loss defined with the canonical cost, namely 
  \begin{equation*}
    \sigma = k(\bar r + R_f)
  \end{equation*}
  in the case of linear utility and 
  \begin{equation*}
    \sigma = k(\bar r+R_f) \exp(\mu\bar r)
  \end{equation*}
  in the case of exponential utility. 
\end{rem}

\paragraph{Algorithmic Stability}
  Because this regret cost has the same $\sigma$-admissibility as the canonical cost, then
  the algorithmic stability also remains the same for both algorithm, ie.
  \begin{equation*}
    \alpha_n \leq \frac{k^2(\bar r+R_f)^2X^2_{\max}}{2\lambda n}
  \end{equation*}
  with linear utility and
  \begin{equation*}
    \alpha_n \leq \frac{k^2 \exp(2\mu\bar r)(\bar r + R_f)^2 X^2_{\max}}{2\lambda n}
  \end{equation*}
  in the case of exponential utility.


\begin{rem}[Generalization Bound]
  Likewise, the only difference here with the canonical cost is the value of the $M$
  variable of Theorem \ref{thm2}, defined as the highest cost on the domain. In the case
  here, this value is again reached for an investment $p=1$ on a losing asset at $r=-\bar
  r$, so that,
  \begin{equation*}
    M = \floor{u(R_f) - u(\bar r)}.
  \end{equation*}

  We are left with the same expression for the generalization bounds, in the linear
  utility case,
  \begin{align*}
    &|\trueRisk(A,S_n) - \hat{R}(A,S_n)| \leq 2\alpha_n + (4n\alpha_n +
      M)\sqrt{\frac{\log(2/\delta)}{2n}}\\
    &\qquad \leq \frac{k^2(\bar r+R_f)^2X^2_{\max}}{\lambda n} + \left(\frac{2k^2(\bar r+R_f)^2X^2_{\max}}{\lambda} + M\right)\sqrt{\frac{\log(2/\delta)}{2n}},
  \end{align*}
  and in the exponential utility case, 
  \begin{align*}
    &|\trueRisk(A,S_n) - \hat{R}(A,S_n)| \leq 2\alpha_n + (4n\alpha_n +
      M)\sqrt{\frac{\log(2/\delta)}{2n}}\\
    &\qquad \leq \frac{k^2 \exp(2\mu\bar r)(\bar r + R_f)^2 X^2_{\max}}{\lambda n} + \left(\frac{2k^2 \exp(2\mu\bar r)(\bar r + R_f)^2 X^2_{\max}}{\lambda} + M\right)\sqrt{\frac{\log(2/\delta)}{2n}}.
  \end{align*}
\end{rem}

\subsection{True Optimal Bound}

One bound that is still of interest is the absolute difference between the in-sample cost
using in-sample decision $\hat R(A,S_n)$ and the expected cost over the unknown
distribution $D$ (ie. the true risk) using the unknown optimal decision $q^\star$ defined
by
\begin{equation*}
  q^\star = \argmin_{\{q:X\subset\real^p\to\real\}}\trueRisk(q).
\end{equation*}

The true optimal bound is therefore expressed by
\begin{equation*}
|\trueRisk(q^\star) - \hat R(A,S_n)|.  
\end{equation*}
We can then apply the triangle inequality to bound this expression by the generalization
bound plus $|\trueRisk(q^\star) - \trueRisk(\hat q)|$. 

\textit{However, whereas \cite{rudin2015} could bound this expression using a theorem from
  quantile regression theory, I'm here at loss (no pun intended...)}





\paragraph{Things to consider.}

Here are some points that should be considered.

\begin{itemize}
\item We have assumed so far we were dealing with datasets taken iid. But in a financial
  setting, this is not necessarily the case, as influence of features might be changing
  with time. We would then have independant, but not necessarily identical draws. We could
  (and should) however asume that underlying probability distribution is changing
  \textsl{smoothly} (moment-generating functions, Kalman filters, change detection, etc.)
\item In the newsvendor article, the generalization bounds actually hold with probability
  $1-\delta - n\gamma$, if we assume $\bar D$ (the highest considered demand) holds with
  probability $1-\gamma$. It is only mentioned in the non-regularized case
  (\cite{rudin2015}, p.~14). A reference would be needed, as the $n\gamma$ terms grows up
  to be of no use [Reference needed !]
\item We are also considering discrete quantities, updated daily. Our model should
  encompass quantities that might change during the day.
\item A section on how data should be normalized must also be added. This is essential to
  go from theory to pratical applications. Hint: SVD.
\item Not to forget: the $p$ features should also include higher order terms of the raw
  features, ie. $x_i^2$ and $x_ix_j$ so we can capture a second order Taylor approximation
  (see \cite{rudin2015}, p.~6). Instead of $p$ features, we would then have $p^2+p$
  features. 
\item The fact we are constraining features to live within a $X_{\max}{}$ radius, ie.
  $\|x\|_2\leq X_{\max}$, is something that we would like to relax. (Like changing the norm,
  or perhaps the metric?)
\end{itemize}
%%% Local Variables:
%%% mode: latex
%%% TeX-master: "master"
%%% End:
