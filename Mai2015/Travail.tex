\documentclass[11pt]{article}

\newcommand{\ts}{\textsuperscript}
\newcommand{\figref}[1]{Fig.~\ref{#1}}
\usepackage{amsmath}
\usepackage{amsthm}
\usepackage{graphicx}
\usepackage{geometry}
\usepackage{subcaption}
\usepackage{bm}
\usepackage{hyperref}
\usepackage[retainorgcmds]{IEEEtrantools}
\usepackage{mathtools}
\usepackage{color}
\usepackage{marginnote}
\usepackage[utf8]{inputenc}


\DeclareMathOperator*{\argmax}{arg\,max}
\DeclareMathOperator*{\argmin}{arg\,min}
\DeclareMathOperator{\st}{s.t.}
\DeclareMathOperator{\epi}{epi}
\DeclarePairedDelimiter\floor{\lfloor}{\rfloor}

\newcommand{\iso}{\simeq}
\newcommand{\dd}{\partial}
\newcommand{\real}{\bm R}


\newcommand{\hilight}[1]{\colorbox{yellow}{#1}}
\let\oldmarginnote\marginnote
\renewcommand{\marginnote}[1]{\oldmarginnote{\footnotesize\emph{#1}}[0cm]}

\theoremstyle{plain}
\newtheorem{prop}{Proposition}
\newtheorem{thm}{Theorem}

\theoremstyle{definition}
\newtheorem*{deff}{Definition}
\newtheorem*{rem}{Remark}


\geometry{letterpaper}
\IEEEeqnarraydefcolsep{0}{\leftmargini}

\title{The Big Data Newsvendor Problem in a Portfolio Optimization Context}
\author{Thierry \textsc{Bazier-Matte}}
\date{Summer 2015}

\begin{document}
\maketitle

\begin{abstract}
  Following \cite{rudin2015}, we provide a portfolio optimization method based on machine
  learning methods.
\end{abstract}


\section{Introduction}
\label{sec:intro}

\marginnote{Maybe considerations about the length of the period should be added? For
  example, it's not specified what's the period length of $R_f$.}  This document considers
a two-asset portfolio, of which one is the risk-free asset, yielding a constant return
rate $R_f$, and the other being a risky asset $s$, typically a stock, yielding a random
return rate $r_{st}$ for each period $t$. We suppose that each risky asset $s$ can be
decribed daily by an \emph{information vector} $x_{st}$ containing potentially useful
information, such as technical, fundamental or news-related information. Furthermore, we
assume that the allocation of each asset of the portfolio $p_{st}$ can be fully determined
using a \emph{decision vector} $q$. The allocation rule is the folowing: $q^Tx_{st}$ is
allocated to the risky asset and $1-q^Tx_{st}$ is allocated to the risk-free asset. Over
the period $t$, the portfolio $p_{st}$ consisting of asset $s$ will therefore yield a
return rate of:
\begin{equation}
  p_{st}(q) = r_{st}q^Tx_{st} + (1-q^Tx_{st})R_f.
\end{equation}

% We further suppose that the \emph{utility} derived from a return rate $p$ can be described
% using a concave two-pieces linear function of the following form:
% \begin{equation}
%   U(p) = p-r_c + \min((\beta-1)(p-r_c),0),
% \end{equation}
% where $r_c$ is an arbitrary \emph{critical rate}. We impose the condition $0<\beta<1$, so
% that the utility derived from rates inferior to the critical rate $r_c$ increases more
% sharply than from rates above $r_c$. Typically, the critical rate could be 0, but it could
% also be $R_f$. This is up to the investor.


The question we now wish to ask is how the decision vector $q$ should be chosen. We assume
we have access to a training dataset $S_n$, comprising of $s$ different assets over
$t$\marginnote{What's the difference between having $n$ points with having $s\times t$
  points? For example, what if $s\gg t$ or the reverse?} periods, such that
$n = s\times t$. % Given such a training set, we then define the \emph{optimal decision
%   vector} $q^\star$ as the decision maximizing the average utility over the training set:
% % TODO: Find a better way to input optimization equations
% \begin{align}
%   q^\star &= \argmax_q \hat U\\
%           &= \argmax_q \frac{1}{n}\sum_{s,t} U(p_{st}(q)).
% \end{align}
% The above optimization problem is linear and can therefore be readily solved with any
% modern computer.

% However, we also wish to add a $L_2$ regularization term to the objective in order to
% avoid overfitting. The new learning algorithm is then
% \begin{align}
%   q^\star &= \argmax_q \frac{1}{n} \sum_{s,t} U(p_{st}(q)) - \lambda \|q\|^2.
% \end{align}
% Now even though this new problem is no longer linear, it is still convex and can therefore be
% efficiently solved.

% We will refer to these two algorithms as respectively the non-regularized or linear
% algorithm and the regularized algorithm.

\section{Definitions and Bounds}

\subsection{Definitions and Notation}

Most of the following notation and defintions follow directly from \cite{bousquet2002}.

Let $S_n$ be a set of $n$ vectors of $\real^p\times\real$ of the form:
\begin{equation}
  S_n = \{(x_1,r_1),\ldots,(x_n,r_n)\}.
\end{equation}
Each component of $S_n$ is a tuple $(x,r)$, where $x$ is the information vector and $r$ is
the observed return rate.

Using $S_n$, we wish to create a decision vector $q_{S_n}\in\real^p$ from which we can
make an investment decision when confronted with a random draw $d=(x,r)$.

\paragraph{Loss and Cost.}
We introduce the loss $\ell$ and the cost $c$ of using $q$ with a random draw $d=(x,r)$:
\begin{equation}
\ell(q,d) = c(q(x),r) = c(q^Tx,r)
\end{equation}

The cost must always be a non-negative quantity. Supposing an utility $U$, we model it as
follows:
\begin{equation}
  c(p,r) =
  \begin{cases}
    \lfloor U(r) - U(pr + (1-p)R_f)\rfloor & \text{if } r>R_f\\
    \lfloor U(R_f) - U(pr + (1-p)R_f)\rfloor &\text{if } r\leq R_f
  \end{cases}
\end{equation}
By $\lfloor . \rfloor$ we mean a function returning its argument if non-negative and zero
otherwise. This means that we don't want to discourage taking risk (borrowing or
short-selling), but it's not encouraged either.

\paragraph{Utility.}
There are two ways we can model our utility, and both are concave shaped, to represent a
risk-averse approach. The first utility is the linear utility of the form
\begin{equation}
  U(r) = r + \min(0, \beta r),
\end{equation}
with $0<\beta<1$. The other utility is exponential:
\begin{equation}
  U(r) = -\exp(-\mu r),
\end{equation}
with $\mu > 0$.

\begin{figure}
  \centering
  \begin{subfigure}{.4\textwidth}
  \includegraphics[width=1.1\textwidth]{ExpULossAboveZero.pdf}
\end{subfigure}%
\begin{minipage}{.4\textwidth}
  \includegraphics[width=1.1\textwidth]{ExpULossBelowZero.pdf}
\end{minipage}
\end{figure}

\paragraph{Algorithm.}
We will be concerned with probabilistic confidence bounds on results produced using the
following algorithm, using dataset $S_n$.
\begin{equation}
  q^\star = \argmin_{q\in\real^p}\frac{1}{n} \sum_{i=1}^{n} c(q^Tx_i,r_i) + \lambda\|q\|^2_2.
\end{equation}

\paragraph{Assumptions.}
We will assume that information vectors have been pre-processed and lie in a $X^2_{\max}$
radius ball. We also assume that the return rates observed are comprised within $[-\bar r,
\bar r]$. This last assumption will be relaxed. 

\marginnote{Include reference for definitions and theorems}
\begin{deff}
  A loss function $\ell$ is $\sigma$-admissible if the associated cost function $c$ is
  convex with respect to its first argument and the following condition holds for any
  $p_1,p_2$ and $r$:
  \begin{equation}
    |c(p_1,r) - c(p_2,r)| \leq \sigma |p_1 - p_2|
  \end{equation}
\end{deff}

\begin{rem}
  Our loss function $\ell$ is $\sigma$-admissible with $\sigma=\bar r+R_f$ in the linear
  case and $\sigma=(\bar r+R_f)\exp(\mu\bar r)$ in the exponential case.
\end{rem}

\begin{proof}
  First, we remark that both forms of $U$ yield a convex function of $p$ with $r$ fixed. 

  Now we'll suppose that $c(p_1,r), c(p_2,r) > 0$. Then the expression
  $|c(p_1,r)-c(p_2,r)|$ reduces to
  \begin{equation}
    \label{eq:above1}
    |U(p_1r + (1-p_1)R_f) - U(p_2r + (1-p_2)R_f|.
  \end{equation}
  Now because $r\in[-\bar r,\bar r]$, $U$ is Lipschitz continuous on its domain, and so
  \eqref{eq:above1} is bounded by
  \begin{equation}
    \label{eq:above2}
    \alpha |p_1r + (1-p_1)R_f - (p_2r + (1-p_2)R_f)| = \alpha|p_1-p_2||r-R_f|
  \end{equation}
  where
  \begin{equation}
    \alpha = \sup_{r\in[-\bar r,\bar r]} |U'(r)|.
  \end{equation}

  In the linear case, the derivative is piecewise constant, and is set to 1 on for returns
  below $r_c$, so that $\alpha=1$. In the exponential case, $U'(r) = \exp\mu r$, and
  $\alpha = \exp \mu \bar r$.

  The bound \eqref{eq:above2} must hold for any $r$. The expression $|r-R_f|$ will reach
  its largest value at $r=-\bar r$, since $R_f$ is assumed to be non-negative.

  Finally we consider the case where, without loss of generality, $c(p_2,r)=0$. Then, if
  $c$ had not been defined using $\lfloor .\rfloor$, then we would have
  \begin{align}
    |\floor*{c(p_1,r)} - \floor*{c(p_2,r)}| &\leq |c(p_1,r) - c(p_2,r)|\\
    &\leq \sigma|p_1-p_2|.\qedhere
  \end{align}
\end{proof}

\begin{thm}
  Let $F$ be a reproducing kernel Hilbert space with kernel $\kappa$ that
  $\forall x\in X$, $\kappa(x,x) \leq \kappa^2 <\infty$. If $\ell$ is $\sigma$-adimissible
  with respect to $F$, then the learning algorithm defined by
  \begin{equation}
    \label{eq:above3}
    A_S = \argmin_{g\in F}\frac{1}{n}\sum_{i=1}^n \ell(g,d_i) + \lambda\|g\|^2_k
  \end{equation}
  has uniform stability $\alpha_n$ with respect to $\ell$ with
  \begin{equation}
    \alpha_n \leq \frac{\sigma^2 \kappa^2}{2\lambda n}.
  \end{equation}
\end{thm}

\begin{rem}
  Our proposed algorithm has the form \eqref{eq:above3}, and so has algorithmic stability
  bounded by
  \begin{equation}
    \alpha_n \leq \frac{(\bar r+R_f)^2X^2_{\max}}{2\lambda n}
  \end{equation}
  with linear utility and
  \begin{equation}
    \alpha_n \leq \frac{\exp(2\mu\bar r)X^2_{\max}}{2\lambda n}
  \end{equation}
  in the case of exponential utility.
\end{rem}

\begin{deff}
  The \emph{true risk} with respect to algorithm $A$ and set $S_n$ is defined as
  \begin{equation}
    R_{\text{true}}(A,S_n) = E_d[\ell(A_{S_n},d)],
  \end{equation}
  which is, in plain words, the expected loss incured when applying the algorithm created
  from training set $S_n$ in the wild, ie. out of sample.
\end{deff}

\begin{deff}
  The \emph{empirical risk} with respect to algorithm $A$ and set $S_n$ is defined as
  \begin{equation}
    \hat R(A,S_n) = \frac{1}{n} \sum_{i=1}^n \ell(A_{S_n},d_i),
  \end{equation}
  which is, in plain words, the average cost incured by our model over all the training
  set.
\end{deff}

\begin{rem}
  The maximum loss we can suffer over a single data point happens when $r_i=-\bar r$ and
  $p=1$, ie.
  \begin{equation}
    c(1,-\bar r) = U(R_f) - U(\bar r).
  \end{equation}
  We shall call this quantity $\gamma$.
\end{rem}

\begin{thm}
  Let $A$ be an algorithm with uniform stability $\alpha_n$ with respect to a loss
  function $\ell$ such that $0\leq\ell(A_{S_n},d)\leq M$ for all $d=(x,r)\sim D$ and all sets
  $S_n$ of size $n$. Then for any $n\geq1$ and any $\delta\in(0,1)$, the following bound
  holds with probability at least $1-\delta$ over the random draw of the sample $S_n$:
  \begin{equation}
    |R_{\text{true}}(A,S_n) - \hat R(A,S_n)| \leq 2\alpha_n + (4n\alpha_n + M) \sqrt{\frac{\log(2/\delta)}{2n}}.
  \end{equation}
\end{thm}

\begin{rem}
  Our alogirthm has a generalization bound of
  \begin{equation}
    |R_{\text{true}}|(A,S_n) - \hat R(A,S_n)| \leq 2\alpha_n + (4n\alpha_n + \gamma) \sqrt{\frac{\log(2/\delta)}{2n}}.
  \end{equation}
\end{rem}

\begin{thebibliography}{99}

\bibitem{rudin2015}
  Cynthia Rudin and Gah-Yi Vahn. \textit{The Big Data Newsvendor: Pratical Insights from
    Machine Learning}, Operations Research, 2015.

\bibitem{bousquet2002}
  Olivier Bousquet and André Elisseeff. \textit{Stability and Generalization}, Journal of
  Machine Learning Research, 2002.

\bibitem{lipschitz}
  ``Si la valeur absolue de la dérivée est majorée par $k$, $f$ est $k$-lipschitzienne''.
  \href{https://fr.wikipedia.org/wiki/Application_lipschitzienne}{Application
    lipschitzienne}.

\bibitem{rockafellar}
  Rockafellar, R.~T. \emph{Convex Analysis}, Princeton University Press, 1970.

\bibitem{supergradients}
  \href{http://people.hss.caltech.edu/~kcb/Notes/Supergrad.pdf}{Supergradients}.

\bibitem{refneeded} Reference needed!

\end{thebibliography}
\end{document}
