\section{Model and Assumptions}\label{sec:model}
% \newcommand{\Expect}{E}
\newcommand{\Prob}{P}
\newcommand{\F}{F}
\newcommand{\Fhat}{{\hat{\F}}}
\newcommand{\qhat}{{\hat{q}}}
\newcommand{\Sx}{{\mathcal{S}_X}}
\newcommand{\Sr}{{\mathcal{S}_R}}
\newcommand{\Sn}{s_n}
\newcommand{\urange}{u_{\mbox{\footnotesize range}}}
%\newcommand{\umin}{u_{\mbox{\footnotesize min}}}
\newcommand{\qhatOp}{{\bold{\qhat}}}


%\newcommand{\EUFq}{\mbox{EU($q;\F$)}}
%\newcommand{\EUFhqh}{\mbox{EU($\qhat;\Fhat$)}}
%\newcommand{\EUFqh}{\mbox{EU($\qhat;\F$)}}
%\newcommand{\CEFq}{\mbox{CE($q;\F$)}}
%\newcommand{\CEFhqh}{\mbox{CE($\qhat;\Fhat$)}}
%\newcommand{\CEFqh}{\mbox{CE($\qhat;\F$)}}





We consider a classical financial portfolio selection problem involving a risky asset with
random return rate $R$ and a risk-free asset with return rate of $0\%$ for simplicity. We
also suppose that the investor's risk aversion can be characterized using expected utility
theory using a strictly increasing concave utility function $u$, and that the investor has
access to side information regarding the returns. This information might be the result of
processing the most recent financial or economic news, etc. We let this information be
described as a vector of $p$ normalized random features $[X_1,X_2,\dots,X_p]$. In this
context, if the the distribution $\F$ of the pair $(X,R)$ of side information and return
is known, a linear investment policy that exploits the side information optimally for this
investor can be obtained by solving the following optimization problem:
\begin{equation}
\maximize_{q\in\Re^p}\;\;\;\E_\F[u(R\cdot q^TX)]\;,\label{EUF}
\end{equation}
where it is assumed that short-selling is permitted. 

In practice however, the exact distribution describing the relation between $X$ and $R$ is
not available at the time of designing the investment policy and one might instead need to
exploit a sample set $\Sn:=\{(x_i,r_i)\}_{i=1}^n$ drawn independently and identically from
$\F$. Unfortunately, when the sample size $n$ is relatively small compared to $p$, it is
well known that the problem \eqref{EUF} using the empirical distribution $\Fhat$ obtained
from sample $s_n$ can suffer from severe overfitting and produce investment policies that
perform badly out of sample. This is for instance illustrated in the following example.

\begin{ex}
  Consider for instance a case where $n=p$ and each feature $X_i$ is independantly and
  identically drawn from a Gaussian distribution. Given that it is well known that the
  probability that the random matrix $\Xi := [X_1\;X_2\;\dots\;X_n]$ be singular is null,
  then one can easily establish that problem \eqref{EUF} with $\Fhat$ is
  unbounded. Indeed, one can verify that $r_i \bar{q}^T x_i = 1$ for all $i=1,\dots,n$
  when $\bar{q}$ is set to ${\Xi^{-1}}^T [1/r_1\;1/r_2\;\dots\;1/r_n]^T$. Hence, one can
  achieve an arbitrarily large empirical expected utility by investing according to
  $\alpha\bar{q}$ for $\alpha>0$.
\end{ex}

To prevent issues associated to overfitting, one might instead seek the optimal solution
of the following regularized empirical expected utility maximization problem:
\begin{equation}
\maximize_{q\in\Re^p}\;\;\;\E_\Fhat[u(R\cdot q^TX)]+\lambda\|q\|^2\;.\label{EUFhatReg}
\end{equation}
Note that when it exists, we will refer to the optimal solution of this problem as $\qhat$.



\section{Out-of-sample performance bounds}\label{sec:oos}

The question remains of understanding what guarantees does one have regarding out of
sample performance of the portfolio investment policy obtained from such a regularized
problem. In particular, since utility functions are expressed in units without any
physical meaning for the investor, any guarantees derived using learning theory should be
reinterpreted in terms of a guarantee on the certainty equivalent\footnote{The fact that
  $c$ is the certainty equivalent of a random return $R$ implies that the investors is
  indifferent between being exposed to the risk of $R$ or getting involved in a risk free
  investment that has a return rate of $c$.} (in percent of return) of the risky
investment produced by $\qhat^T X$. In other words, we will be interested in bounding how
different the in-sample certainty equivalent performance of $\qhat$ might be compared to
the out-of-sample certainty equivalent performance. % Likewise, we will also show how we can
% expect the policy $\hat q$ to converge toward an unknown market optimal investment policy
% based on $u$ and $\lambda$.

In order to shed some light on this question, we first make the following assumptions.

\begin{assumption}\label{ass:R}
  The random return $R$ is supported on a bounded interval
  $\Sr\subseteq [-\bar{r},\bar{r}]$ such that $\Prob(|R|\leq \bar{r})=1$.
\end{assumption}

\begin{assumption}\label{ass:X}
  The random vector of side-information $X$ is supported on bounded set $\Sx$ such that
  $\Prob(\|X\|\leq \xi)=1$. 
\end{assumption}

\begin{assumption}\label{ass:u}
  The utility function is normalized such that $u(0)=0$ and $\lim_{r\to0^+}u'(r) =
  1$. Furthermore, it is Lipschitz continuous with a Lipschitz constant of $\gamma$, i.e.,
  for any $r_1\in\Re$ and $r_2\in\Re$, we have that
  $|u(r_1) - u(r_2)| \leq \gamma|r_1-r_2|$.
\end{assumption}

The first assumption is relatively realistic given that one can usually assess from
historical data a large enough interval of returns which could be assumed to contain $R$
with probability one. For instance, when looking at the last 35 years of daily returns for
an index such as S\&P 500, this interval can legitimately be set to $[-25\% , 25\%]$ daily
returns. If some side information are not known to be bounded, the second assumption might
require one to pre-process the vector of side information in order to rely on the results
that will be presented. This could typically be done by projecting this vector on the
surface of a ball of radius $\xi$ when $\|X\|>\xi$, which is as simple as replacing $X$
with $(\xi/\|X\|)\cdot X$. This assumption will be further studied in Section
\ref{sec:bigdata}. Finally, while the last assumption is fairly common for establishing
generalization bounds and can certainly accommodate any piecewise linear utility function
(often used by numerical optimization methods), it is important to mention that it is not
one that is commonly made in modern portfolio theory. If, for instance, an investor
expresses an absolute risk aversion uniformly equal to $\alpha$, this suggests the use of
$u(r):=(1/\alpha)(1-\exp(-\alpha r))$ which is not Lipschitz continuous. Fortunately, the
theory that will be used only exploits the fact that the function is Lipschitz continuous
on the interval $[-\bar{r}^2\xi^2/(2\lambda), \bar{r}^2\xi^2/(2\lambda)]$.

%For example, any piece-wise linear utility would fit the Lipschitz requirements.

We are now in a position to exploit a well-known learning theory result to establish a
bound on the out-of-sample portfolio performance of $\qhat$ based its in-sample
estimation:
\begin{thm}\label{thm:outsampleBound1}
  Given that assumptions \ref{ass:R}, \ref{ass:X} and \ref{ass:u} are satisfied, the
  certainty equivalent of the out-of-sample performance is at most $O(1/\sqrt{n})$ worse
  than the in-sample one. Specifically,
  % \[ \CE(\qhat;\F) \geq \CE(\qhat;\Fhat) -
  %   u_{-1}'(\CE(\qhat;\Fhat))\\frac{(\gamme^2\bar{r}\xi)^2}{2\lambda} \left(\frac{1}{n}
  %     + \frac{4\sqrt{\log(1/\delta)}}{\sqrt{2n}}\right), \]
  \[ 
    \CE(\qhat;\F) \geq \CE(\qhat;\Fhat) -
    \Omega_1/\lim_{\epsilon\to0^-}u'(\CE(\qhat;\Fhat)+\epsilon)\;,
  \]
  where
  \begin{gather*}
    \CE(\qhat;\F):=u^{-1}(\E_\F[u(R\cdot\qhat^T X)])\;,\\
    \CE(\qhat;\Fhat):=u^{-1}(n^{-1}\sum_{i=1}^n u(r_i\,\qhat^T x_i))\;,
  \end{gather*}
  and where
  \[
    \Omega_1 := \frac{\bar{r}^2 \xi^2}{2\lambda} \left(\frac{\gamma^2}{n} +
      \frac{(2\gamma^2+\gamma+1)\sqrt{\log(1/\delta)}}{\sqrt{2n}}\right)
  \] 
  with probability $1-\delta$,
\end{thm}

Our proof of Theorem \ref{thm:outsampleBound1} proceeds as follow. First, borrowing from
the terminology introduced by \cite{bousquet2002stability}, we show that the algorithm
which produces $\qhat$ from the sample set is $\beta$-stable. We then show that for any
$\qhat$ generated from a sample of $\F$, the amount of utility generated from implementing
the $\qhat$ decision necessarily lies on an interval of bounded size. Given that these two
conditions are satisfied, we can then rely on Bousquet-Ellisseef's out-sample error bound
theorem (typically used for inference problems) in order to establish out-of-sample
guarantees in terms of expected utility. By exploiting the concavity of $u(\cdot)$, we are
finally able to describe the implications in terms of certainty equivalent that are
expressed in our theorem.



%%% Local Variables:
%%% mode: latex
%%% TeX-master: "big_data_portfolio_optimization"
%%% End:

\subsection{Market efficiency and true optimal}
\newcommand{\EU}{{\bf EU}}
We now turn our attention to the suboptimality of the decision $\hat q$, specifically,
how well it performs in comparison to the optimal decision $q^\star =
\argmax_q\Psi(q)$. There are indeed some conditions on $u$ and on $M$. Consider a
risk-neutral utility $u(r)=r$. Then,
\[
  \E[u(R\,q^TX)] = q_1\E[RX_i]+\cdots+q_p\E[RX_p].
\]
Therefore, if we set $q_i = \infty$ when $\E(RX_i)>0$ and $q_i=-\infty$ when $\E(RX_i)<0$,
then $\CE(q)=\infty$ and no bounds on the suboptimality can be set. Likewise, if
$\pp\{RX_i>0\}=1$, we can again set $q_i=\infty$ and the same kind of divergence will
happen. This relates to the notion of arbitrage: we say a feature $X_i$ induces arbitrage
if $\pp\{RX_i>0\}=1$ or $\pp\{RX_i<0\}=1$.

\begin{thm}
  \label{thm_truopt}
  If $r=o(u(r))$, ie. $u(r)$ is sublinear and if no feature in the market induces
  arbitrage, then 
  \[
    \CE(q^\star) \geq \CE(\hat q) - \omega\cdot\grad(u^{-1})(\Psi(\hat q)),
  \]
  where
  \[
    \omega = \gamma\bar r\xi\|q^\star - \hat q\|,
  \]
  and $\omega$ is finite. 
\end{thm}

Following is the proof of the first part of Theorem \ref{thm_truopt}. Note first that 
\begin{align*}
  |\Psi(q^\star)-\Psi(\hat q)| &\leq |\E[u(R\,{q^\star}^T)] - \E[u(R\,\hat q^TX)]|\\
                             &\leq \E[|u(R{q^\star}^TX) - u(R\,\hat q^TX)|]\\
                             &\leq \gamma \E[R({q^\star}-\hat q)^TX]\\
                             &\leq \gamma\bar r\xi\|q^\star - \hat q\|=\omega,
\end{align*}
so that $\Psi(q^\star) \geq \Psi(\hat q) - \omega$. We can invert this result back in the
result space using the same method as in Theorem \ref{thm:outsampleBound1} and the result
follows.

Next, we show that $\omega$ is bounded by proving that $\|q^\star\|$ is finite. Since the
other values are also bounded the second part of Theorem \ref{thm_truopt} follows. 

Let $Z(q)=R\,q^TX\subsetsim\real$. Instead of optimizing $q$ on $\real^p$, we can optimize
its scale by taking $s^\star = \argmax_{s>0}g(s)$ where $g(s)$ is the solution of
\begin{align*}
  \maximizeEquationSt{\E[u(sZ(q))]}[\|q\|= 1].
\end{align*}
Let $q\in\real^p$ such that $\|q\|=1$. Since no feature induce arbitrage, it follows that,
for any $q$, there exists a $\delta_q<0$ such that $\pp\{Z(q)<\delta_q\}=\varrho>0$. Now,
let $B(q)$ be a discrete random variable with two states such that
$\pp\{B(q)=\delta_q\}=1-\pp\{B=\bar r\xi\}=\varrho$. Since $|Z(q)|<\bar r\xi$, we have
that $\pp\{B(q)\geq r\} \geq \pp\{Z(q)\geq r\}$, so that $\E B(q) \geq \E Z(q)$, from which
it follows that $\E[u(sB(q))]\geq \E[u(sZ(q))]$. But, by the sublinearity asumption on
$u$, 
\[
  \lim_{s\to\infty}\E[u(sB(q))] = \lim_{s\to\infty}\big(\varrho
  u(s\delta_q)+(1-\varrho)u(s\bar r\xi)\big) = -\infty.
\]
And therefore, $\lim_{s\to\infty}\E[u(sZ(q))]=-\infty$ for all $q$, which shows that
$s^\star$, and therefore $\|q^\star\|$, is bounded.


%%% Local Variables:
%%% mode: latex
%%% TeX-master: "big_data_portfolio_optimization"
%%% End:

\section{Big Data Phenomenon}\label{sec:bigdata}

In this section, we question how realistic assumption \ref{ass:X} is in a big data
context. In particular, we expose two sets of natural conditions for the generation of the
side information vector $X$ that leads to motivating the use of a support set which
diameter grow proportionally to the square root of $p$.

\begin{ex}
  Consider a case where every terms of $X$ are independant from each other, while each
  $X_i$ has a mean $\E[X_i]=0$, a variance $\Var[X_i]=1$, and are supported on their
  respective intervals $\Prob(X_i\in [-\nu, \nu])=1$ for all $i$. By Hoeffding's
  inequality, one can establish that
  \[
    \Prob\left(\Bigl|\|X\|^2 - \sum_{i=1}^p \E[X_i^2]\Bigr| \leq
      \sqrt{2p\log(\delta/2)\nu^2}\right) \geq 1-\delta
  \]
  so that
  $|\|X\|^2 \in [p- \sqrt{2p\log(\delta/2)\nu^2}, p+ \sqrt{2p\log(\delta/2)\nu^2}]$ with
  probability $1-\delta$. Hence, any ball of fixed radius $\xi$ will contain $X$ with a
  probability that asymptotically converges to zero as $p$ increases, more specifically
  $\Prob(\|X\|^2\leq \xi^2)\leq 2\exp(-2p(1-\xi^2/\sqrt{p})^2/\nu^2)$. On the other hand,
  this inequality somehow also prescribes that the diameter of the support $\Sx$ should
  increase proportionally to $\sqrt{p}$ in order to still contain $X$ with high
  probability as $p$ increases.
\end{ex}

\begin{ex}
  Consider a similar case as above but where the independance assumption is dropped. In
  this context, although we might not have as much of a strong argument to discredit the
  use of a constant diameter for $\Sx$, there is still a good motivation for employing a
  radius that grows proportionally to $\sqrt{p}$. Namely, if each $X_i$ has a mean
  $\E[X_i]=0$ and a variance $\Var[X_i]=1$ then the random variable $Z:=\|X\|^2$ is
  necessarily positive with an expected value of $p$. Based on Markov inequality, this
  implies that with probability $1-\delta$, we have that $\|X\|\leq \sqrt{p/\delta}$.
\end{ex}

Since we believe these two examples provide strong arguments for replacing assumption
\ref{ass:X} with the assumption that it is within a ball of radius $\xi\sqrt{p}$, we
reformulate our previous two results as follows.

\begin{coro}\label{coro:outsampleBoundBigData}
  Given that assumptions \ref{ass:R} and \ref{ass:u} are satisfied, and that
  $\Prob(\|X\|\leq \xi\sqrt{p})=1$, the certainty equivalent of the out-of-sample
  performance is at most $O(p/\sqrt{n})$ worse than the in-sample one. Specifically, with
  probability $1-\delta$,
  \[
    \CE(\qhat;\F) \geq \CE(\qhat;\Fhat) - \Omega_3/
    \lim_{\epsilon\to0^-}u'(\CE(\qhat;\Fhat)+\epsilon)\;,
  \]
  where
  \[
    \Omega_3 := \frac{\gamma\bar{r}^2\xi^2}{\lambda} \left(\frac{\gamma p}{2n} +
      \frac{(1+\gamma)p\sqrt{\log(1/\delta)}}{\sqrt{2n}}\right).
  \]
  Likewise, the suboptimality of the decision $\hat q$ will reach a constant bound due to
  regularization a rate of at most $O(p/\sqrt{n})$:
  \[
    \CE(\hat q;F) \geq \CE(q^\star;F) - \Omega_4/\lim_{\epsilon\to0^{-}}u'(CE(\hat q;F)+\epsilon)\;,
  \]
  where
  \[
    \Omega_4 = \lambda\|q^\star\|^2 +\frac{8\gamma^2p\xi^2(32+\log(1/\delta))}{n\lambda}
    +\frac{2\gamma\bar rp\xi^2}{\lambda}\sqrt{\frac{32+\log(1/\delta)}{n}},
  \]
  with probability $1-\delta$.
\end{coro}

%This should be in conclusion of this section
Note that assumption \ref{ass:X} was inspired by \cite{rudin2015big} who also studied
asymptotic properties of a regularized decision problem in its Big data regime, \ie, when
$n$ and $p$ go to infinity simultaneously. Our analysis indicate that the convergence in
accuracy that is reported there for regime $p\propto n$ might be misleading for many
problems, \eg, when the features can be considered independent from each other.  In
particular, our new results states that asymptotic convergence in accuracy when the sample
set is large only guaranteed to occur when $p=o(\sqrt{n})$. 

However, it is important to understand that Corollary 1 serves as a worst case scenario
and that we don't necessarily expect to observe downgrading performances as soon as
$n=o(p^2)$. Still, no matter what, there is a cost to pay in pouring more and more
features into such a portfolio selection problem, and this cost is directly exhibited
through $\xi^2$ and the loosening of the guarantees bound. One might therefore wish to be
prudent when facing such high-risk regimes.

% \begin{ex}
%   As a final example, consider Figure 1 where the out-of-sample deviation $\CE(\hat q;F) -
%   \CE(\hat q;\hat F)$ is shown. 
% \end{ex}


%%% Local Variables:
%%% mode: latex
%%% TeX-master: "big_data_portfolio_optimization"
%%% End:



%%% Local Variables:
%%% mode: latex
%%% TeX-master: "big_data_portfolio_optimization"
%%% End:
