\usepackage{nips_2016}

% \usepackage{times}
\usepackage{amsmath}
% %\usepackage{pdfcomment}
\usepackage{amsthm,amsfonts}
\usepackage{graphicx}
% \usepackage{geometry}
% \usepackage{subcaption}
\usepackage{bm}
% \usepackage{hyperref}
% \usepackage[retainorgcmds]{IEEEtrantools}
\usepackage{xparse}
% \usepackage{mathtools}
% \usepackage{color}
% \usepackage{marginnote}
% \usepackage{booktabs}
\usepackage[utf8]{inputenc}
\usepackage{mathrsfs}
% \usepackage{enumitem}

%\usepackage{biblatex}
%\addbibresource{bibliography.bib}

% \setlength{\parindent}{0em}
% \setlength{\parskip}{0.8em}

\DeclareMathOperator*{\argmax}{arg\,max}
\DeclareMathOperator*{\argmin}{arg\,min}
\DeclareMathOperator*{\maximize}{maximize}
\DeclareMathOperator*{\minimize}{minimize}
\DeclareMathOperator{\st}{s.t.}
\DeclareMathOperator{\epi}{epi}
\DeclareMathOperator{\diag}{diag}
\DeclareMathOperator{\dom}{dom}
\DeclareMathOperator{\tr}{tr}
\DeclareMathOperator{\Var}{Var}
\DeclareMathOperator{\CE}{CE}
\DeclareMathOperator{\Cov}{Cov}
% \DeclareMathOperator{\E}{E}

% \DeclarePairedDelimiter\floor{\lfloor}{\rfloor}


\newcommand{\ts}{\textsuperscript}
\newcommand{\figref}[1]{Fig.~\ref{#1}}

\newcommand{\iso}{\simeq}
\newcommand{\dd}{\partial}
\newcommand{\real}{\mathbb{R}}
\renewcommand{\Re}{\real}
\newcommand{\normal}{\mathscr{N}}
\newcommand{\trueRisk}{R_{\mathrm{true}}}
\newcommand{\uInv}{u^{-1}}
\newcommand{\qHat}{{\hat q}}
\newcommand{\qStar}{{q^\star}}
% \newcommand{\xMax}{X_{\max}}
\newcommand{\xMax}{\xi}
\newcommand{\grad}{\nabla}
% \newcommand{\dd}{\partial}
\newcommand{\E}{\bm E} 
\newcommand{\pp}{\bm P} 
% \newcommand{\CE}{\bm{CE}}
\newcommand\subsetsim{\mathrel{%
  \ooalign{\raise0.2ex\hbox{$\subset$}\cr\hidewidth\raise-0.8ex\hbox{\scalebox{0.9}{$\sim$}}\hidewidth\cr}}}

\def\rcurs{{\mbox{$\resizebox{.09in}{.08in}{\includegraphics[trim= 1em 0 14em
        0,clip]{ScriptR}}$}}}
\newcommand{\rf}{\rcurs_f}

\renewcommand{\rf}{r_0}

\newcommand{\starsection}{\vspace{1em}\begin{center}$\star\quad\star\quad\star$\end{center}\vspace{1em}}

% http://tex.stackexchange.com/questions/111551/recursive-multiple-subscript-and-superscript-with-xparse
\NewDocumentCommand{\minimizeEquationSt}{m >{\SplitList{,}}O{}}
{\text{minimize} & \quad #1\\ \text{subject to} \ProcessList{#2}{\stCommand}}
\newcommand{\minimizeEquation}[1]{\text{minimize} \quad {#1}}
\NewDocumentCommand{\stCommand}{m}
{& \quad #1}

\NewDocumentCommand{\maximizeEquationSt}{m >{\SplitList{,}}O{}}
{\text{maximize} & \quad #1\\ \text{subject to} \ProcessList{#2}{\stCommand}}
\newcommand{\maximizeEquation}[1]{\text{maximize} \quad {#1}}
% \NewDocumentCommand{\stCommand}{m}
% {& \quad #1}

\newcommand{\comment}[1]{\textbf{[#1]}}
\newcommand{\Erick}[1]{\textbf{[Erick says: #1]}}

\theoremstyle{plain}
\newtheorem{prop}{Proposition}
\newtheorem{thm}{Theorem}
\newtheorem{coro}{Corollary}
\newtheorem*{thm*}{Theorem}
\newtheorem{claim}{Claim}
\newtheorem{lemma}{Lemma}
\newtheorem{assumption}{Assumption}

\theoremstyle{definition}
\newtheorem*{definition}{Definition}
\newtheorem*{rem}{Remark}
\newtheorem*{ex}{Example}

%% http://tex.stackexchange.com/a/43971/4233
\makeatletter
\def\th@plain{%
  \thm@notefont{}% same as heading font
  \itshape % body font
}
\def\th@definition{%
  \thm@notefont{}% same as heading font
  \normalfont % body font
}
\makeatother


%\geometry{letterpaper}
%\IEEEeqnarraydefcolsep{0}{\leftmargini}
%%% Local Variables:
%%% mode: latex
%%% TeX-master: "big_data_portfolio_optimization"
%%% End:
