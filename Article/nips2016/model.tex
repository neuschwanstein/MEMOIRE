\section{Model and Main Results}

\subsection{Assumptions and definitions}


\comment{Nécesaire?} In the following, $\bm A$ (capital boldface) are assumed to represent
a real subset of any dimension, $A$ (capital case) represents random variables (or
distributions) and $a$ (lower case) represents deterministic variables or
realizations. $\real$ represents the real set, $\subsetsim$ the support of a random
variable, and $\|\cdot\|$ is the euclidean norm.

Our model considers the market $M$ as being a $p+1$-variate random distribution, with on
its first margin a random (finite) return
$R\subsetsim\bm R=[r_{\min},r_{\max}]\subset\real$ \comment{Il peut être plus simple
  d'avoir $|R|\leq\bar r$, notamment dans l'expression de $\Omega$} and on the other
margin a random vector of features $(X_1,\dots,X_p)$, which would typically represent
financial or economic news, etc. We will assume that all features are pairwise
independant.

We also suppose that the investor is endowed with a monotonically increasing concave
utility function $\bar u:\bm R\to\bm U$, such that $\bar u$ can be rescaled to $u$ with
$\bar u(r) = ku(r) + l$, with the additionnal requirements that $u(0)=0$,
$\lim_{r\to0^+}\grad u(r) = 1$ and that $u$ is $\gamma$-Lipschitz, ie. such that for any
$r_1,r_2\in\bm R$, $|u(r_1) - u(r_2)| \leq \gamma|r_1-r_2|$. For example, any piece-wise
linear utility would fit the Lipschitz requirements.

Our method studies optimal linear investment decisions $q\in\bm Q\subseteq\real^p$ over
the random features so as to maximize the certainty equivalent $\CE(q)$ of the portfolio,
where:
\[
  \CE(q) = u^{-1}(\Psi(q)),
\]
with
\[
  \Psi(q) = \E_M[u(R\,q^TX)]
\]
the out-sample utility of $q$. We would typically add a riskless return rate to the
equation, however we set it to 0 for the sake of simplicity. Additionnally, given a sample
$\{(x_i,r_i)\}_{i=1}^n$ drawn from $M^n$, we will also study the sample certainty
equivalent $\hat\CE$:
\[
  \hat\CE(q) = u^{-1}(\hat\Psi(q)),
\]
with
\[
  \hat\Psi(q) = n^{-1}\sum_{i=1}^n u(r_i\,q^Tx_i)
\]
the in-sample utility of $q$. 

\subsection{Out-sample complexity}

Supposing we have a size $n$ sample drawn \textit{i.i.d.} from the market, then a natural
choice for $q$ would be the optimal solution of the regularized in-sample utility:
\[
  \hat q = \argmax_q\{\hat\Psi(q)-\lambda\|q\|^2\}  = \argmax_q \left\{\frac{1}{n}\sum_{i=1}^n u(r_i\,q^Tx_i) - \lambda\|q\|^2\right\},
\]
where $\lambda\|q\|^2$ is here to avoid overfitting on the training sample. We will
sometimes refer to $\hat q$ as the algorithm mapping from a market sample to the decision
vector, rather than the decision vector itself.

Before going on, we will assume that the random features vector is bounded,
ie. $\|X\|\leq\xi$, although we will relax this hypothesis in the next subsection.

We are now in a position to present a bound on the out-sample error:
\begin{thm}
  With probability $1-\delta$, the error between the in- and out-sample certainty
  equivalent is bounded by the following relation:
  \[
    \CE(\hat q) \geq \hat\CE(\hat q) - \Omega\cdot\nabla u^{-1}(\hat\CE(\hat q)),
  \]
  where
  \[
    \Omega = \frac{(\bar r\xi)^2}{2\lambda} \left(\frac{\gamma^2}{n} + \frac{\gamma(1+3\gamma)}{\sqrt{2n}}\sqrt{\log(1/\delta)}\right).
  \]
In particular, this implies that the error bound shrinks at a $O(1/\sqrt{n})$ rate. 
\end{thm}

Our proof of Theorem 1 proceeds as follow. First, borrowing from the terminology
introduced by \cite{bousquet2002stability}, we show that the algorithm leading to $\hat q$
is $\beta$-stable. We then show that for any $\hat q$ generated from a sample of $M$, the
utility derived from applying this decision will be absolutely bounded, regardless of the
outcome from $M$. These last two conditions can therefore lead to a direct application of
Bousquet-Ellisseef out-sample error bound theorem on the $\bm U$ space. We finally show how
this result can be inverted back to the $\bm R$ space.

\begin{lemma}
  \label{sigma-adm}
  Let $q_1,q_2\in\bm Q$, $x\sim X$ and $r\sim R$. The algorithm generating $\hat q$ has
  $\sigma$-admissiblity of $\gamma\bar r$, ie.
  \[
    |u(r\,q_1^T x) - u(r\,q_2^T x)| \leq \gamma\bar r|q_1^Tx - q_2^Tx|.
  \]
\end{lemma}
\begin{proof}
  This lemma follows trivially from the Lipschitz property of $u$. See Definition 19 from
  \cite{bousquet2002stability} for more details.
\end{proof}

\begin{lemma}
  \label{beta-bound}
  The algorithm generating $\hat q$ has $\beta$-stability, ie. with $s_n \sim M^n$ and
  $s_n'$ differing from $s_n$ by a single resampling from $M$, then, 
  \[
    |u(R\,\hat q(s_n)^T X) - u(R\,\hat q(s_n')^T X)| \leq \beta,
  \]
  where
  \[
    \beta \leq \frac{(\gamma\bar r\xMax)^2}{2\lambda n}.
  \]
\end{lemma}
\begin{proof}
  Using Lemma \ref{sigma-adm}, this follows directly from Theorem 22 in
  \cite{bousquet2002stability}.
\end{proof}

\begin{lemma}
  \label{q-bound}
  The norm of the decision $\hat q$ is bounded:
  \[
    \|\hat q\| \leq \frac{\gamma\bar r\xMax}{2\lambda}.
  \]
\end{lemma}
\begin{proof}
  Let $\{(x,r)\}_n \sim M^n$ be an \textit{i.i.d.} sample of the market. The empirical decision
  algorithm is equivalent to
  \begin{align*}
    \maximizeEquationSt{n^{-1}\sum_{i=1}^n u(r_i\,q^Tx_i) - \lambda s^2}[s\geq0\\,\|q\|=1],
  \end{align*}
  where the optimization variables are now the direction $q$ and the scale $s$. Therefore,
  for any direction $q$, we can define a concave function $g(s)$ which becomes the
  objective:
  \begin{align*}
    \maximizeEquationSt{g(s) = n^{-1}\sum_{i=1}^n u(r_i\,sq^Tx_i) - \lambda s^2}[s\geq 0].
  \end{align*}
  
  Because $g:\real_{+}\to\real$ is concave, we can consider two cases: either the maximum
  is realized at the boundary, ie. $s^\star=0$, or there exists an optimal value
  $s^\star > 0$ such that $\nabla g(s^\star)=0$. To derive a bound on $s^\star$, we can seek a
  value $\bar s$ such that for any $q$, $\nabla{}g(\bar s)<0$ and therefore $s^\star < \bar s$.

  To do so, we first note that 
  \[
    \grad g(s) = n^{-1}\sum_{i=1}^nr_i\,q^Tx_i\,u'(r_i\,sq^Tx_i) - 2\lambda s,
  \]
  bu because $\|q\|=1$, we have $q^Tx_i\leq\|x_i\|\leq\xMax$. We also have
  $r_i\leq \bar r$ and $u'\leq\gamma$, so that
  \[
    \grad g(s) \leq \gamma\bar r\xi - 2\lambda s.
  \]
  Therefore, with
  \[
    \bar s = \frac{\gamma\bar r\xMax}{2\lambda},
  \]
  we have $\grad g(\bar s)\leq 0$. 
\end{proof}

\begin{lemma}
  \label{u-bound}
  For any $(x,r)\sim M$ and any $\hat q$,
  \[
    -\frac{(\gamma\xi \bar r)^2}{2\lambda} \leq u(r\,\hat q^Tx) \leq \frac{\gamma(\xi\bar r)^2}{2\lambda}.
  \]
\end{lemma}

\begin{proof}
  The maximum utility will be realized when $r = \bar r$, so that
  \[
    u(r\,\hat q^T x) \leq r {\hat q}^T x \leq \frac{\gamma(\xi\bar r)^2}{2\lambda},
  \]
  since the identity function bounds $u$ above. Likewise for negative returns, although
  this time $\gamma$ applies. 
\end{proof}

The following theorem was first proven in \cite{bousquet2002stability}, although its
statement is adapted from \cite{mohri2012foundations} and is presented in accordance to
our particular setting.
\begin{thm*}[Bousquet-Ellisseef Outsample Error Theorem]
  Let $s_n=\{(x_i,r_i)\}_{i=1}^n$ by a size $n$ sample drawn \textit{i.i.d.} from $M$. If
  $\hat q$ has $\beta$-stability and
  $\hat u_{\min}\leq u(R\,\hat q^TX) \leq \hat u_{\max}$, then, with probability
  $1-\delta$, 
  \[
    \Psi(\hat q) \geq \hat\Psi(\hat q) - \Omega_u,
  \]
  where
  \[
    \Omega_u = \beta + (2n\beta + (\hat u_{\max}-\hat u_{\min}))\sqrt{\frac{\log(1/\delta)}{2n}}.
  \]
\end{thm*}

Using directly Lemma \ref{beta-bound} and \ref{u-bound}, we therefore find the following outsample
error bound on the utility
\[
  \Omega_u = \frac{(\bar r\xi)^2}{2\lambda} \left(\frac{\gamma^2}{n} + \frac{\gamma(1+3\gamma)}{\sqrt{2n}}\sqrt{\log(1/\delta)}\right).
\]

We now show how to transform this last result on a bound on the CE of the decision. Note
that for any convex function $f$, $f(a+b) \geq f(a) + b\cdot\grad f(a)$. Therefore, from
the out-sample error bounding theorem, we have
\[
  u^{-1}(\Psi(\hat q)) \geq u^{-1}(\hat\Psi(\hat q) - \Omega_u) \geq u^{-1}(\hat\Psi(\hat
  q)) - \Omega_u\cdot\grad(u^{-1})(\hat\Psi(\hat q)),
\]
since $u^{-1}$ is also a monotonic function. This proves Theorem 1. 


\subsection{Big Data Phenomenon}
We now take a closer look on the effect the dimension of the feature space can have on the
bound $\Omega$ stated in Theorem 1, and in particular on the bound $\xi^2$. If we let
$Z^2 = \sum_{i=1}^nX_i^2$ be the random squared norm of $X$, we can show that $Z^2$ is of
the order $O(p)$ with high probabiltiy . This implies that the algorithm $\hat q$ has in
fact a sample complexity $O(p/\sqrt{n})$.

We present three cases, each with additional generalization properties. In what follows,
we will assume with no loss of generality (because it is an affine transformation) that
$\E X_i=0$ and $\Var X_i = 1$, which already implies that $\E X_i^2=1$, and therefore
$\E Z^2=p$.

\comment{Ajouter de l'intuition pour le lecteur.}

Let us first consider the specific case where $X\sim\normal(0,I)$, ie. $X$ is a
$p$-mutlinormal random vector. It then follows that $Z^2\sim\chi^2(p)$. But we know from
\cite{laurent2000adaptive} that a chi-square distribution has the following property for
all $t$:
\[
\pp\{Z^2-p \geq 2\sqrt{pt} + 2t\} \leq e^{-t},
\] 
which is equivalent, with probability $1-\delta$ to:
\[
  Z^2 < p + 2\sqrt{p\log(1/\delta)} + 2\log(1/\delta).
\]

As a somewhat more natural example, without making any asumption on the distribution of
the features, we can consider the case where each of them is bounded, either by truncation
in the pre-processing step or because their support is known to be finite. If
$X_i^2 \leq \nu_i$, and we let $\nu^2_0 = \sum_{i=1}^p \nu_i^2$, then, by Hoeffding's
theorem,
\[
  \pp\{Z^2 - p\geq t\} \leq \exp\left(-\frac{t^2}{\nu_0^2}\right),
\]
which, again, can be reexpressed as the following inequality with probability $1-\delta$:
\[
  Z^2 < p + \nu_0\sqrt{\log(1/\delta)}.
\]
Finally, Markov's inequality provides the most general theorem for the situation, since it
simply states that with probability $1-\delta$, 
\[
  Z^2 < \delta^{-1}p.
\]
\begin{thm}
  With probability $1-(\delta_1+\delta_2)$, the error between the in- and out-sample certainty
  equivalent is bounded by the following relation:
  \[
    \CE(\hat q) \geq \hat\CE(\hat q) - \Omega\cdot\nabla u^{-1}(\hat\CE(\hat q)),
  \]
  where
  \[
    \Omega = \frac{\bar r^2p}{2\lambda\delta_1} \left(\frac{\gamma^2}{n} + \frac{\gamma(1+3\gamma)}{\sqrt{2n}}\sqrt{\log(1/\delta_2)}\right).
  \]
  If the features are bounded, then 
  \[
    \Omega = (p+\nu_0\sqrt{\log(1/\delta_1)})\frac{\bar r^2}{2\lambda} \left(\frac{\gamma^2}{n} + \frac{\gamma(1+3\gamma)}{\sqrt{2n}}\sqrt{\log(1/\delta_2)}\right).
  \]
In particular, this implies that the error bound shrinks at a $O(p/\sqrt{n})$ rate. 
\end{thm}

\subsection{Market efficiency and true optimal}

The last theoretical topic we want to discuss is how our model relates to the theory of
market efficiency. 
\begin{deff}
  Let $q^\star = \argmax_q\Psi(q)$. Then $M$ is said to be efficient with respect to $u$
  if $\|q^\star\|$ is bounded. 
\end{deff}



























\section{Old.}

% \begin{assumption}
%   Every feature $X_i$ and $X_j$ is pairwise independant. 
% \end{assumption}

% \begin{assumption}
%   Each feature has been standardized, ie. $\E X_i = 0$ and $\Var X_i = 1$. In particular,
%   this implies that $\E X_i^2 = 1$.
% \end{assumption}

\begin{assumption}
  The random return has a finite support, ie.
  $R\subsetsim \bm R \subseteq [r_{\min{}},r_{\max{}}]$. Additionally, $|R|\leq \bar r$.
\end{assumption}

\begin{assumption}
  The portfolio manager is endowed with an utility function $\bar u:\bm R\to \bm U$ with
  these properties:
  \begin{itemize}
  \item $\bar u$ can be reexpressed as $\bar u(r) = ku(r) + l$, $k>0$, with $u(0) = 0$ and
    $\lim_{r\to0^+}u'(r) = 1$.
  \item $u(r) = o(r)$, ie. the investor is risk-averse;
  \item $|u(r_1) - u(r_2)| \leq \gamma|r_1-r_2|$, ie. $u$ is $\gamma$-Lipschitz;
  \item $u$ is monotonically increasing;
  \item With $u(r) = u_-(r)\bm1_{\{r<0\}}+u_+(r)\bm 1_{\{r\geq0\}}$, then $u_+(r) =
    o(u_-(r))$. In other words, $u_-$ decreases faster than $u_+$ increases. 
  \end{itemize}
\end{assumption}

\begin{deff}
  Let $\ell:\bm M\times \bm Q\to\bm U$ be a loss function defined by
  \[
    \ell(m,q) = \ell(x,r,q) = -u(rq^Tx)
  \]
  where $\rf$ is the risk free return rate. We also define the cost function
  $c:\bm I\times\bm R\to\bm U$ as
  \[
    c(p,r) = -u(rp),
  \]
  so that $\ell(x,r,q) = c(q^Tx,r)$. 
\end{deff}

\begin{deff}
  The in-sample risk $\hat R: \bm M^n\times \bm Q \to \bm U$ associated with decision $q$
  and market sample $\mu_n$ is given by
  \[
    \hat R_{\mu_n}(q) = n^{-1} \sum_{i=1}^n \ell(m_i,q).
  \]
\end{deff}

\begin{deff}
  The empirical decision algorithm $\hat A_n:\bm M^n \to \bm Q$ associated with
  market sample $\mu_n$ is the optimal value of the problem
  \[
    \text{minimize}\quad\hat R_{\mu_n}(q) + \lambda\|q\|^2.
  \]
\end{deff}

From now on, as a notation shortcut, let $\hat q_n := \hat A_n(\mu_n)$ the in-sample
decision associated with random market sample $\mu_n$ and $\hat R:=\hat R_{\mu_n}$ the
in-sample risk function.

\begin{deff}
  The in-sample certainty equivalent $\hat\CE:\bm M^n\times\bm Q\to\bm R$ associated with
  decision $q$ and market sample $\mu_n$ is given by
  \[
    \hat\CE(q) = ku^{-1}(-\hat R(q)) + l.
  \]
\end{deff}

% \comment{Remove unnecessary definitions.}

\begin{deff}
  The true risk $R:\bm Q\to\bm U$ associated with decision $q$ is given by
  \[
    R(q) = \E\ell(M,q).
  \]
\end{deff}

% \begin{deff}
%   The true regularized risk $R_\lambda:\bm Q\to\bm U$ associated with decision $q$ is
%   given by
%   \[
%     R_\lambda(q) = \E\ell(M,q) + \lambda\|q\|^2.
%   \]
% \end{deff}

% \begin{deff}
%   The optimal decision $q^\star$ is the optimal value of the problem
%   \[
%     \text{minimize}\quad R(q).
%   \]
% \end{deff}

% \begin{deff}
%   The optimal regularized decision $q^\star_\lambda$ is the optimal value of the
%   regularized problem
%   \[
%         \text{minimize}\quad R_\lambda(q).
%   \]
% \end{deff}

\begin{deff}
  The true certainty equivalent $\CE$ associated with decision $q$ is given by
  \[
    \CE(q) = ku^{-1}(-R(q)) + l.
  \]
\end{deff}

\subsection{Performance Bounds}

We are concerned about how the in sample performance can deviate from the expected out sample
performance, that is we want to identify $f_1$ such that
\[
  \CE(\hat q) \geq \hat\CE(\hat q) - f_1(n,p,\lambda)
\]
with high probability. We are also interested in the suboptimality of the problem, namely
the function $f_2$ such that
\[
  \CE(q^\star) \geq \CE(\hat q) - f_2(n,p,\lambda),
\]
also with high probability. 

The following theorem is adapted from \cite{bousquet2002stability}, and is the starting point of our
analysis. 

\begin{thm}
  \label{thm1}
  The in-sample and out-sample performance of the algorithm given by $\hat q$ is bounded
  by the following expression with probability $1-\delta$:
  \begin{align*}
    R(\hat q) &\leq \hat R(\hat q) + \frac{(\gamma\bar r\xMax)^2}{2\lambda n} +
    \left(\frac{(\gamma\bar r\xMax)^2}{\lambda} + \frac{\gamma(\gamma+1)\xMax^2\,r_{\max}\bar
    r}{2\lambda}\right)\sqrt{\frac{\log 1/\delta}{2n}}\\
              &:= \hat R(\hat q) + \Omega.
  \end{align*}
\end{thm}

\begin{proof}
  \comment{See claim ????? for further details.}
\end{proof}

\begin{thm}
  \label{thm2}
  The following inequality holds with probability $1-\delta$:
  \[
    \CE(\hat q) \geq \hat\CE(\hat q) + ku^{-1}(\Omega).
  \]
\end{thm}

\begin{proof}
  The following steps follow directly from the monotonicity, convexity, and
  superadditivity of $u^{-1}$ :
  \begin{align*}
    & R(\hat q) \leq \hat R(\hat q) + \Omega\\
    \iff & -R(\hat q) \geq -\hat R(\hat q) - \Omega\\
    \iff & u^{-1}(-R(\hat q)) \geq u^{-1}(-\hat R(\hat q) - \Omega)\\
    \iff & u^{-1}(-R(\hat q)) \geq u^{-1}(-\hat R(\hat q)) + u^{-1}(-\Omega)\\
    \iff & ku^{-1}(-R(\hat q))+l \geq ku^{-1}(-\hat R(\hat q))+l + ku^{-1}(-\Omega)\\
    \iff & \CE(\hat q) \geq \hat\CE(\hat q) + ku^{-1}(-\Omega).\qedhere
  \end{align*}
\end{proof}

\begin{thm}\
  \label{thm4}
  Likewise, the following inequality holds:
  \[
    \E_{\mu_n}[\CE(\hat q)] \geq 
  \]
\end{thm}

Since $\Omega>0$, it follows that $u^{-1}(-\Omega) > -\Omega$, therefore we have the
following relation:
\[
  \CE(\hat q) \geq \hat\CE(\hat q) - O\left(\frac{\xMax^2}{\sqrt{n}\lambda}\right).
\]


\subsection{Big Data Situation}

The literature revolving around Theorem \ref{thm1} and its applications ([Shai-Shalev,
Rudin]) generally leaves the $\xMax$ as an afterthought, but in real big-data contexts, if
$n$ is insufficiently large compared to $p$, than out-of-sample convergence might not be
certain. Actually, with $n=o(p^2)$, divergence is almost certain. 

Let's assume that $\E X_i = 0$ for all features, and let $Z^2 = \sum_{i=1}^p X_i^2$. In a
general setting, if $\E X_i^2 \leq M$, then $\E Z^2 \leq Mp$, and Markov's inequality
applies:
\[
  \pp\{Z^2 \geq t\} \leq \frac{\E Z^2}{t} \leq \frac{Mp}{t}.
\]
Equivalently, with probability $1-\delta$, 
\[
  Z^2 \leq \delta^{-1}Mp = O(p).
\]

If we further assume pairwise independance of features and that each feature is supported
by a closed interval, wether because the support of the feature is known to be bounded or
because it's been saturated in pre-processing, then each feature can be rescaled by $a_i$
so that so that $a_iX_i = \tilde X_i \subsetsim[-1,1]$, or $\tilde X_i^2
\subsetsim[0,1]$. Then, using Hoeffding's theorem,
\[
  \pp\{\tilde Z^2 \geq \tilde Mp + t\} \leq \exp\left(-\frac{2t^2}{p}\right),
\]
equivalently with probability $1-\delta$, 
\[
  \tilde Z^2 \leq \tilde Mp + \sqrt{\frac{p\log(1/\delta)}{2}} = O(p)
\]
Of course, it is easy to see that such a transformation is reversible. 

The point is that in general cases, we expect to have $\xMax^2 =O(p)$, so that the
convergence of our algorithm will be given by 
\[
  \CE(\hat q) \geq \hat\CE(\hat q) - O\left(\frac{p}{\sqrt{n}\lambda}\right).
\]

%%% Local Variables:
%%% mode: latex
%%% TeX-master: "big_data_portfolio_optimization"
%%% End:
