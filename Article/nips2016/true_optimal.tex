\subsection{Suboptimality performance bounds}

We now turn our attention to the suboptimality of the problem, \ie, we would like to
understand the behaviour of the performance of the empirical investment policy $\hat q$
compared to the optimal policy $q^\star := \argmax_q \E_F[u(R\cdot q^T X)]$. It is
important to realize that in many instances, the optimal performance might be unbounded,
and therefore any bound on the suboptimality would be meaningless.

\begin{ex}
  Consider for instance a risk neutral investor, \ie, such that $u(r)=r$ and suppose
  $\E X_i=0$. The expected utility simply becomes
  \[
    \E_F[u(R\cdot q^T X)] = \sum_{i=1}^{n}q_i\Cov(R,X_i).
  \]
  If we simply let $\bar q_i = \Cov(R,X_i)$, it follows immediately that the expected
  utility of $\bar q$ is infinite.
\end{ex}

\begin{ex}
  Consider another example where feature $X_i$ induces arbitrage over $F$, by which we
  mean that either $\pp\{RX_i<0\} = 1$ or $\pp\{RX_i > 0\} = 1$. In such a case, if we let
  $\bar q_j = \delta_{ij}$, then the expected utility of $\alpha\bar q$ can once again
  take arbitrarily lage values as $\alpha\to\infty$. 
\end{ex}

Given those two examples, we now introduce two new assumptions in order to obtain finite
optimal values.

\begin{assumption}\label{ass:usublinear}
  The utility function is sublinear, \ie, $u(r) = o(r)$. 
\end{assumption}

\begin{assumption}\label{ass:arbitrage}
  The feature space $X$ induces no arbitrage over $F$, that is, for all $X_i$, $\pp\{RX_i
  < 0\} < 1$ and $\pp\{RX_i > 0\} < 1$. 
\end{assumption}

In a financial context, assumption \ref{ass:usublinear} is certainly realistic since the
most utility functions describing risk preferences of an investor are usually sublinear
(for example, with $O(\sqrt{r})$, $O(\log r)$ or $O(-e^{-r})$ shapes being the most common
ones). As for assumption \ref{ass:arbitrage}, this notion or arbitrage relates directly to
the notion or market efficiency, and in particular to the semi-strong version of it (see
\cite{malkiel1970efficient} and \cite{fama1991efficient}).

\begin{thm}
  \label{thm_truopt}
  Given that assumptions \ref{ass:R}, \ref{ass:X}, \ref{ass:u} are satisfied, the
  suboptimaliy of the policy $\hat q$ can be expressed with confidence $1-3\delta$ by
  \[
    \CE(\hat q;F) \geq \CE(q^\star;F) - \omega_1/\lim_{\epsilon\to0^{-}}u'(CE(\hat q;F)+\epsilon)\;,
  \]
  where
  \[
    \omega_1 = -\psi -\frac{8\gamma^2\xi^2(32+\log(1/\delta))}{n\lambda} -\frac{2\gamma\bar
    r\xi^2}{\lambda}\sqrt{\frac{32+\log(1/\delta)}{n}},
  \]
  and $\psi:=E_F[u(R\cdot {q^{\star}}^T X)] - E_F[u(R\cdot {q^\star_\lambda}^TX)]$.
  Furthermore, if assumptions \ref{ass:usublinear} and \ref{ass:arbitrage} are satisfied, then
  $\CE(q^\star;F)$ is finite. 
\end{thm}

Roughly, what Theorem \ref{thm_truopt} states is that the suboptimality of the problem
will converge at a rate $O(1/\sqrt{n})$ toward the optimal regularized certainty
equivalent of the problem, while leaving a gap $\psi$ that depends both on $F$ and on
$\lambda$.


%%% Local Variables:
%%% mode: latex
%%% TeX-master: "big_data_portfolio_optimization"
%%% End:
