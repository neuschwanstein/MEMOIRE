\subsection{Market efficiency and true optimal}
\newcommand{\EU}{{\bf EU}}
We now turn our attention to the suboptimality of the decision $\hat q$, specifically,
how well it performs in comparison to the optimal decision $q^\star =
\argmax_q\Psi(q)$. There are indeed some conditions on $u$ and on $M$. Consider a
risk-neutral utility $u(r)=r$. Then,
\[
  \E[u(R\,q^TX)] = q_1\E[RX_i]+\cdots+q_p\E[RX_p].
\]
Therefore, if we set $q_i = \infty$ when $\E(RX_i)>0$ and $q_i=-\infty$ when $\E(RX_i)<0$,
then $\CE(q)=\infty$ and no bounds on the suboptimality can be set. Likewise, if
$\pp\{RX_i>0\}=1$, we can again set $q_i=\infty$ and the same kind of divergence will
happen. This relates to the notion of arbitrage: we say a feature $X_i$ induces arbitrage
if $\pp\{RX_i>0\}=1$ or $\pp\{RX_i<0\}=1$.

\begin{thm}
  \label{thm_truopt}
  If $r=o(u(r))$, ie. $u(r)$ is sublinear and if no feature in the market induces
  arbitrage, then 
  \[
    \CE(q^\star) \geq \CE(\hat q) - \omega\cdot\grad(u^{-1})(\Psi(\hat q)),
  \]
  where
  \[
    \omega = \gamma\bar r\xi\|q^\star - \hat q\|,
  \]
  and $\omega$ is finite. 
\end{thm}

Following is the proof of the first part of Theorem \ref{thm_truopt}. Note first that 
\begin{align*}
  |\Psi(q^\star)-\Psi(\hat q)| &\leq |\E[u(R\,{q^\star}^T)] - \E[u(R\,\hat q^TX)]|\\
                             &\leq \E[|u(R{q^\star}^TX) - u(R\,\hat q^TX)|]\\
                             &\leq \gamma \E[R({q^\star}-\hat q)^TX]\\
                             &\leq \gamma\bar r\xi\|q^\star - \hat q\|=\omega,
\end{align*}
so that $\Psi(q^\star) \geq \Psi(\hat q) - \omega$. We can invert this result back in the
result space using the same method as in Theorem \ref{thm:outsampleBound1} and the result
follows.

Next, we show that $\omega$ is bounded by proving that $\|q^\star\|$ is finite. Since the
other values are also bounded the second part of Theorem \ref{thm_truopt} follows. 

Let $Z(q)=R\,q^TX\subsetsim\real$. Instead of optimizing $q$ on $\real^p$, we can optimize
its scale by taking $s^\star = \argmax_{s>0}g(s)$ where $g(s)$ is the solution of
\begin{align*}
  \maximizeEquationSt{\E[u(sZ(q))]}[\|q\|= 1].
\end{align*}
Let $q\in\real^p$ such that $\|q\|=1$. Since no feature induce arbitrage, it follows that,
for any $q$, there exists a $\delta_q<0$ such that $\pp\{Z(q)<\delta_q\}=\varrho>0$. Now,
let $B(q)$ be a discrete random variable with two states such that
$\pp\{B(q)=\delta_q\}=1-\pp\{B=\bar r\xi\}=\varrho$. Since $|Z(q)|<\bar r\xi$, we have
that $\pp\{B(q)\geq r\} \geq \pp\{Z(q)\geq r\}$, so that $\E B(q) \geq \E Z(q)$, from which
it follows that $\E[u(sB(q))]\geq \E[u(sZ(q))]$. But, by the sublinearity asumption on
$u$, 
\[
  \lim_{s\to\infty}\E[u(sB(q))] = \lim_{s\to\infty}\big(\varrho
  u(s\delta_q)+(1-\varrho)u(s\bar r\xi)\big) = -\infty.
\]
And therefore, $\lim_{s\to\infty}\E[u(sZ(q))]=-\infty$ for all $q$, which shows that
$s^\star$, and therefore $\|q^\star\|$, is bounded.


%%% Local Variables:
%%% mode: latex
%%% TeX-master: "big_data_portfolio_optimization"
%%% End:
