\subsection{Proof of Theorem \ref{thm_truopt}}

First, in order to tidy up the proof, let
\begin{gather*}
  \EU(q) := \E_F(u(R\cdot q^T X))\,;\\
  \EU_\lambda(q) := \E_F(u(R\cdot q^T X)) - \lambda\|q\|^2,
\end{gather*}
with $q^\star := \argmin_q \EU(q)$ and $q^\star_\lambda := \argmin_q \EU_\lambda(q)$.

\begin{thm}[Theorem 1 in \cite{sridharan2009fast}]\label{thm:shai}
  Given that assumptions \ref{ass:R}, \ref{ass:X}, and \ref{ass:u} are satisfied, and
  since $EU_\lambda$ is $2\lambda$-strongly convex, then one has with confidence of
  $1-\delta$ that
  \[
    -\lambda\|\hat q - q^\star_\lambda\| \geq \EU_\lambda(\hat q) - \EU_\lambda(q^\star_\lambda) \geq -\omega,
  \]
  where
  \[
    \omega = \frac{4\gamma^2\xi^2(32+\log(1/\delta))}{\lambda n}.
  \]
\end{thm}

Notice that Theorem \ref{thm:shai} implies with confidence $1-\delta$ that
\[
  \EU(\hat q) - \EU(q^\star_\lambda) \geq \lambda\bigl(\|\hat q\|^2 -
  \|q^\star_\lambda\|^2\bigr) -\omega
  \geq -\lambda\bigl(\|\hat q -
  q^\star_\lambda\|^2 +2\|\hat q\|\|\hat q - q^\star_\lambda\|\bigr) - \omega.
\]
As shown in Section \ref{sec:thm1_proof}, $\|\hat q\| \leq \bar r\xi/(2\lambda)$. Reusing
Theorem \ref{thm:shai}, we again obtain that
$\|\hat q - q^\star_\lambda\|^2 \leq \omega/\lambda$, and therefore
$\|\hat q - q^\star_\lambda\| \leq \sqrt{\omega/\lambda}$, with probability $1-\delta$
each, so that we end up with
\[
  \EU(\hat q) - \EU(q^\star_\lambda) \geq -2\omega - \bar r\xi\sqrt{\frac{\omega}{\lambda}}.
\]
with probability $1-3\delta$.

If now we let $\psi := \EU(q^\star) -\EU(q^\star_\lambda)$, we can then bound the
suboptimality of decision $\hat q$ with probability $1-3\delta$ in the following fashion:
\begin{align*}
  \EU(\hat q) &= \EU(q^\star) + \EU(\hat q) - \EU(q^\star_\lambda) + \EU(q^\star_\lambda)
                - \EU(q^\star)\\
              &\geq \EU(q^\star) - \psi - \bar r\xi\sqrt{\frac{\omega}{\lambda}} - 2\omega.
\end{align*}
The can finally invert this last result in terms of CE using the same trick as in Section
\ref{sec:thm1_proof}.

Next, we show that $\CE(q^\star;F)$ is bounded by proving that $\|q^\star\|$ is
finite. Since the other values are also bounded the second part of Theorem
\ref{thm_truopt} follows.

Let $Z(q)=R\,q^TX\subsetsim\real$. Instead of optimizing $q$ on $\real^p$, we can optimize
its scale by taking $s^\star = \argmax_{s>0}g(s)$ where $g(s)$ is the solution of
\begin{align*}
  \maximizeEquationSt{\E[u(sZ(q))]}[\|q\|= 1].
\end{align*}
Let $q\in\real^p$ such that $\|q\|=1$. Since no feature induce arbitrage, it follows that,
for any $q$, there exists a $\delta_q<0$ such that $\pp\{Z(q)<\delta_q\}=\varrho>0$. Now,
let $B(q)$ be a discrete random variable with two states such that
$\pp\{B(q)=\delta_q\}=1-\pp\{B=\bar r\xi\}=\varrho$. Since $|Z(q)|<\bar r\xi$, we have
that $\pp\{B(q)\geq r\} \geq \pp\{Z(q)\geq r\}$, so that $\E B(q) \geq \E Z(q)$, from which
it follows that $\E[u(sB(q))]\geq \E[u(sZ(q))]$. But, by the sublinearity asumption on
$u$, 
\[
  \lim_{s\to\infty}\E[u(sB(q))] = \lim_{s\to\infty}\big(\varrho
  u(s\delta_q)+(1-\varrho)u(s\bar r\xi)\big) = -\infty.
\]
And therefore, $\lim_{s\to\infty}\E[u(sZ(q))]=-\infty$ for all $q$, which shows that
$s^\star$, and therefore $\|q^\star\|$, is bounded.






%%% Local Variables:
%%% mode: latex
%%% TeX-master: "big_data_portfolio_optimization"
%%% End:
