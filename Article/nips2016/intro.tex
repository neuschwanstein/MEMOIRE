\section{Introduction}
\label{sec:intro}

Ever since it was formally theorized by Markowitz \cite{markowitz1952portfolio}, one-step
theoretical portfolio management has mostly kept the same approach: maximize the returns
while minimizing the variance using a trade-off parameter. However, such an approach
suffers from a fatal flaw, as it needs to make asumptions on the underlying distribution
of the returns. While Markowitz considered gaussian returns, others have investigated more
sophisticated distributions, using eg. jump diffusion, gamma returns,
etc. \comment{citations needed.}  On the other side, \cite{cover1991universal} exhibits a
clever stock-picking algorithm with asymptotic performance guarantees
\comment{retravailler, plus de détails}. However, unlike classical portfolio theory, this
method assumes a risk-neutral behaviour, ie. an investor wishing only to maximize profits,
with no regards to the possible loss. \comment{Et ici aussi.}

This work is an attempt at bridging these two concepts. Using a size $n$ sample of the
(unknown) market distribution consisting of market features and market returns, a
regularization parameter $\lambda$ and by specifying an arbitrary concave utility
function, we can derive an in-sample optimal linear investment policy by optimizing the
certainty equivalent on the sample. We first show that the out-sample performance of the
policy is bounded by a $O(1/\sqrt{n})$ error term. Second, We also investigate how this
this method scales when the number of market features $p$ is of the order of $n$, ie. in a
\textit{big-data} regime, and show that the performance scales linearly in the number $p$
of available features. As far as we are aware, this situation has not been studied by the
learning theory, and consequently we hope to enrich the field. \comment{Remanier.}
Finally, we determine the conditions under which the true optimal solution in regard to
the market distribution can be attained. \comment{We conclude by presenting numerical results from
different degenerated distributions.}

The \textit{market} considered by this document could  be any  asset
traded on the market.\comment{Incorporer quelque part.}

At a higher level, this document should be mostly understood as providing guidance to
portfolio managers who would wish to incorporate general statistical and machine learning
strategies in order to uncover market returns indicators. In fact, as more and more
features are poured into a model (for example by considering polynomial kernels
\comment{reference needed}), there is real possibility that the out-sample performance
becomes degraded, and we wish to show how it can be prevented. 



Most of this work derives from statistical learning theory, and in particular from
stability theory, as exposed by Bousquet and Elisseef in their seminal paper
\cite{bousquet2002stability}.  The author showed, using powerful concentration
inequalities, how the empirical risk minimization of a Lipschitz loss function with
additional convexity driven by a $\ell_2$ regularization on the decision would converge in
the size of the sample toward the out-sample performance. In particular, their results
were a departure from classical learning theory as the tools they were using stems
strictly from algorithmic and convexity analysis. 

We also improve on results from \cite{rudin2015big} \comment{anonynimize?} who study the
application of learning theory to a feature based newsvendor problem. However, while they
explicitly consider the big-data regime, we believe our model is more general in the sense
that we directly show the effects of $p$ on the performance of the algorithm. 

\comment{Padder davantage, plus de details sur 1. theorie moderne de portefeuille,
  2. portefeuille universel, 3. Theorie de la stabilité, 4. Donner plus de références.}

%%% Local Variables:
%%% mode: latex
%%% TeX-master: "master"
%%% End:
