\documentclass{article}
\newcommand{\ts}{\textsuperscript}
\newcommand{\figref}[1]{Fig.~\ref{#1}}
\usepackage{amsmath}
\usepackage{amsthm}
\usepackage{graphicx}
\usepackage{geometry}
\usepackage{subcaption}
\usepackage{bm}
\usepackage{hyperref}
\usepackage[retainorgcmds]{IEEEtrantools}
\usepackage{mathtools}
\usepackage{color}
\usepackage{marginnote}
\usepackage[utf8]{inputenc}


\DeclareMathOperator*{\argmax}{arg\,max}
\DeclareMathOperator*{\argmin}{arg\,min}
\DeclareMathOperator{\st}{s.t.}
\DeclareMathOperator{\epi}{epi}
\DeclareMathOperator{\diag}{diag}
\DeclareMathOperator{\dom}{dom}
\DeclareMathOperator{\tr}{tr}
\DeclareMathOperator*{\minimize}{minimize}
\DeclarePairedDelimiter\floor{\lfloor}{\rfloor}

\newcommand{\iso}{\simeq}
\newcommand{\dd}{\partial}
\newcommand{\real}{\bm R}
\newcommand{\trueRisk}{R_{\mathrm{true}}}

\newcommand{\starsection}{\vspace{1em}\begin{center}$\star\quad\star\quad\star$\end{center}\vspace{1em}}


\newcommand{\hilight}[1]{\colorbox{yellow}{#1}}
\let\oldmarginnote\marginnote
\renewcommand{\marginnote}[1]{\oldmarginnote{\footnotesize\emph{#1}}[0cm]}

\theoremstyle{plain}
\newtheorem{prop}{Proposition}
\newtheorem{thm}{Theorem}

\theoremstyle{definition}
\newtheorem*{deff}{Definition}
\newtheorem*{rem}{Remark}


\geometry{letterpaper}
\IEEEeqnarraydefcolsep{0}{\leftmargini}


\title{SVM Formulation of the Portfolio Optimization Problem}
\author{Thierry Bazier-Matte}

\begin{document}
\maketitle

\section{Linearly Separable and Step Utility Function}


Under an SVM formulation of our problem, we look for a decision $q$ such that $q^Tx$ is
the same sign as its output. Under a classical SVM formulation, where the problem is one
of classification, corresponding to a loss step function, the output is $y=\pm 1$.

Let us suppose for now that returns $r_i$ are linearly separable. Then we can scale the
decision $q$ to obtain $\min_S|q^Tx| = 1$, with $S$ the sample. Then, the distance of any
feature point $x_0\in\real^p$ will be given by
\[
  \frac{|q^Tx_0|}{\|q\|_2}.
\]
Therefore the margin $\rho$ of the SVM will be given by
\[
  \rho = \frac{1}{\|q\|_2}.
\]
Geometrically, we wih to have a margin as large as possible or, equivalently, to minimize
$\|q\|_2$, or more simply $\frac{1}{2}\|q\|^2_2$. 

Under the classical formulation of the SVM, where $y_i=\pm1$, we wish to have $\sign
q^Tx_i = \sign y_i$. This can also be expressed as
\[
  y_i\,q^Tx_i \geq 1,
\]
since $q^Tx_i\geq1$. Under the porfolio formulation, the output $y_i$ is actually the
portfolio return $r_i\in\real$. Let us pretend for the moment that we are endowed with a
step utility function. In other words, we only care for the sign of our investment. Then
we would again like to obtain $\sign r_i = \sign q^Tx_i$. Let us further suppose that
there is no $j$ such that $r_j=0$. Let $\bar r = \min_S r_i$. Then using the
transformation $\tilde r_i = r_i/\bar r$, we obtain $r_i \geq 1$. We therefore obtain the
same formulation as with the SVM, that is,
\begin{align*}
  \minimizeEquationSt{\|q\|^2}{\tilde r_i\,q^Tx_i\geq 1}.
\end{align*}


\section{Utility dependant case}

Under the classical SVM formulation, when the set is not separable, we can add slack
variables $\xi_i\geq 0$ to the margin so that
\[
  y_i\,q^Tx_i \geq 1 - \xi_i.
\]
The SVM formulation can therefore be expressed as
\begin{align*}
  \minimizeEquationSt{\|q\|^2+\lambda\|\xi\|^p}[y_i\,q^Tx_i\geq 1-\xi_i\\,\xi_i\geq 0],
\end{align*}
where $\mu$ is the trade off parameter between the slack and the margin width.

Under the classical framework, $\xi_i$ is only some slack, and it loss function will
typically be quadratic or linear (hinge loss). However, under the portfolio optimization
framework, this slack $\xi_i$ actually means much more since we can encourage or
discourage going in certain directions proportionally to the utility of the investor.

Let us first suppose that we have no preference for positive returns, or simply that we
have a flat utility for $r\geq 0$. Such an asumption corresponds to $\xi_i\geq 0$: as long
as the returns $r_i\,q^Tx_i$ are deep enough out of the margin, we are satisfied. But we
can now quantify how much dissatisfaction we retire from going toward the margin. We can
impose a penalty $-u(\xi_i)$ on the slack on the whole domain. We can further drop the
absence of preference on positive returns, by simply stating that we impose $-u(\xi_i)$ on
the slack. This way, the utility perception of the investor is transposed in the SVM
algorithm using the slacks:
\begin{align*}
  \minimizeEquationSt{\|q\|^2-\lambda\sum_{i=1}^nu(\xi_i)}{\tilde r_i\,q^Tx_i\geq 1 - \xi_i}
\end{align*}
The optimization problem remains convex, since $u$ is concave.

Under such a formulation, we see how close have gotten to the original problem: compare
this last expression with the following:
\begin{align*}
  \maximizeEquation{\frac{1}{n}\sum_{i=1}^nu(r_i\,q^Tx_i) - \lambda\|q\|^2}.
\end{align*}
It is pratically the same: we now see how close to an actual SVM we are. In wit, this
means that are theoretical properties of SVMs also apply to our problem. \todo{Show formally.}

Now what interpretation can be given to those two traded off terms? The first expression
$\|q\|^2$ means how wide the margin will be, that is how probable it is that we actually
have overall positive returns. It can also be meant to represent the complexity of the
solution: a tighter margin will accomodate more outlier points, thus reducing their
numbers and thus favouring a more complex solution, whereas a wider margin will be
simpler, albeit with more outliers. The other expression $-\sum u(\xi_i)$ can be
understood as the average of the utility for each observed return $r_i\,q^Tx_i$.



\end{document}

%%% Local Variables:
%%% mode: latex
%%% TeX-master: t
%%% End:
