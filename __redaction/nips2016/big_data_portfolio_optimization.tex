\documentclass{article}

\newcommand{\ts}{\textsuperscript}
\newcommand{\figref}[1]{Fig.~\ref{#1}}
\usepackage{amsmath}
\usepackage{amsthm}
\usepackage{graphicx}
\usepackage{geometry}
\usepackage{subcaption}
\usepackage{bm}
\usepackage{hyperref}
\usepackage[retainorgcmds]{IEEEtrantools}
\usepackage{mathtools}
\usepackage{color}
\usepackage{marginnote}
\usepackage[utf8]{inputenc}


\DeclareMathOperator*{\argmax}{arg\,max}
\DeclareMathOperator*{\argmin}{arg\,min}
\DeclareMathOperator{\st}{s.t.}
\DeclareMathOperator{\epi}{epi}
\DeclareMathOperator{\diag}{diag}
\DeclareMathOperator{\dom}{dom}
\DeclareMathOperator{\tr}{tr}
\DeclareMathOperator*{\minimize}{minimize}
\DeclarePairedDelimiter\floor{\lfloor}{\rfloor}

\newcommand{\iso}{\simeq}
\newcommand{\dd}{\partial}
\newcommand{\real}{\bm R}
\newcommand{\trueRisk}{R_{\mathrm{true}}}

\newcommand{\starsection}{\vspace{1em}\begin{center}$\star\quad\star\quad\star$\end{center}\vspace{1em}}


\newcommand{\hilight}[1]{\colorbox{yellow}{#1}}
\let\oldmarginnote\marginnote
\renewcommand{\marginnote}[1]{\oldmarginnote{\footnotesize\emph{#1}}[0cm]}

\theoremstyle{plain}
\newtheorem{prop}{Proposition}
\newtheorem{thm}{Theorem}

\theoremstyle{definition}
\newtheorem*{deff}{Definition}
\newtheorem*{rem}{Remark}


\geometry{letterpaper}
\IEEEeqnarraydefcolsep{0}{\leftmargini}


\title{Generalization Bounds for Regularized Portfolio Selection with Market Side Information}
\author{
  Erick Delage\\
  Department of Decision Sciences\\
  HEC Montreal\\
  Montreal...\\
  \texttt{erick.delage@hec.ca}\\
  \And
  Thierry Bazier-Matte\\
  HEC Montreal\\
  \texttt{thierry.bazier-matte@hec.ca}\\
}

\begin{document}
\maketitle

\begin{abstract}
%  In recent years,  much pressure has been applied in order to allow portfolio management theory to address was required to evolve in order to  
  Drawing on statistical learning theory, we derive out-of-sample and suboptimal
  guarantees about the investment strategy obtained from a regularized portfolio
  optimization model which attempts to exploit side information about the financial market
  in order to reach an optimal risk-return tradeoff. This side information might include
  for instance recent stock returns, volatility indexes, financial news indicators,
  etc. In particular, we demonstrate that an investment policy that linearly combines this
  side information in a way that is optimal from the perspective of a random sample set is
  guaranteed to perform also relatively well (\ie, within an additive perturbation factor
  of $O(1/\sqrt{n})$) with respect to the unknown distribution that generated this sample
  set. Finally, we evaluate the sensitivity of these results in a high dimensional regime
  where the size of the side information vector is of an order that is comparable to the
  sample size.

  % \Erick{Voici l'ancien abstract: Drawing on statistical learning theory, we propose a
  %   robust portfolio optimization mechanism agnostic to market distribution based on side
  %   information and on market returns. In particular, we exhibit a linear investment
  %   policy based on the risk preferences of an investor and on a sample of the market and
  %   show that out-of-sample and suboptimality guarantees on the quality of the certainty
  %   equivalent of the proposed investment can be provided. In addition, we also consider
  %   the high dimensional case where the number of these side informations is of the order
  %   of the sample size. }
\end{abstract}

\section{Introduction}

\subsection{Définition du problème}


Ce mémoire vise à établir clairement et rigoureusement de quelle façon l'apprentissage
machine et la théorie de la statistique moderne permettent à un investisseur doté d'une
grande quantité d'information relative au marché (information que nous appelerons sans
discrimination \textit{facteurs de marché}, \textit{caractéristiques de marché},
... \cit).

Mathématiquement, on peut représenter cette situation en modélisant le \textit{marché}
comme étant une distribution statistique à haute dimension, ayant sur une marge une
variable aléatoire de rendement $R \in \R \subseteq \Re$, et ayant $p$ autres marges décrivant
l'information complémentaire $(X_1,\dots,X_m) \in \X \subseteq \Re^m$. La distribution complète de
marché $M \in \R \times \X$ est alors décrite par
\begin{equation}
  M = (R,X_1,\dots,X_m).
\end{equation}
Notons que la dépendance entre ces coordonnées marginales est inconnue.

On a d'autre part l'aspect d'aversion au risque qui est modélisé par une fonction
d'utilité $u:\R\to\U$, où $\R \subseteq \Re$ est le domaine (fermé ou non) des rendements considérés
et $\U \subseteq \Re$ celui des \textit{utilités}.

Bien qu'en pratique il soit plus facile de travailler sur des fonctions possédant des
valeurs dans $\U$, en pratique cet espace est adimensionel\cit, de sorte que nos résultats
seront présentés dans l'espace des rendements $\R$.

Donnés ces éléments de base, le but de ce mémoire sera alors de déterminer une fonction de
décision d'investissement $q: \X \to \P$ maximisant l'utilité espérée de l'investissement. 

Mathématiquement on a donc le problème fondamental suivant:
\begin{equation}
  \maximizeEquation{\E u(R \cdot q(X)).}
\end{equation}
\todo{Expliquer.}

\subsection{Hypothèses restrictives}



\subsection{Raffinements et hypothèses supplémentaires}


Formulé de façon statistique, lorsqu'on dispose de $n$ observations de $M$, le problème
peut se réexprimer comme
\begin{equation}
  \maximizeEquation{n^{-1}\sumi u(r_i\,q(x_i)).}
\end{equation}



%%% Local Variables:
%%% mode: latex
%%% TeX-master: "main_intro"
%%% End:

\section{Model and Assumptions}\label{sec:model}
% \newcommand{\Expect}{E}
\newcommand{\Prob}{P}
\newcommand{\F}{F}
\newcommand{\Fhat}{{\hat{\F}}}
\newcommand{\qhat}{{\hat{q}}}
\newcommand{\Sx}{{\mathcal{S}_X}}
\newcommand{\Sr}{{\mathcal{S}_R}}
\newcommand{\Sn}{s_n}
\newcommand{\urange}{u_{\mbox{\footnotesize range}}}
%\newcommand{\umin}{u_{\mbox{\footnotesize min}}}
\newcommand{\qhatOp}{{\bold{\qhat}}}


%\newcommand{\EUFq}{\mbox{EU($q;\F$)}}
%\newcommand{\EUFhqh}{\mbox{EU($\qhat;\Fhat$)}}
%\newcommand{\EUFqh}{\mbox{EU($\qhat;\F$)}}
%\newcommand{\CEFq}{\mbox{CE($q;\F$)}}
%\newcommand{\CEFhqh}{\mbox{CE($\qhat;\Fhat$)}}
%\newcommand{\CEFqh}{\mbox{CE($\qhat;\F$)}}





We consider a classical financial portfolio selection problem involving a risky asset with
random return rate $R$ and a risk-free asset with return rate of $0\%$ for simplicity. We
also suppose that the investor's risk aversion can be characterized using expected utility
theory using a strictly increasing concave utility function $u$, and that the investor has
access to side information regarding the returns. This information might be the result of
processing the most recent financial or economic news, etc. We let this information be
described as a vector of $p$ normalized random features $[X_1,X_2,\dots,X_p]$. In this
context, if the the distribution $\F$ of the pair $(X,R)$ of side information and return
is known, a linear investment policy that exploits the side information optimally for this
investor can be obtained by solving the following optimization problem:
\begin{equation}
\maximize_{q\in\Re^p}\;\;\;\E_\F[u(R\cdot q^TX)]\;,\label{EUF}
\end{equation}
where it is assumed that short-selling is permitted. 

In practice however, the exact distribution describing the relation between $X$ and $R$ is
not available at the time of designing the investment policy and one might instead need to
exploit a sample set $\Sn:=\{(x_i,r_i)\}_{i=1}^n$ drawn independently and identically from
$\F$. Unfortunately, when the sample size $n$ is relatively small compared to $p$, it is
well known that the problem \eqref{EUF} using the empirical distribution $\Fhat$ obtained
from sample $s_n$ can suffer from severe overfitting and produce investment policies that
perform badly out of sample. This is for instance illustrated in the following example.

\begin{ex}
  Consider for instance a case where $n=p$ and each feature $X_i$ is independantly and
  identically drawn from a Gaussian distribution. Given that it is well known that the
  probability that the random matrix $\Xi := [X_1\;X_2\;\dots\;X_n]$ be singular is null,
  then one can easily establish that problem \eqref{EUF} with $\Fhat$ is
  unbounded. Indeed, one can verify that $r_i \bar{q}^T x_i = 1$ for all $i=1,\dots,n$
  when $\bar{q}$ is set to ${\Xi^{-1}}^T [1/r_1\;1/r_2\;\dots\;1/r_n]^T$. Hence, one can
  achieve an arbitrarily large empirical expected utility by investing according to
  $\alpha\bar{q}$ for $\alpha>0$.
\end{ex}

To prevent issues associated to overfitting, one might instead seek the optimal solution
of the following regularized empirical expected utility maximization problem:
\begin{equation}
\maximize_{q\in\Re^p}\;\;\;\E_\Fhat[u(R\cdot q^TX)]+\lambda\|q\|^2\;.\label{EUFhatReg}
\end{equation}
Note that when it exists, we will refer to the optimal solution of this problem as $\qhat$.



\section{Out-of-sample performance bounds}\label{sec:oos}

The question remains of understanding what guarantees does one have regarding out of
sample performance of the portfolio investment policy obtained from such a regularized
problem. In particular, since utility functions are expressed in units without any
physical meaning for the investor, any guarantees derived using learning theory should be
reinterpreted in terms of a guarantee on the certainty equivalent\footnote{The fact that
  $c$ is the certainty equivalent of a random return $R$ implies that the investors is
  indifferent between being exposed to the risk of $R$ or getting involved in a risk free
  investment that has a return rate of $c$.} (in percent of return) of the risky
investment produced by $\qhat^T X$. In other words, we will be interested in bounding how
different the in-sample certainty equivalent performance of $\qhat$ might be compared to
the out-of-sample certainty equivalent performance. % Likewise, we will also show how we can
% expect the policy $\hat q$ to converge toward an unknown market optimal investment policy
% based on $u$ and $\lambda$.

In order to shed some light on this question, we first make the following assumptions.

\begin{assumption}\label{ass:R}
  The random return $R$ is supported on a bounded interval
  $\Sr\subseteq [-\bar{r},\bar{r}]$ such that $\Prob(|R|\leq \bar{r})=1$.
\end{assumption}

\begin{assumption}\label{ass:X}
  The random vector of side-information $X$ is supported on bounded set $\Sx$ such that
  $\Prob(\|X\|\leq \xi)=1$. 
\end{assumption}

\begin{assumption}\label{ass:u}
  The utility function is normalized such that $u(0)=0$ and $\lim_{r\to0^+}u'(r) =
  1$. Furthermore, it is Lipschitz continuous with a Lipschitz constant of $\gamma$, i.e.,
  for any $r_1\in\Re$ and $r_2\in\Re$, we have that
  $|u(r_1) - u(r_2)| \leq \gamma|r_1-r_2|$.
\end{assumption}

The first assumption is relatively realistic given that one can usually assess from
historical data a large enough interval of returns which could be assumed to contain $R$
with probability one. For instance, when looking at the last 35 years of daily returns for
an index such as S\&P 500, this interval can legitimately be set to $[-25\% , 25\%]$ daily
returns. If some side information are not known to be bounded, the second assumption might
require one to pre-process the vector of side information in order to rely on the results
that will be presented. This could typically be done by projecting this vector on the
surface of a ball of radius $\xi$ when $\|X\|>\xi$, which is as simple as replacing $X$
with $(\xi/\|X\|)\cdot X$. This assumption will be further studied in Section
\ref{sec:bigdata}. Finally, while the last assumption is fairly common for establishing
generalization bounds and can certainly accommodate any piecewise linear utility function
(often used by numerical optimization methods), it is important to mention that it is not
one that is commonly made in modern portfolio theory. If, for instance, an investor
expresses an absolute risk aversion uniformly equal to $\alpha$, this suggests the use of
$u(r):=(1/\alpha)(1-\exp(-\alpha r))$ which is not Lipschitz continuous. Fortunately, the
theory that will be used only exploits the fact that the function is Lipschitz continuous
on the interval $[-\bar{r}^2\xi^2/(2\lambda), \bar{r}^2\xi^2/(2\lambda)]$.

%For example, any piece-wise linear utility would fit the Lipschitz requirements.

We are now in a position to exploit a well-known learning theory result to establish a
bound on the out-of-sample portfolio performance of $\qhat$ based its in-sample
estimation:
\begin{thm}\label{thm:outsampleBound1}
  Given that assumptions \ref{ass:R}, \ref{ass:X} and \ref{ass:u} are satisfied, the
  certainty equivalent of the out-of-sample performance is at most $O(1/\sqrt{n})$ worse
  than the in-sample one. Specifically,
  % \[ \CE(\qhat;\F) \geq \CE(\qhat;\Fhat) -
  %   u_{-1}'(\CE(\qhat;\Fhat))\\frac{(\gamme^2\bar{r}\xi)^2}{2\lambda} \left(\frac{1}{n}
  %     + \frac{4\sqrt{\log(1/\delta)}}{\sqrt{2n}}\right), \]
  \[ 
    \CE(\qhat;\F) \geq \CE(\qhat;\Fhat) -
    \Omega_1/\lim_{\epsilon\to0^-}u'(\CE(\qhat;\Fhat)+\epsilon)\;,
  \]
  where
  \begin{gather*}
    \CE(\qhat;\F):=u^{-1}(\E_\F[u(R\cdot\qhat^T X)])\;,\\
    \CE(\qhat;\Fhat):=u^{-1}(n^{-1}\sum_{i=1}^n u(r_i\,\qhat^T x_i))\;,
  \end{gather*}
  and where
  \[
    \Omega_1 := \frac{\bar{r}^2 \xi^2}{2\lambda} \left(\frac{\gamma^2}{n} +
      \frac{(2\gamma^2+\gamma+1)\sqrt{\log(1/\delta)}}{\sqrt{2n}}\right)
  \] 
  with probability $1-\delta$,
\end{thm}

Our proof of Theorem \ref{thm:outsampleBound1} proceeds as follow. First, borrowing from
the terminology introduced by \cite{bousquet2002stability}, we show that the algorithm
which produces $\qhat$ from the sample set is $\beta$-stable. We then show that for any
$\qhat$ generated from a sample of $\F$, the amount of utility generated from implementing
the $\qhat$ decision necessarily lies on an interval of bounded size. Given that these two
conditions are satisfied, we can then rely on Bousquet-Ellisseef's out-sample error bound
theorem (typically used for inference problems) in order to establish out-of-sample
guarantees in terms of expected utility. By exploiting the concavity of $u(\cdot)$, we are
finally able to describe the implications in terms of certainty equivalent that are
expressed in our theorem.



%%% Local Variables:
%%% mode: latex
%%% TeX-master: "big_data_portfolio_optimization"
%%% End:

\subsection{Market efficiency and true optimal}
\newcommand{\EU}{{\bf EU}}
We now turn our attention to the suboptimality of the decision $\hat q$, specifically,
how well it performs in comparison to the optimal decision $q^\star =
\argmax_q\Psi(q)$. There are indeed some conditions on $u$ and on $M$. Consider a
risk-neutral utility $u(r)=r$. Then,
\[
  \E[u(R\,q^TX)] = q_1\E[RX_i]+\cdots+q_p\E[RX_p].
\]
Therefore, if we set $q_i = \infty$ when $\E(RX_i)>0$ and $q_i=-\infty$ when $\E(RX_i)<0$,
then $\CE(q)=\infty$ and no bounds on the suboptimality can be set. Likewise, if
$\pp\{RX_i>0\}=1$, we can again set $q_i=\infty$ and the same kind of divergence will
happen. This relates to the notion of arbitrage: we say a feature $X_i$ induces arbitrage
if $\pp\{RX_i>0\}=1$ or $\pp\{RX_i<0\}=1$.

\begin{thm}
  \label{thm_truopt}
  If $r=o(u(r))$, ie. $u(r)$ is sublinear and if no feature in the market induces
  arbitrage, then 
  \[
    \CE(q^\star) \geq \CE(\hat q) - \omega\cdot\grad(u^{-1})(\Psi(\hat q)),
  \]
  where
  \[
    \omega = \gamma\bar r\xi\|q^\star - \hat q\|,
  \]
  and $\omega$ is finite. 
\end{thm}

Following is the proof of the first part of Theorem \ref{thm_truopt}. Note first that 
\begin{align*}
  |\Psi(q^\star)-\Psi(\hat q)| &\leq |\E[u(R\,{q^\star}^T)] - \E[u(R\,\hat q^TX)]|\\
                             &\leq \E[|u(R{q^\star}^TX) - u(R\,\hat q^TX)|]\\
                             &\leq \gamma \E[R({q^\star}-\hat q)^TX]\\
                             &\leq \gamma\bar r\xi\|q^\star - \hat q\|=\omega,
\end{align*}
so that $\Psi(q^\star) \geq \Psi(\hat q) - \omega$. We can invert this result back in the
result space using the same method as in Theorem \ref{thm:outsampleBound1} and the result
follows.

Next, we show that $\omega$ is bounded by proving that $\|q^\star\|$ is finite. Since the
other values are also bounded the second part of Theorem \ref{thm_truopt} follows. 

Let $Z(q)=R\,q^TX\subsetsim\real$. Instead of optimizing $q$ on $\real^p$, we can optimize
its scale by taking $s^\star = \argmax_{s>0}g(s)$ where $g(s)$ is the solution of
\begin{align*}
  \maximizeEquationSt{\E[u(sZ(q))]}[\|q\|= 1].
\end{align*}
Let $q\in\real^p$ such that $\|q\|=1$. Since no feature induce arbitrage, it follows that,
for any $q$, there exists a $\delta_q<0$ such that $\pp\{Z(q)<\delta_q\}=\varrho>0$. Now,
let $B(q)$ be a discrete random variable with two states such that
$\pp\{B(q)=\delta_q\}=1-\pp\{B=\bar r\xi\}=\varrho$. Since $|Z(q)|<\bar r\xi$, we have
that $\pp\{B(q)\geq r\} \geq \pp\{Z(q)\geq r\}$, so that $\E B(q) \geq \E Z(q)$, from which
it follows that $\E[u(sB(q))]\geq \E[u(sZ(q))]$. But, by the sublinearity asumption on
$u$, 
\[
  \lim_{s\to\infty}\E[u(sB(q))] = \lim_{s\to\infty}\big(\varrho
  u(s\delta_q)+(1-\varrho)u(s\bar r\xi)\big) = -\infty.
\]
And therefore, $\lim_{s\to\infty}\E[u(sZ(q))]=-\infty$ for all $q$, which shows that
$s^\star$, and therefore $\|q^\star\|$, is bounded.


%%% Local Variables:
%%% mode: latex
%%% TeX-master: "big_data_portfolio_optimization"
%%% End:

\subsection{Big Data Phenomenon}\label{sec:bigdata}

In this section, we question how realistic assumption \ref{ass:X} is in a big data
context. In particular, we expose two sets of natural conditions for the generation of the
side information vector $X$ that leads to motivating the use of a support set which
diameter grow proportionally to the square root of $p$.

\begin{ex}
  Consider a case where every terms of $X$ are independant from each other, while each
  $X_i$ has a mean $\Expect[X_i]=0$, a variance $\Var[X_i]=1$, and are supported on their
  respective intervals $\Prob(X_i\in [-\nu, \nu])=1$ for all $i$. By Hoeffding's
  inequality, one can establish that
  \[
    \Prob\left(\left|\|X\|^2 - \sum_{i=1}^p \Expect[X_i^2]\right| \leq
      \sqrt{2p\ln(\delta/2)\nu^2}\right) \geq 1-\delta
  \]
  so that $|\|X\|^2 \in [p- \sqrt{2p\ln(\delta/2)\nu^2}, p+ \sqrt{2p\ln(\delta/2)\nu^2}]$
  with probability $1-\delta$. Hence, any ball of fixed radius $\xi$ will contain $X$ with
  a probability that assymptotically converges to zero as $p$ increases, more specifically
  $\Prob(\|X\|^2\leq \xi^2)\leq 2\exp(-2p(1-\xi^2/\sqrt{p})^2/\nu^2)$. On the other hand,
  this inequality somehow also prescribes that the diameter of the support $\Sx$ should
  increase proportionally to $\sqrt{p}$ in order to still contain $X$ with high
  probability as $p$ increases.
\end{ex}

\begin{ex}
  Consider a similar case as above but where the independance assumption is dropped. In
  this context, although we might not have as much of a strong argument to discredit the
  use of a constant diameter for $\Sx$, there is still a good motivation for employing a
  radius that grows proportionally to $\sqrt{p}$. Namely, if each $X_i$ has a mean
  $\Expect[X_i]=0$ and a variance $\Var[X_i]=1$ then the random variable $Z:=\|X\|^2$ is
  necessarily positive with an expected value of $p$. Based on Markov inequality, this
  implies that with probability $1-\delta$, we have that $\|X\|\leq \sqrt{p/\delta}$.
\end{ex}

Since we believe these two examples provide strong arguments for replacing assumption
\ref{ass:X} with the assumption that it is within a ball of radius $\xi\sqrt{p}$, we
reformulate our previous two results as follows.

\begin{coro}\label{coro:outsampleBoundBigData}
  Given that assumptions \ref{ass:R} and \ref{ass:u} are satisfied, and that
  $\Prob(\|X\|\leq \xi\sqrt{p})=1$, the certainty equivalent of the out-of-sample
  performance is at most $O(p/\sqrt{n})$ worse than the in-sample one. Specifically,
  \[
    \CE(\qhat;\F) \geq \CE(\qhat;\Fhat) - \Omega_2/
    \lim_{\epsilon\to0^-}u'(\CE(\qhat;\Fhat)+\epsilon)\;,
  \]
  where
  \[
    \Omega_2 := \frac{\gamma\bar{r}^2\xi^2}{\lambda} \left(\frac{\gamma p}{2n} +
      \frac{(1+\gamma)p\sqrt{\log(1/\delta)}}{\sqrt{2n}}\right)
  \]
  and\\
  \Erick{missing the big data result for suboptimality}\\
  with probability $1-\delta$, 
\end{coro}

%This should be in conclusion of this section
Note that assumption \ref{ass:X} was inspired by \cite{rudin2015big} who also studied
assymptotic properties of a regularized decision problem in its Big data regime, i.e. when
$n$ and $p$ go to infinity simultaneously. Our analysis indicate that the convergence in
accuracy that is reported there for regime $p\propto n$ might be misleading for many
practical problems.  In particular, our new results states that asymptotic convergence in
accuracy when the sample set is large can only occur if $p=o(\sqrt{n})$. The numerical
experiments that follow will empirically confirm this important insight.



%%% Local Variables:
%%% mode: latex
%%% TeX-master: "big_data_portfolio_optimization"
%%% End:



%%% Local Variables:
%%% mode: latex
%%% TeX-master: "big_data_portfolio_optimization"
%%% End:

% \section{Empirical Results}

We now present empirical results driven by our model. We first show how the big-data
phenomenon arises for different regimes $p=O(n^k)$ and then we move on using comparative
analysis on a real dataset using alternatively only a $u$-loss regression with
regularization qon the returns, market statistics such as market volatility as measured on
the last $m$ days and previous average returns, and finally with a NLP reprensentation of
market financial news.


\subsection{Synthetic distributions}

Our experminental setup consists of a normal distribution $R\sim\normal(\mu=0,\sigma=10)$
in the return margin and of long tailed $t$-Distribution$(\nu=4)$.


%%% Local Variables:
%%% mode: latex
%%% TeX-master: "big_data_portfolio_optimization"
%%% End:

\section{Conclusion}
\label{sec:conclusion}


En conclusion, nous jugeons que l'algorithme d'investissement présenté dans ce mémoire est
d'un grand intérêt et ce, pour plusieurs raisons.

En premier lieu, il permet de représenter naturellement le niveau de risque auquel un
investisseur est prêt à s'exposer grâce à la maximisation de la fonction d'utilité. Cet
algorithme permet de plus de traiter de façon relativement simple toutes sortes de
variables de marché que l'investisseur pourrait estimer intéressantes. Par apprentissage,
on s'attend alors à ce que l'algorithme décide de lui même sur quelles variables de marché
devraient reposer les décisions d'investissement. De plus, la méthode de noyaux permet de
rendre compte de situations complexes non linéaires dans la relation entre ces variables
de marché et le rendement aléatoire.

Finalement, les garanties statistiques sur l'erreur de généralisation de l'équivalent
certain, au delà de l'aspect purement numérique, et d'ailleurs souvent extrêmement
relâchée, offrent néanmoins une bonne idée de leur comportement en fonction de la taille
de l'échantillon employé et du nombre de variables de marché considérées. En particulier,
nous avons tenté de mettre en lumière comment l'interaction entre ces deux quantités peut
donner lieu à des situations périlleuses. Nous avons également établi rigoureusement un
intervalle de vitesse dans lequel un investisseur peut chercher à réduire son erreur de
sous optimalité, tout en maintenant de garanties hors échantillon.

D'un point de vue théorique aussi, ce mémoire a cherché à illustrer comment la gestion de
portefeuille peut donner lieu à une autre forme d'optimisation où la fonction objectif
n'est pas une régression aux moindres carrés sur certains paramètres d'un modèle donné,
mais bien la fonction utilité elle même. À notre avis, il s'agit là d'une différence
profonde de ce qui est typiquement observé en statistiques ou dans des problèmes
d'inférence classiques.

Cependant, ces résultats se font faits au prix de plusieurs hypothèses, certaines assez
contraignantes. On peut penser à la stationarité de la loi de marché. Bien que ce soit une
hypothèse couramment faite dans un contexte théorique, elle n'en demeure pas moins
problématique aussitôt qu'on cherchera à appliquer les idées présentées dans ce mémoire
dans un contexte appliqué. Les idées assez proposées dans
\cite{kuznetsov2017generalization} nous semblent de cette façon dignes d'intérêt,
puisqu'elles offrent le même genre de garantie de généralisation à des processus mélangés
\textsl{(mixing processes)} non stationnaires. Il s'agit en fait de limiter la façon dont
la loi de $M_t$ peut changer au cours d'un intervalle de temps $\delta$ donné.

Mais au delà de ces considérations théoriques, notre algorithme souffre évidemment du fait
qu'il ne considère que les portefeuilles dotés d'un seul actif. On peut néanmoins
généraliser assez facilement la fonction objectif à un portefeuille à $k$ actifs dans le
cas d'une décision matricielle $Q$:
\begin{equation}
  \maximizeEquation[Q \in \Re^{k \times p}]{n^{-1}\sumi u(r_i^TQ x_i) - \frac{\lambda}{2}\|Q\|^2_F.}
\end{equation}
Par contre, il n'est pas clair quel rôle le nombre d'actifs $k$ viendrait jouer dans
l'erreur de généralisation. Ceci étant dit, il y a probablement moyen de s'inspirer des
SVM multiclasses qui disposent justement de telles garanties. On peut noter au passage que
ces garanties se dégradent généralement selon $\bigO(k^2)$ où $k$ représente le nombre de
classes possibles. Il est donc possible qu'un tel portefeuille multi-actifs soit exposé au
même type de risque de généralisation. 



%%% Local Variables:
%%% mode: latex
%%% TeX-master: "memoire"
%%% End:

\section{Appendix}

\subsection{Proof of Theorem \ref{thm:outsampleBound1}}

In this proof, we will employ a theorem made famous by Bousquet-Ellisseef to analyse
relevant asymptotic statistical properties of the following estimator.
\begin{definition}
  Let $\qhatOp:\Re^{(p+1)\times n}\rightarrow \Re^p$ be the procedure that generates the
  optimal solution of problem \eqref{EUFhatReg} based on a sample set
  $\{(x_i^1,r_i^1)\}_{i=1}^n$.
\end{definition}

We start by presenting two lemmas that establish some important properties of problem
\eqref{EUFhatReg}.
\begin{lemma}\label{beta-bound}
  When assumptions \ref{ass:R} and \ref{ass:u} are satisfied, the estimator
  $\bold{\qhat}(\cdot)$ has $\beta$-stability with
  $ \beta = \frac{(\gamma\bar r\xMax)^2}{2\lambda n}$. Namely, for any two sample sets
  $\Sn^1:=\{(x_i^1,r_i^1)\}_{i=1}^n$ and $\Sn^2:=\{(x_i^2,r_i^2)\}_{i=1}^n$ that are
  exactly identical except for the $j$-th sample, i.e., $(x_i^1,r_i^1)=(x_i^2,r_i^2)$ for
  all $i\neq j$, the following holds:
  \[
    |u(r\,\qhatOp(\Sn^1)^T x) - u(r\,\qhatOp(\Sn^2)^T x)| \leq \beta\,,\,\forall\,x\in\Sx\,,\,\forall\,r\in\Sr\;.
  \]
\end{lemma}

\begin{proof}
  First, following the terminology presented in \cite{bousquet2002stability} (see
  Definition 19), we can establish that $\qhatOp(\cdot)$ has $\sigma$-admissibility of
  $\gamma\bar{r}$. This is simply done by exploiting the fact that $\Sr$ is bounded and
  that $u(\cdot)$ is Lipschitz continuous. The detailed derivations consider that for any
  pair $(q_1,q_2)\in\Re^p\times\Re^p$, one has that
  \[ 
    |u(r\,q_1^T x) - u(r\,q_2^T x)| \leq \gamma |rq_1^Tx - rq_2^Tx|\leq \gamma\bar{r}\
    |q_1^Tx - q_2^Tx| \,,\,\forall\,r\in\Sr\,,\,\forall\,x\in\Sx\;.
  \]
  The $\beta$-stability of $\bold{\qhat}(\cdot)$ then follows directly from Theorem 22 in
  \cite{bousquet2002stability}.
\end{proof}

\begin{lemma}\label{u-bound}
  When assumptions \ref{ass:R}, \ref{ass:X} and \ref{ass:u} are satisfied, the maximum
  difference in amount of utility attained by implementing two investment strategies
  obtained using different sample sets $\Sn^1$ and $\Sn^2$ is bounded by
  \[
    |u(r\,\qhatOp(\Sn^1)^T x)-u(r\,\qhatOp(\Sn^2)^T x)| \leq \urange :=
    \frac{(\gamma+1)\xi^2
      \bar{r}^2}{2\lambda}\,,\,\forall\,x\in\Sx\,,\,\forall\,r\in\Sr\;.
  \]
\end{lemma}

\begin{proof}
  This proof relies mostly on demonstrating that $\|\qhatOp(\Sn)\|\leq B$ for some $B>0$
  with probability one for all possible sample sets $\Sn$. Indeed, when this is the case,
  then we have that
  \[
    |u(r\,\qhatOp(\Sn^1)^T x)-u(r\,\qhatOp(\Sn^2)^T x)| \leq u(\bar{r}\xi B)-u(-
    \bar{r}\xi B) \leq (\gamma + 1)\bar{r}\xi B\;.
  \] 
  In order to show that $\qhatOp(\Sn)$ is bounded, we reformulate problem
  \eqref{EUFhatReg} as follows
  \begin{eqnarray*}
    \maximize_{s\in\Re,v\in\Re^p} && \frac{1}{n}\sum_{i=1}^n u(s R_i\,X_i^T v) - \lambda s^2\\
    \st&& s\geq0\;,\;\|v\|=1\;,
  \end{eqnarray*}
  such that $\qhatOp(\Sn) =s^*\cdot v^*$ when $(s^*,v^*)$ is the pair of optimal
  assignments for this optimization problem. It is therefore clear that
  $s^*=\|\qhatOp(\Sn)\|$ and our proof reduces to establishing an upper bound for $s^*$.
  By recognizing that
  $s^*=\argmax_{s\geq 0} g(s):=\frac{1}{n}\sum_{i=1}^n u(s R_i\,X_i^T v^*) - \lambda s^2$
  and that $g(s)$ is a concave function, then it is necessarily the case that if there
  exists a $\bar{s}\geq 0$ such that $g(\cdot)$ is non-increasing at $\bar{s}$ then
  $s^* \leq \bar{s}$. We can actually show that this is the case for
  $\bar{s}:= \bar{r}\xi/(2\lambda)$ by upper bounding the impact of taking a step of
  $\Delta>0$:
  \begin{eqnarray*}
    g(\bar{s}+\Delta)-g(\bar{s}) &=& \frac{1}{n}\sum_{i=1}^n (u((\bar{s}+\Delta) R_i\,X_i^T v^*) - u(\bar{s} R_i\,X_i^T v^*) ) - \lambda ((\bar{s}+\Delta)^2-\bar{s}^2)\\
                                 &\leq& \frac{1}{n}\sum_{i=1}^n (u((\bar{s}+\Delta) |R_i\,X_i^T v^*|) - u(\bar{s} |R_i\,X_i^T v^*|) ) - \lambda ((\bar{s}+\Delta)^2-\bar{s}^2)\\
                                 &\leq & \frac{1}{n}\sum_{i=1}^n  \Delta R_i\,X_i^T v^* - \lambda (2\bar{s}\Delta + \Delta^2)\\
                                 &\leq &  \Delta \bar{r} \xi  - 2\lambda \bar{s}\Delta -  \Delta^2 = -  \Delta^2 \leq 0\;,
  \end{eqnarray*}
  where we first used the fact that $u(\cdot)$ is increasing, next that
  $u(y+\Delta)\leq u(y)+\Delta$ when $\Delta\geq 0$ since it is a concave function with a
  subgradient of one at zero.  Finally, we exploited assumptions \ref{ass:R} and
  \ref{ass:X}. This completes our proof.
\end{proof}

Having established the above properties, the following theorem follows directly from
Bousquet-Ellisseef Outsample Error Theorem \Erick{Is this the original name of the Theorem
  ?}. While we omit to describe the original theorem in this article for sake of
compactness, we refer interested readers to the form presented Theorem XXX \Erick{Theorem
  number in reference ?} in \cite{mohri2012foundations} for more details.

\begin{thm*}[Bousquet-Ellisseef Outsample Error Theorem]\label{thm:outsampleBound2}
  Given that assumptions \ref{ass:R}, \ref{ass:X}, and \ref{ass:u} are satisfied, then one
  has with confidence of $1-\delta$ that
  \[
    \Expect_\F[u(R\qhatOp(\Sn)^TX)] \geq \Expect_\F[u(R\qhatOp(\Sn)^TX)] - \beta -
    \left(2n\beta+\urange\right)\sqrt{\frac{\log(1/\delta)}{2n}}\;,
  \]
  where $\beta$ refers to the $\beta$-stability of $\qhatOp$ and $\hat{u}_{\mbox{abs}}$
  refers to a uniform bound $\Prob(|u(R\qhatOp(\Sn)X)|\leq \hat{u}_{\mbox{abs}})=1$.
  Overall, this reduces to
  $\Expect_\F[u(R\qhatOp(\Sn)^TX)] \geq \Expect_\F[u(R\qhatOp(\Sn)^TX)] -
  \Omega_1$. Hence, the out-of-sample performance in terms of expected utility of the
  investment policy $\qhatOp(\Sn)$ is at most $O(1/\sqrt{n})$ worse than the in-sample
  one.
\end{thm*}

We conclude this section by demonstrating how Theorem \ref{thm:outsampleBound1} follows
from Theorem \ref{thm:outsampleBound2}. In particular, by concavity of the utility
function, we have that
\[u(\CE(\qhat;\F)\leq u(\CE(\qhat;\Fhat))+(\CE(\qhat;\F)-\CE(\qhat;\Fhat))\partial
  u(\CE(\qhat;\Fhat))\;,\] where $\partial u(r)$ denotes any supergradient of $u(\cdot)$
at $r$. In particular, since $u(\cdot)$ is an increasing concave
$\lim_{\epsilon\to0^-}u'(\CE(\qhat;\Fhat)+\epsilon\geq 0$ is one of the supergradient at
$\CE(\qhat;\Fhat)$. Combining this inequality with the inequality presented in Theorem
\ref{thm:outsampleBound2}, we get
\[ u(\CE(\qhat;\Fhat)) - \Omega_1 \leq
  u(\CE(\qhat;\Fhat))+(\CE(\qhat;\F)-\CE(\qhat;\Fhat))\partial u(\CE(\qhat;\Fhat)\] so
that
\[ \CE(\qhat;\F) \geq \CE(\qhat;\Fhat) - \Omega_1/\partial u(\CE(\qhat;\Fhat))\] follows
since it was assumed that $u(\cdot)$ is strictly increasing. This completes the proof of
Theorem \ref{thm:outsampleBound1}.


%%% Local Variables:
%%% mode: latex
%%% TeX-master: "big_data_portfolio_optimization"
%%% End:

\subsection{Proof of Theorem \ref{thm_truopt}}

First, in order to tidy up the proof, let
\begin{gather*}
  \EU(q) := \E_F(u(R\cdot q^T X))\,;\\
  \EU_\lambda(q) := \E_F(u(R\cdot q^T X)) - \lambda\|q\|^2,
\end{gather*}
with $q^\star := \argmin_q \EU(q)$ and $q^\star_\lambda := \argmin_q \EU_\lambda(q)$.

\begin{thm}[Theorem 1 in \cite{sridharan2009fast}]\label{thm:shai}
  Given that assumptions \ref{ass:R}, \ref{ass:X}, and \ref{ass:u} are satisfied, and
  since $EU_\lambda$ is $2\lambda$-strongly convex, then one has with confidence of
  $1-\delta$ that
  \[
    -\lambda\|\hat q - q^\star_\lambda\| \geq \EU_\lambda(\hat q) - \EU_\lambda(q^\star_\lambda) \geq -\omega,
  \]
  where
  \[
    \omega = \frac{4\gamma^2\xi^2(32+\log(1/\delta))}{\lambda n}.
  \]
\end{thm}

Notice that Theorem \ref{thm:shai} implies with confidence $1-\delta$ that
\[
  \EU(\hat q) - \EU(q^\star_\lambda) \geq \lambda\bigl(\|\hat q\|^2 -
  \|q^\star_\lambda\|^2\bigr) -\omega
  \geq -\lambda\bigl(\|\hat q -
  q^\star_\lambda\|^2 +2\|\hat q\|\|\hat q - q^\star_\lambda\|\bigr) - \omega.
\]
As shown in Section \ref{sec:thm1_proof}, $\|\hat q\| \leq \bar r\xi/(2\lambda)$. Reusing
Theorem \ref{thm:shai}, we again obtain that
$\|\hat q - q^\star_\lambda\|^2 \leq \omega/\lambda$, and therefore
$\|\hat q - q^\star_\lambda\| \leq \sqrt{\omega/\lambda}$, with probability $1-\delta$
each, so that we end up with
\[
  \EU(\hat q) - \EU(q^\star_\lambda) \geq -2\omega - \bar r\xi\sqrt{\frac{\omega}{\lambda}}.
\]
with probability $1-3\delta$.

If now we let $\psi := \EU(q^\star) -\EU(q^\star_\lambda)$, we can then bound the
suboptimality of decision $\hat q$ with probability $1-3\delta$ in the following fashion:
\begin{align*}
  \EU(\hat q) &= \EU(q^\star) + \EU(\hat q) - \EU(q^\star_\lambda) + \EU(q^\star_\lambda)
                - \EU(q^\star)\\
              &\geq \EU(q^\star) - \psi - \bar r\xi\sqrt{\frac{\omega}{\lambda}} - 2\omega.
\end{align*}
The can finally invert this last result in terms of CE using the same trick as in Section
\ref{sec:thm1_proof}.

Next, we show that $\CE(q^\star;F)$ is bounded by proving that $\|q^\star\|$ is
finite. Since the other values are also bounded the second part of Theorem
\ref{thm_truopt} follows.

Let $Z(q)=R\,q^TX\subsetsim\real$. Instead of optimizing $q$ on $\real^p$, we can optimize
its scale by taking $s^\star = \argmax_{s>0}g(s)$ where $g(s)$ is the solution of
\begin{align*}
  \maximizeEquationSt{\E[u(sZ(q))]}[\|q\|= 1].
\end{align*}
Let $q\in\real^p$ such that $\|q\|=1$. Since no feature induce arbitrage, it follows that,
for any $q$, there exists a $\delta_q<0$ such that $\pp\{Z(q)<\delta_q\}=\varrho>0$. Now,
let $B(q)$ be a discrete random variable with two states such that
$\pp\{B(q)=\delta_q\}=1-\pp\{B=\bar r\xi\}=\varrho$. Since $|Z(q)|<\bar r\xi$, we have
that $\pp\{B(q)\geq r\} \geq \pp\{Z(q)\geq r\}$, so that $\E B(q) \geq \E Z(q)$, from which
it follows that $\E[u(sB(q))]\geq \E[u(sZ(q))]$. But, by the sublinearity asumption on
$u$, 
\[
  \lim_{s\to\infty}\E[u(sB(q))] = \lim_{s\to\infty}\big(\varrho
  u(s\delta_q)+(1-\varrho)u(s\bar r\xi)\big) = -\infty.
\]
And therefore, $\lim_{s\to\infty}\E[u(sZ(q))]=-\infty$ for all $q$, which shows that
$s^\star$, and therefore $\|q^\star\|$, is bounded.






%%% Local Variables:
%%% mode: latex
%%% TeX-master: "big_data_portfolio_optimization"
%%% End:


%%% Local Variables:
%%% mode: latex
%%% TeX-master: "big_data_portfolio_optimization"
%%% End:


\bibliographystyle{abbrv}
\bibliography{bibliography}


\end{document}


%%% Local Variables:
%%% mode: latex
%%% TeX-master: t
%%% End:






