\subsection{Proof of Theorem \ref{thm:outsampleBound1}}\label{sec:thm1_proof}

In this proof, we will employ a theorem made famous by Bousquet-Ellisseef to analyse
relevant asymptotic statistical properties of the following estimator.
\begin{definition}
  Let $\qhatOp:\Re^{(p+1)\times n}\rightarrow \Re^p$ be the procedure that generates the
  optimal solution of problem \eqref{EUFhatReg} based on a sample set
  $\{(x_i,r_i)\}_{i=1}^n$.
\end{definition}

We start by presenting two lemmas that establish some important properties of problem
\eqref{EUFhatReg}.
\begin{lemma}\label{beta-bound}
  When assumptions \ref{ass:R} and \ref{ass:u} are satisfied, the estimator
  $\bold{\qhat}(\cdot)$ has $\beta$-stability with
  $ \beta = \frac{(\gamma\bar r\xMax)^2}{2\lambda n}$. Namely, for any two sample sets
  $\Sn^1:=\{(x_i^1,r_i^1)\}_{i=1}^n$ and $\Sn^2:=\{(x_i^2,r_i^2)\}_{i=1}^n$ that are
  exactly identical except for the $j$-th sample, i.e., $(x_i^1,r_i^1)=(x_i^2,r_i^2)$ for
  all $i\neq j$, the following holds:
  \[
    |u(r\,\qhatOp(\Sn^1)^T x) - u(r\,\qhatOp(\Sn^2)^T x)| \leq \beta\,,\,\forall\,x\in\Sx\,,\,\forall\,r\in\Sr\;.
  \]
\end{lemma}

\begin{proof}
  First, following the terminology presented in \cite{bousquet2002stability} (see
  Definition 19), we can establish that $\qhatOp(\cdot)$ has $\sigma$-admissibility of
  $\gamma\bar{r}$. This is simply done by exploiting the fact that $\Sr$ is bounded and
  that $u(\cdot)$ is Lipschitz continuous. The detailed derivations consider that for any
  pair $(q_1,q_2)\in\Re^p\times\Re^p$, one has that
  \[ 
    |u(r\,q_1^T x) - u(r\,q_2^T x)| \leq \gamma |rq_1^Tx - rq_2^Tx|\leq \gamma\bar{r}\
    |q_1^Tx - q_2^Tx| \,,\,\forall\,r\in\Sr\,,\,\forall\,x\in\Sx\;.
  \]
  The $\beta$-stability of $\bold{\qhat}(\cdot)$ then follows directly from Theorem 22 in
  \cite{bousquet2002stability}.
\end{proof}

\begin{lemma}\label{u-bound}
  When assumptions \ref{ass:R}, \ref{ass:X} and \ref{ass:u} are satisfied, the maximum
  difference in amount of utility attained by implementing two investment strategies
  obtained using different sample sets $\Sn^1$ and $\Sn^2$ is bounded by
  \[
    |u(r\,\qhatOp(\Sn^1)^T x)-u(r\,\qhatOp(\Sn^2)^T x)| \leq \urange :=
    \frac{(\gamma+1)\xi^2
      \bar{r}^2}{2\lambda}\,,\,\forall\,x\in\Sx\,,\,\forall\,r\in\Sr\;.
  \]
\end{lemma}

\begin{proof}
  This proof relies mostly on demonstrating that $\|\qhatOp(\Sn)\|\leq B$ for some $B>0$
  with probability one for all possible sample sets $\Sn$. Indeed, when this is the case,
  then we have that
  \[
    |u(r\,\qhatOp(\Sn^1)^T x)-u(r\,\qhatOp(\Sn^2)^T x)| \leq u(\bar{r}\xi B)-u(-
    \bar{r}\xi B) \leq (\gamma + 1)\bar{r}\xi B\;.
  \] 
  In order to show that $\qhatOp(\Sn)$ is bounded, we reformulate problem
  \eqref{EUFhatReg} as follows
  \begin{eqnarray*}
    \maximize_{s\in\Re,v\in\Re^p} && \frac{1}{n}\sum_{i=1}^n u(s R_i\,X_i^T v) - \lambda s^2\\
    \st&& s\geq0\;,\;\|v\|=1\;,
  \end{eqnarray*}
  such that $\qhatOp(\Sn) =s^*\cdot v^*$ when $(s^*,v^*)$ is the pair of optimal
  assignments for this optimization problem. It is therefore clear that
  $s^*=\|\qhatOp(\Sn)\|$ and our proof reduces to establishing an upper bound for $s^*$.
  By recognizing that
  $s^*=\argmax_{s\geq 0} g(s):=\frac{1}{n}\sum_{i=1}^n u(s R_i\,X_i^T v^*) - \lambda s^2$
  and that $g(s)$ is a concave function, then it is necessarily the case that if there
  exists a $\bar{s}\geq 0$ such that $g(\cdot)$ is non-increasing at $\bar{s}$ then
  $s^* \leq \bar{s}$. We can actually show that this is the case for
  $\bar{s}:= \bar{r}\xi/(2\lambda)$ by upper bounding the impact of taking a step of
  $\Delta>0$:
  \begin{eqnarray*}
    g(\bar{s}+\Delta)-g(\bar{s}) &=& \frac{1}{n}\sum_{i=1}^n (u((\bar{s}+\Delta) R_i\,X_i^T v^*) - u(\bar{s} R_i\,X_i^T v^*) ) - \lambda ((\bar{s}+\Delta)^2-\bar{s}^2)\\
                                 &\leq& \frac{1}{n}\sum_{i=1}^n (u((\bar{s}+\Delta) |R_i\,X_i^T v^*|) - u(\bar{s} |R_i\,X_i^T v^*|) ) - \lambda ((\bar{s}+\Delta)^2-\bar{s}^2)\\
                                 &\leq & \frac{1}{n}\sum_{i=1}^n  \Delta R_i\,X_i^T v^* - \lambda (2\bar{s}\Delta + \Delta^2)\\
                                 &\leq &  \Delta \bar{r} \xi  - 2\lambda \bar{s}\Delta -  \Delta^2 = -  \Delta^2 \leq 0\;,
  \end{eqnarray*}
  where we first used the fact that $u(\cdot)$ is increasing, next that
  $u(y+\Delta)\leq u(y)+\Delta$ when $\Delta\geq 0$ since it is a concave function with a
  subgradient of one at zero.  Finally, we exploited assumptions \ref{ass:R} and
  \ref{ass:X}. This completes our proof.
\end{proof}

Having established the above properties, the following theorem follows directly from
Theorem \ref{thm:outsampleBound2}. While we omit to describe the original theorem in this
article for sake of compactness, we refer interested readers to the form presented in
Theorem 11.1 of \cite{mohri2012foundations} for more details.

\begin{thm}[Bousquet-Ellisseef Outsample Error Theorem]\label{thm:outsampleBound2}
  Given that assumptions \ref{ass:R}, \ref{ass:X}, and \ref{ass:u} are satisfied, then one
  has with confidence of $1-\delta$ that
  \[
    \E_{\F}[u(R\cdot\qhatOp(\Sn)^TX)] \geq \E_{\hat\F}[u(R\cdot\qhatOp(\Sn)^TX)] - \beta -
    \left(2n\beta+\hat{u}_{\mbox{abs}}\right)\sqrt{\frac{\log(1/\delta)}{2n}}\;,
  \]
  where $\beta$ refers to the $\beta$-stability of $\qhatOp$ and $\hat{u}_{\mbox{abs}}$
  refers to a uniform bound $\Prob(|u(R\qhatOp(\Sn)X)|\leq \hat{u}_{\mbox{abs}})=1$.
  Overall, this reduces to
  $\E_{\F}[u(R\cdot\qhatOp(\Sn)^TX)] \geq \E_{\hat\F}[u(R\cdot\qhatOp(\Sn)^TX)] - \Omega_1$. Hence,
  the out-of-sample performance in terms of expected utility of the investment policy
  $\qhatOp(\Sn)$ is at most $O(1/\sqrt{n})$ worse than the in-sample one.
\end{thm}

We conclude this section by demonstrating how Theorem \ref{thm:outsampleBound1} follows
from Theorem \ref{thm:outsampleBound2}. In particular, by concavity of the utility
function, we have that
\[
  u(\CE(\qhat;\F))\leq u(\CE(\qhat;\Fhat))+(\CE(\qhat;\F)-\CE(\qhat;\Fhat))\,\partial
  u(\CE(\qhat;\Fhat))\;,
\] 
where $\partial u(r)$ denotes any supergradient of $u(\cdot)$ at $r$. In particular, since
$u(\cdot)$ is an increasing concave function, it follows that
$\lim_{\epsilon\to0^-}u'(\CE(\qhat;\Fhat)+\epsilon)\geq 0$ is one of the supergradient at
$\CE(\qhat;\Fhat)$. Combining this inequality with the inequality presented in Theorem
\ref{thm:outsampleBound2}, we get
\[ 
  u(\CE(\qhat;\Fhat)) - \Omega_1 \leq
  u(\CE(\qhat;\Fhat))+(\CE(\qhat;\F)-\CE(\qhat;\Fhat))\partial u(\CE(\qhat;\Fhat))
\]
so that
\[
  \CE(\qhat;\F) \geq \CE(\qhat;\Fhat) - \Omega_1/\partial u(\CE(\qhat;\Fhat))
\]
follows since it was assumed that $u(\cdot)$ is strictly increasing. This completes the
proof of Theorem \ref{thm:outsampleBound1}.


%%% Local Variables:
%%% mode: latex
%%% TeX-master: "big_data_portfolio_optimization"
%%% End:
