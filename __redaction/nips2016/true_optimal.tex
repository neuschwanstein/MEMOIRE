\section{Suboptimality performance bounds}\label{sec:sub}

We now turn our attention to the suboptimality of the problem, \ie, we would like to
understand the behaviour of the performance of the empirical investment policy $\hat q$
compared to the optimal policy $q^\star := \argmax_q \E_F[u(R\cdot q^T X)]$. It is
important to realize that in general, there are situations in which the optimal
performance according of \eqref{EUF} could be unbounded. Thus, if one wishes to establish
a bound on the sub-optimality of an investment policy, it is necessary to impose
additional assumptions on the class of problem that he is facing. The two following
examples motivate these assumptions.

\begin{ex}
  Consider for instance a risk neutral investor, \ie, such that $u(r)=r$ and suppose
  $\E X_i=0$. The expected utility simply becomes
  \[
    \E_F[u(R\cdot q^T X)] = \sum_{i=1}^{n}q_i\Cov(R,X_i).
  \]
  If we simply let $\bar q_i = \Cov(R,X_i)$, it follows immediately that the expected
  utility of $\alpha \bar q$ can become arbitrarily large as $\alpha$ goes to infinity.
\end{ex}

\begin{ex}
  Consider another example in which there exist a $j$ for which feature $X_j$ induces
  arbitrage over $F$, namely that $\pp\{RX_j>0\} = 1$. In such a case, if we let
  $\bar q_i = 1$ only when $i=j$ and otherwise zero, then the expected utility of
  $\alpha\bar q$ can once again take an arbitrarily large value as $\alpha$ goes to
  infinity.
\end{ex}

Given those two examples, we now introduce two new assumptions that will ensure that
problem \eqref{EUF} is bounded, \ie, it has a finite optimal solution.

\begin{assumption}\label{ass:usublinear}
  The utility function is sublinear, \ie, $u(r) = o(r)$. 
\end{assumption}

\begin{assumption}\label{ass:arbitrage}
  The side information $X$ induces no arbitrage opportunities, that is, for all $X_i$,
  $\pp\{RX_i < 0\} > 0$ and $\pp\{RX_i > 0\} > 0$.
\end{assumption}

In a financial context, assumption \ref{ass:usublinear} is certainly realistic since a
financial investor behaviour is usually taken to be risk averse, thus implying asumption
\ref{ass:usublinear}.  As for assumption \ref{ass:arbitrage}, this notion or arbitrage relates
directly to the notion or market efficiency, and in particular to the semi-strong version
of it, which states that it should be impossible for an investor to constantly beat the
market using publicly available information. See \cite{malkiel1970efficient} and
\cite{fama1991efficient} for more details.

\begin{thm}
  \label{thm_truopt}
  Given that assumptions \ref{ass:R}, \ref{ass:X}, \ref{ass:u} are satisfied, the
  suboptimality of the policy $\hat q$ can be expressed with confidence $1-\delta$ by
  \[
    \CE(\hat q;F) \geq \CE(q^\star;F) - \Omega_2/\lim_{\epsilon\to0^{-}}u'(CE(\hat q;F)+\epsilon)\;,
  \]
  where
  \[
    \Omega_2 = \lambda\|q^\star\|^2 +\frac{8\gamma^2\xi^2(32+\log(1/\delta))}{n\lambda} +\frac{2\gamma\bar
    r\xi^2}{\lambda}\sqrt{\frac{32+\log(1/\delta)}{n}}\;.
  \]
%  and $\psi:=E_F[u(R\cdot {q^{\star}}^T X)] - E_F[u(R\cdot {q^\star_\lambda}^TX)]$.
  Furthermore, if assumptions \ref{ass:usublinear} and \ref{ass:arbitrage} are satisfied, then
  $\CE(q^\star;F)$ is finite. 
\end{thm}

The first term in $\Omega_2$ shows that, unless the regularization constant $\lambda$ is
brought to zero as $n$ increases, the empirical maximization problem \eqref{EUFhatReg}
will asymptotically converge toward a constant suboptimality bound based on the particular
market distribution $F$ and on $\lambda$. The two other terms in $\Omega_2$ show that this
bound will be reached at a $O(1/\sqrt{n})$ rate in the same fashion as with Theorem
1. Therefore, the best suboptimality performance that can be hoped to be reached is at
most $-\lambda\|q^\star\|^2 /\lim_{\epsilon\to0^{-}}u'(CE(\hat q;F)+\epsilon)$.


%%% Local Variables:
%%% mode: latex
%%% TeX-master: "big_data_portfolio_optimization"
%%% End:
