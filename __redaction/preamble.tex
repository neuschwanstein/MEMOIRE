\usepackage[utf8]{inputenc}
\usepackage[T1]{fontenc}
\usepackage[greek,french]{babel}
\usepackage{textgreek}
\usepackage{times}
\usepackage{amsmath}
\usepackage{amsthm,amsfonts}
\usepackage{graphicx}
\usepackage{mathtools}
\usepackage{bm}
\usepackage{xparse}
\usepackage{mathrsfs}
\usepackage{eufrak}
\usepackage{parskip}
\usepackage{booktabs}
\usepackage{siunitx}
\usepackage{caption}
\usepackage{subcaption}
\usepackage{newfloat}
\usepackage{etoolbox}
\usepackage{braket}
\usepackage{xspace}
\usepackage{kbordermatrix}
\renewcommand{\kbldelim}{(}
\renewcommand{\kbrdelim}{)}
\allowdisplaybreaks


% \DeclareFloatingEnvironment[
%     fileext=los,
%     listname=List of Schemes,
%     name=Exemple,
%     placement=tbhp,
%     within=section,
%     ]{numex}

% \setcounter{tocdepth}{4}
    
\usepackage[font={small}]{caption}
\captionsetup{width=0.9\textwidth}
% \newcommand{\ccaption}[2][]{\caption{\textsc{#1.}~p#2}}

\newcounter{numex}[section]
\newenvironment{numex}[2][]
{
  \bigskip\refstepcounter{numex}Exemple~\thenumex.~\textsc{#2}~--
}{}
    
\newcommand\figref[1]{Fig.~\ref{#1} (p.~\pageref{#1})}
\newcommand\lemref[1]{Lemme \ref{#1}}


% KERNEL
\newcommand{\lag}{\mathscr{L}}
\newcommand{\sumij}{\sum_{i,j=1}^n}

% À classer.

\DeclareMathOperator*{\argmax}{arg\,max}
\DeclareMathOperator*{\argmin}{arg\,min}
\DeclareMathOperator*{\maximize}{\mbox{maximiser}}
\DeclareMathOperator*{\minimize}{\mbox{minimiser}}
\DeclareMathOperator{\epi}{epi}
\DeclareMathOperator{\diag}{diag}
\DeclareMathOperator{\dom}{dom}
\DeclareMathOperator{\tr}{tr}

\DeclareMathOperator{\sign}{sign}
\DeclareMathOperator{\Pdim}{Pdim}

\DeclareMathOperator{\E}{\mbox{\textbf{\textit{E}}}}
\DeclareMathOperator{\CE}{\mbox{\textbf{\textit{CE}}}}
\DeclareMathOperator{\nCE}{\mbox{\textit{CE}}}
\DeclareMathOperator{\hCE}{\widehat\CE}
\DeclareMathOperator{\nhCE}{\widehat\nCE}
\DeclareMathOperator{\nEU}{\mbox{\textit{EU}}}
\DeclareMathOperator{\EU}{\mbox{\textbf{\textit{EU}}}}
\DeclareMathOperator{\hEU}{\widehat\EU}
\DeclareMathOperator{\nhEU}{\widehat\nEU}
% \newcommand{\hEN}{\widehat{\bm{E1}}}
\DeclareMathOperator{\EN}{\mbox{\textbf{\textit{E1}}}}
\DeclareMathOperator{\hEN}{\widehat\EN}
\DeclareMathOperator{\hE}{\hat\E}
\DeclareMathOperator{\pp}{\bm P}
% \newcommand{\Var}{\bm{Var}}
\DeclareMathOperator{\Var}{\mbox{\textbf{\textit{Var}}}}
\DeclareMathOperator{\Cov}{\mbox{\textbf{\textit{Cov}}}}
\DeclareMathOperator{\Corr}{\mbox{\textbf{\textit{Corr}}}}
\newcommand{\rn}{\mbox{\textbf{\textit{1}}}}
\DeclarePairedDelimiter\inp{\langle}{\rangle}
\newcommand{\coloneq}{\coloneqq}
% \newcommand{\Cov}{\bm{Cov}}
% \DeclarePairedDelimiter\floor{\lfloor}{\rfloor}

\newcommand{\LEU}{\mbox{\textit{LEU}}}

\newcommand{\qu}{\hat q_u}
\newcommand{\qn}{\hat{q}_1}

\newcommand{\ie}{\textit{i.e.}}
\newcommand{\eg}{\textit{e.g.}}
\newcommand{\iid}{i.i.d.}

\newcommand{\ts}{\textsuperscript}
% \renewcommand{\figref}[1]{Fig.~\ref{#1}}

\newcommand{\qs}{q^\star}
\newcommand{\qsl}{q^\star_\lambda}
\newcommand{\tqsl}{\tilde q^\star_\lambda}
\newcommand{\qh}{\hat q}

\newcommand{\X}{\bm{X}}
\newcommand{\R}{\bm{R}}
\newcommand{\Ut}{\bm{U}}
\newcommand{\Q}{\bm Q}
\newcommand{\T}{\bm T}
\newcommand{\M}{\bm M}
\renewcommand{\S}{\mathcal S}
\renewcommand{\H}{\bm H}
\renewcommand{\P}{\bm{P}}
\newcommand{\alg}{\mathcal{Q}}
\newcommand{\au}{\mathcal{U}}
\newcommand{\bigO}{\mathcal O}


% \newcommand{\nq}[1]{\|#1\|_{\Q}}
\newcommand{\nq}[1]{\|#1\|}
% \renewcommand{\O}[1]{O\left(#1\right)}

\newcommand{\rad}{\hat{\mathfrak{R}}}
\newcommand{\radd}{\mathcal{R}}
\newcommand{\iso}{\simeq}
\newcommand{\dd}{\partial}
\newcommand{\real}{\mathscr{R}}
\renewcommand{\Re}{\real}
\newcommand{\normal}{\mathscr{N}}
\newcommand{\trueRisk}{R_{\mathrm{true}}}
\newcommand{\uInv}{u^{-1}}
\newcommand{\qHat}{{\hat q}}
% \newcommand{\xMax}{X_{\max}}
\newcommand{\xMax}{\xi}
\newcommand{\grad}{\nabla}
\newcommand{\sumi}{\sum_{i=1}^n}
\newcommand{\sumjn}{\sum_{j=1}^n}
\newcommand{\sumj}{\sum_{j=1}^p}
\newcommand{\sumjt}{\sum_{j=1}^{\tilde p}}
\newcommand{\rmax}{\bar r}
\newcommand{\qsh}{q^\star_\lambda}
% \newcommand{\dd}{\partial}

% \newcommand{\pp}{P}
% \newcommand{\CE}{\bm{CE}}
\newcommand\subsetsim{\mathrel{%
  \ooalign{\raise0.2ex\hbox{$\subset$}\cr\hidewidth\raise-0.8ex\hbox{\scalebox{0.9}{$\sim$}}\hidewidth\cr}}}

\def\rcurs{{\mbox{$\resizebox{.09in}{.08in}{\includegraphics[trim= 1em 0 14em
        0,clip]{ScriptR}}$}}}
\newcommand{\rf}{\rcurs_f}

\renewcommand{\rf}{r_0}

\newcommand{\starsection}{\vspace{1em}\begin{center}$\star\quad\star\quad\star$\end{center}\vspace{1em}}

% http://tex.stackexchange.com/questions/111551/recursive-multiple-subscript-and-superscript-with-xparse
\NewDocumentCommand{\minimizeEquationSt}{m >{\SplitList{;}}O{}}
{\begin{array}{rl}\minimize & \displaystyle{#1}\\\mbox{tel que}
    \ProcessList{#2}{\stCommand}\end{array}}

\NewDocumentCommand{\maximizeEquationSt}{m >{\SplitList{;}}O{}}
{\begin{array}{rl}\minimize & \displaystyle{#1}\\\mbox{tel que}
    \ProcessList{#2}{\stCommand}\end{array}}

\NewDocumentCommand{\stCommand}{m}{& #1}

% \newcommand{\minimizeEquation}[1]{\begin{array}{ll}\minimize & \displaystyle{#1}\end{array}}
% \newcommand{\maximizeEquation}[1]{\begin{array}{ll}\maximize & \displaystyle{#1}\end{array}}

\newcommand{\minimizeEquation}[2][]{%
  \begin{array}{ll}
    \displaystyle{\minimize_{#1}} & \displaystyle{#2}
  \end{array}
}

\newcommand{\maximizeEquation}[2][]{%
  \displaystyle{\begin{array}{ll}
    \displaystyle{\maximize_{#1}} & \displaystyle{#2}
  \end{array}}
}

\newcommand{\comment}[1]{\textbf{[#1]}}
\newcommand{\Erick}[1]{\textbf{[Erick says: #1]}}
\newcommand{\Thierry}[1]{\textbf{[Thierry says: #1]}}
\newcommand{\todo}[1]{\textbf{[Todo:} #1\textbf{]}}

\newcommand{\cit}{\textit{[Citation needed]}}
\newcommand{\reph}[1]{#1\textsuperscript{[Rephrase]}}
\newcommand{\nec}{\textsuperscript{[Nécessaire?]}}

\theoremstyle{plain}
\newtheorem{prop}{Proposition}
\newtheorem{thm}{Théorème}
\newtheorem{coro}{Corollary}
\newtheorem*{thm*}{Théorème}
\newtheorem{claim}{Claim}
\newtheorem{lemma}{Lemma}
\newtheorem{assumption}{Hypothèse}
\renewcommand*{\proofname}{Démonstration}

\theoremstyle{definition}
\newtheorem*{deff}{Définition}
\newtheorem*{ex}{Example}
\newtheorem{lemme}{Lemme}

\theoremstyle{remark}
\newtheorem*{rem}{Remarque}

% \begingroup
%     \makeatletter
%     \@for\theoremstyle:=definition,remark,plain\do{%
%         \expandafter\g@addto@macro\csname th@\theoremstyle\endcsname{%
%             \addtolength\thm@preskip\parskip
%             }%
%         }
% \endgroup
%% http://tex.stackexchange.com/a/43971/4233
% \makeatletter
% \def\th@plain{%
%   \thm@notefont{}% same as heading font
%   \itshape % body font
% }
% \def\th@definition{%
%   \thm@notefont{}% same as heading font
%   \normalfont % body font
% }
% \makeatother

\makeatletter
\def\thm@space@setup{%
  \thm@preskip=0.35cm
  \thm@postskip=\thm@preskip % or whatever, if you don't want them to be equal
}
  \makeatother


\usepackage{amsthm}
\makeatletter
\def\th@plain{%
  \thm@notefont{}% same as heading font
  \itshape % body font
}
\def\th@definition{%
  \thm@notefont{}% same as heading font
  \normalfont % body font
}
\makeatother


% http://tex.stackexchange.com/questions/57284/pdf-contains-incorrect-bookmark-hierarchy
\usepackage[unicode]{hyperref}
