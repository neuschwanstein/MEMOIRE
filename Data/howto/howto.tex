\documentclass{article}[10pt]

\usepackage{listings}

\title{The Guide on Using Market Side-Information}
\author{Thierry \textsc{Bazier-Matte}\\HEC Montreal}

\begin{document}
\maketitle

Let's follow through the algorithm. First off, we simply query off our database of all its
articles. We make sure to properly sort them: 
\begin{verbatim}
articles = list(Articles(['id','date']))
articles = Namedtuples(articles)
articles = sorted(articles,key=lambda a: a.date)
\end{verbatim}

We must be very careful in how we aggregate the articles to run our algorithm. First off,
we have two lists: the first one contains daily log returns at the New york stock
exchange, and the other one a list of articles published throughout the day. The
carefulness must be in that of causaulity which must not be present either in the test or
training sets. So how should we model the behaviour of the response of the market to the
financial information available throughout the day? If we did possess high frequency data,
then each new article would be in itself a sample, and the next observed logreturn would
be the corresponding response of the distribution. However, we are here dealing with
fundamentally time dependant problem. The solution of the problem should actually be
invariant to the time scale of the logreturns measures, and therefore should be able to
scale to weekly or even monthly measures.

A naive approach would simply to consider the global influence some news have on the
following day. Therefore, if the market opens at 9AM in the morning, at this precise time
we must be able to know exactly how much will be invested in the market, regardless of
future events. In short, at a given date 

What was actually done is slightly different. For each logreturn record, we know between
which dates it occured.

We then query Yahoo! Finance web API and return a special structure called \verb+Record+
consisting of the following fields:
\verb+odict_keys(['beg_date', 'end_date', 'logreturn'])+. We make sure to only query the
date for which we know we have text information. 
\begin{verbatim}
sp500 = get_sp500_records(beg_date=articles[0].date,end_date=articles[-1].date)
\end{verbatim}

\end{document}

%%% Local Variables:
%%% mode: latex
%%% TeX-master: t
%%% End:
