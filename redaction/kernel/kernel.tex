\documentclass{article}
\newcommand{\ts}{\textsuperscript}
\newcommand{\figref}[1]{Fig.~\ref{#1}}
\usepackage{amsmath}
\usepackage{amsthm}
\usepackage{graphicx}
\usepackage{geometry}
\usepackage{subcaption}
\usepackage{bm}
\usepackage{hyperref}
\usepackage[retainorgcmds]{IEEEtrantools}
\usepackage{mathtools}
\usepackage{color}
\usepackage{marginnote}
\usepackage[utf8]{inputenc}


\DeclareMathOperator*{\argmax}{arg\,max}
\DeclareMathOperator*{\argmin}{arg\,min}
\DeclareMathOperator{\st}{s.t.}
\DeclareMathOperator{\epi}{epi}
\DeclareMathOperator{\diag}{diag}
\DeclareMathOperator{\dom}{dom}
\DeclareMathOperator{\tr}{tr}
\DeclareMathOperator*{\minimize}{minimize}
\DeclarePairedDelimiter\floor{\lfloor}{\rfloor}

\newcommand{\iso}{\simeq}
\newcommand{\dd}{\partial}
\newcommand{\real}{\bm R}
\newcommand{\trueRisk}{R_{\mathrm{true}}}

\newcommand{\starsection}{\vspace{1em}\begin{center}$\star\quad\star\quad\star$\end{center}\vspace{1em}}


\newcommand{\hilight}[1]{\colorbox{yellow}{#1}}
\let\oldmarginnote\marginnote
\renewcommand{\marginnote}[1]{\oldmarginnote{\footnotesize\emph{#1}}[0cm]}

\theoremstyle{plain}
\newtheorem{prop}{Proposition}
\newtheorem{thm}{Theorem}

\theoremstyle{definition}
\newtheorem*{deff}{Definition}
\newtheorem*{rem}{Remark}


\geometry{letterpaper}
\IEEEeqnarraydefcolsep{0}{\leftmargini}

\newcommand{\lag}{\mathscr{L}}
\newcommand{\sumi}{\sum_{i=1}^n}
\newcommand{\sumij}{\sum_{i,j=1}^n}
\renewcommand{\dd}{\nabla}

\title{The Use of Kernels in the Portfolio Optimization Problem}
\author{Thierry Bazier-Matte}

\begin{document}
\maketitle

\section{Introduction}


We recall that the basic problem of chosing a linear investment decision
$q^\star\in\real^p$ based on $p$ features can be described using the following
formulation:
\begin{equation}
  \label{primal}
  q^\star = \argmax_q\{ n^{-1} \sum_{i=1}^n u(r_i\,q^Tx_i) - \lambda\|q\|^2_2\}.
\end{equation}

However, this solution has the disadvantage of being linear in the feature
space. Therefore, unless `good' and `bad' investments (as perceived by the investor) can
be neatly separated by halfspaces in the feature space, which is unlikely, the decision is
at risk of incurring poor performance.

Using kernel methods we can resolve this problem by augmenting the feature space using a
mapping $\phi:\real^p\to\mathscr{H}$ where $\mathscr{H}$ is expected to be of higher
dimension than $p$ (possibly of an infinite dimension). \todo{explain polynomial
  kernels as a concrete example}. In this new space, a new linear decision can be found
which, when mapped back to $\real^p$, is nonlinear. 

However, optimizing the problem \eqref{primal} in such a space can quickly become hard as
$\binom{p+2}{2}$ grows quite fast for a quadratic kernel. In what follows, we will see
that we can solve the problem in dual space where only the inner product of observations
$\langle \phi(x_i),\phi(x_j)\rangle$ in $\mathscr{H}$ will be needed.


\section{Dual formulation}
We start with the following problem:
\begin{equation*}
  \minimizeEquation{-\sum_{i=1}^nu(r_i\,q^Tx_i) + n\lambda\|q\|^2}.
\end{equation*}
Or if we introduce the slack variable $\xi$, we obtain
\begin{align*}
  \minimizeEquationSt{-\sum_{i=1}^nu(\xi_i)+n\lambda\|q\|^2}[\xi_i=r_i\,q^Tx_i].
\end{align*}
The lagrangian $\lag$ for this problem is then defined for
$\xi,q\in\real^p$ and $\alpha\in\real^n$ as:
\begin{equation}
  \label{lag}
  \lag = -\sumi u(\xi_i)+\sumi \alpha_i(\xi_i-r_i\,q^Tx_i)+n\lambda \|q\|^2.
\end{equation}
KKT conditions apply \todo{explain and understand why...}. We must first have
$\dd_q\lag=0$ and $\dd_{\xi_i}\lag=0$. These two conditions yield:
\begin{equation}
  \label{kkt1}
  q = \frac{1}{2n\lambda}\sumi \alpha_ir_ix_i
\end{equation}
and
\begin{equation}
  \label{kkt2}
  \alpha_i = u'(\xi_i).
\end{equation}
Now, denote $u^*$ the convex conjugate of $u$. Then, by standard properties of the
conjugate function [Boyd p.~95] and from \eqref{kkt2}, we obtain
\begin{equation}
  \label{ustar}
  u^*(\alpha_i) = \xi_iu'(\xi_i) - u(\xi_i) = \xi_i\alpha_i - u(\xi_i).
\end{equation}
Therefore, using \eqref{ustar} and \eqref{kkt1} to eliminate successively $\xi$ and $q$
from \eqref{lag}, we find
\begin{align}
  \lag &= \sumi(u^*(\alpha_i)-\xi_i\alpha_i) + \sumi \xi_i\alpha_i 
         - \sumi r_i\,q^Tx_i + n\lambda\|q\|^2 \nonumber\\
       &= \sumi u^*(\alpha_i)
         - \frac{1}{2n\lambda}\sumij \alpha_i\alpha_jr_ir_jx_i^Tx_j
         + \frac{1}{4n\lambda}\sumij \alpha_i\alpha_jr_ir_jx_i^Tx_j \nonumber\\
       &=  \sumi u^*(\alpha_i)
         - \frac{1}{4n\lambda}\sumij \alpha_i\alpha_jr_ir_jx_i^Tx_j\nonumber\\
       &= \sumi u^*(\alpha_i) - \mu(\alpha \circ r)^TK(\alpha \circ r),\label{gen}
\end{align}
with $\mu$ the new regularization parameter and $\circ$ the Hadamard (component-wise)
product and $K$ is the matrix of inner products in $\mathscr{H}$.

The new optimization problem becomes
\begin{align*}
  \maximizeEquation{\sumi u^*(\alpha_i) - \mu(\alpha\circ r)^TK(\alpha\circ r)}
\end{align*}
over $\alpha\in\real^n$.

Once a solution $\alpha^\star$ for the dual problem has been found, then the investment
decision based on a set of features $x_0$ can be expressed as follows:
\begin{align*}
  p &= q^Tx_0\\
    &= \sumi\alpha_ir_iK(x_i,x_0)\\
    &= (\alpha \circ r)^T\phi(x_0),
\end{align*}
where
\[
  \phi(x) =
  \begin{pmatrix}
    K(x_i,x)\\
    \vdots\\
    K(x_n,x)
  \end{pmatrix}.
\]


\subsection{Exponential utility}
Consider for instance $u(r) = e^{\beta r}$.
The expression \eqref{gen} for $\lag$ can be simplified if we consider an exponential
utility of the form $u(r) = -\exp(-\beta r)$. Then we have $w(r) = \exp(-\beta r)$ which
in turn gives $w'(r) = -\beta\exp(-\beta r)$, so that $v(r) =
-\log(-r/\beta)$. Therefore
\[
  w(v(-\alpha_i)) = \left(\frac{\alpha_i}{\beta}\right)^\beta
\]
and
\[
  \alpha_iv(-\alpha_i) = -\alpha_i\log(\alpha_i/\beta).
\]
In what follows we let $\beta=1$ for simplicity. $\lag$ becomes
\begin{align*}
  \lag &= \sumi\alpha_i(1-\log \alpha_i) - \frac{1}{4\mu}\sumij
         \alpha_i\alpha_jr_ir_jx_i^Tx_j.\\
       &= \alpha^T(1-\log\alpha) - \tilde\lambda(\alpha\circ r)^TK(\alpha \circ r)
\end{align*}
using a new regularization parameter $\tilde\lambda$ and $K$ the Gram matrix of inner
products.

The prescribed investment scalar based on feature $x_0$ is given by:
where $\phi(x)_i = K(x_i,x)$. \todo{introduce representer theorem?}.

\subsection{Risk neutral utility}

The expression of \eqref{gen} is further simplified if we instead consider a risk neutral
utility $u(r) = r$. \todo{Degenerate case? We have $\alpha=1$ from \eqref{kkt2}}.


% \section{Old}


% Using kernel theory, we can extend this algorithm to include non-linear decisions. For
% example, consider the sigmoid kernel $K_\sigma$:
% \[
%   K_\sigma(q,x) = \frac{\exp(q^T x)}{1+\exp(q^Tx)}.
% \]
% Other kernels include \eg polynomial kernel or gaussian distance kernel. In any case, let
% $K$ be a positive definite symmetric (PDS) kernel. Then the associated non linear learning
% problem therefore becomes:
% \[
%   q^\star = \argmin_q\{ n^{-1}\sum_{i=1}^n u(r_i\,K(q,x_i)) + \lambda\|q\|^2_K\},
% \]
% with $\|\cdot\|_K$ the associated kernel space norm. 

% The problem of this representation is that the investment decision $q$ is confined within
% the kernel space, and previous guarantees offered for scalar investment $q^Tx_i$ might not
% hold. Therefore, we are interested in an intermediate representation using a feature
% mapping $\phi$ of the following form:
% \[
%   \phi(x) =
%   \begin{pmatrix}
%     K(x_1,x)\\
%     \vdots\\
%     K(x_n,x)
%   \end{pmatrix}
% \]

% Then, provided that $K$ is normalized, the representer theorem states that
% \[
%   q^\star = \argmin_q\{n^{-1}\sum_{i=1}^n u(r_i\,q^T\phi(x_i)) + \lambda\|q\|^2_2\}.
% \]

\end{document}

%%% Local Variables:
%%% mode: latex
%%% TeX-master: t
%%% End:
