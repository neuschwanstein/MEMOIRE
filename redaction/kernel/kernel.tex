\documentclass{article}
\newcommand{\ts}{\textsuperscript}
\newcommand{\figref}[1]{Fig.~\ref{#1}}
\usepackage{amsmath}
\usepackage{amsthm}
\usepackage{graphicx}
\usepackage{geometry}
\usepackage{subcaption}
\usepackage{bm}
\usepackage{hyperref}
\usepackage[retainorgcmds]{IEEEtrantools}
\usepackage{mathtools}
\usepackage{color}
\usepackage{marginnote}
\usepackage[utf8]{inputenc}


\DeclareMathOperator*{\argmax}{arg\,max}
\DeclareMathOperator*{\argmin}{arg\,min}
\DeclareMathOperator{\st}{s.t.}
\DeclareMathOperator{\epi}{epi}
\DeclareMathOperator{\diag}{diag}
\DeclareMathOperator{\dom}{dom}
\DeclareMathOperator{\tr}{tr}
\DeclareMathOperator*{\minimize}{minimize}
\DeclarePairedDelimiter\floor{\lfloor}{\rfloor}

\newcommand{\iso}{\simeq}
\newcommand{\dd}{\partial}
\newcommand{\real}{\bm R}
\newcommand{\trueRisk}{R_{\mathrm{true}}}

\newcommand{\starsection}{\vspace{1em}\begin{center}$\star\quad\star\quad\star$\end{center}\vspace{1em}}


\newcommand{\hilight}[1]{\colorbox{yellow}{#1}}
\let\oldmarginnote\marginnote
\renewcommand{\marginnote}[1]{\oldmarginnote{\footnotesize\emph{#1}}[0cm]}

\theoremstyle{plain}
\newtheorem{prop}{Proposition}
\newtheorem{thm}{Theorem}

\theoremstyle{definition}
\newtheorem*{deff}{Definition}
\newtheorem*{rem}{Remark}


\geometry{letterpaper}
\IEEEeqnarraydefcolsep{0}{\leftmargini}


\title{The Use of Kernels in the Portfolio Optimization Problem}
\author{Thierry Bazier-Matte}

\begin{document}
\maketitle

We recall that the basic problem on chosing a linear investment $q^\star\in\real^p$ based on $p$
features can be described using the following formulation:
\[
  q^\star = \argmin_q\{ n^{-1} \sum_{i=1}^n u(r_i\,q^Tx_i) + \lambda\|q\|^2_2\}.
\]

Using kernel theory, we can extend this algorithm to include non-linear decisions. For
example, consider the sigmoid kernel $K_\sigma$:
\[
  K_\sigma(q,x) = \frac{\exp(q^T x)}{1+\exp(q^Tx)}.
\]
Other kernels include \eg polynomial kernel or gaussian distance kernel. In any case, let
$K$ be a positive definite symmetric (PDS) kernel. Then the associated non linear learning
problem therefore becomes:
\[
  q^\star = \argmin_q\{ n^{-1}\sum_{i=1}^n u(r_i\,K(q,x_i)) + \lambda\|q\|^2_K\},
\]
with $\|\cdot\|_K$ the associated kernel space norm. 

The problem of this representation is that the investment decision $q$ is confined within
the kernel space, and previous guarantees offered for scalar investment $q^Tx_i$ might not
hold. Therefore, we are interested in an intermediate representation using a feature
mapping $\phi$ of the following form:
\[
  \phi(x) =
  \begin{pmatrix}
    K(x_1,x)\\
    \vdots\\
    K(x_n,x)
  \end{pmatrix}
\]

Then, provided that $K$ is normalized, the representer theorem states that
\[
  q^\star = \argmin_q\{n^{-1}\sum_{i=1}^n u(r_i\,q^T\phi(x_i)) + \lambda\|q\|^2_2\}.
\]

\end{document}

%%% Local Variables:
%%% mode: latex
%%% TeX-master: t
%%% End:
