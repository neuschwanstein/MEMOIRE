\documentclass{article}
\newcommand{\ts}{\textsuperscript}
\newcommand{\figref}[1]{Fig.~\ref{#1}}
\usepackage{amsmath}
\usepackage{amsthm}
\usepackage{graphicx}
\usepackage{geometry}
\usepackage{subcaption}
\usepackage{bm}
\usepackage{hyperref}
\usepackage[retainorgcmds]{IEEEtrantools}
\usepackage{mathtools}
\usepackage{color}
\usepackage{marginnote}
\usepackage[utf8]{inputenc}


\DeclareMathOperator*{\argmax}{arg\,max}
\DeclareMathOperator*{\argmin}{arg\,min}
\DeclareMathOperator{\st}{s.t.}
\DeclareMathOperator{\epi}{epi}
\DeclareMathOperator{\diag}{diag}
\DeclareMathOperator{\dom}{dom}
\DeclareMathOperator{\tr}{tr}
\DeclareMathOperator*{\minimize}{minimize}
\DeclarePairedDelimiter\floor{\lfloor}{\rfloor}

\newcommand{\iso}{\simeq}
\newcommand{\dd}{\partial}
\newcommand{\real}{\bm R}
\newcommand{\trueRisk}{R_{\mathrm{true}}}

\newcommand{\starsection}{\vspace{1em}\begin{center}$\star\quad\star\quad\star$\end{center}\vspace{1em}}


\newcommand{\hilight}[1]{\colorbox{yellow}{#1}}
\let\oldmarginnote\marginnote
\renewcommand{\marginnote}[1]{\oldmarginnote{\footnotesize\emph{#1}}[0cm]}

\theoremstyle{plain}
\newtheorem{prop}{Proposition}
\newtheorem{thm}{Theorem}

\theoremstyle{definition}
\newtheorem*{deff}{Definition}
\newtheorem*{rem}{Remark}


\geometry{letterpaper}
\IEEEeqnarraydefcolsep{0}{\leftmargini}

\newcommand{\lag}{\mathscr{L}}
\newcommand{\sumi}{\sum_{i=1}^n}
\newcommand{\sumij}{\sum_{i,j=1}^n}
\renewcommand{\dd}{\nabla}
\allowdisplaybreaks

\title{The Use of Kernels in the Portfolio Optimization Problem}
\author{Thierry Bazier-Matte}

\begin{document}
\maketitle


\section{Original Problem}


Let us consider the following problem, optimized over $q\in\real^p$:
\begin{equation}
  \label{primal}
  \minimizeEquation{\sumi \ell(r_i\,q^Tx_i) + n\lambda\|q\|^2},
\end{equation}
where $\ell=-u$. Alternatively, this problem can be respecified using slack vector
$\xi\in\real^n$ as
\begin{align}
  \minimizeEquationSt{\label{primal2}\sumi \ell(\xi_i) + n\lambda\|q\|^2}[\label{affineconstraints}\xi_i = r_i\,q^Tx_i.] 
\end{align}
Let $\alpha\in\real^n$. The Lagrangian of \eqref{primal2} can be written as
\begin{equation}
  \lag(q,\xi,\alpha) = \sumi \ell(\xi_i) + n\lambda\|q\|^2 + \sumi \alpha_i(r_iq^Tx_i - \xi_i).
\end{equation}
Because the objective \eqref{primal2} is convex and the constraints
\eqref{affineconstraints} are affine in $q$ and $\xi$, Slater's theorem states that the
duality gap of the problem is zero. In other words, solving \eqref{primal} is equivalent
to maximizing the Lagrange dual function $g$ over $\alpha$:
\begin{equation}
  \maximizeEquation{g(\alpha) = \inf_{q,\xi} \lag(q,\xi,\alpha).}
\end{equation}
Now, note that
\begin{align}
  g(\alpha)
  &= \inf_{q,\xi} \left\{\sumi \ell(\xi_i) + n\lambda\|q\|^2 + \sumi \alpha_i(r_iq^Tx_i -
    \xi_i)\right\}\\
  &= \inf_\xi\left\{\sumi \ell(\xi_i) - \alpha^T\xi\right\} + \inf_q\left\{\sumi \alpha_ir_iq^Tx_i +
    n\lambda\|q\|^2\right\}\\
  & = -\sup_\xi\left\{a^T\xi - \sumi \ell(\xi_i)\right\} + \inf_q\left\{\sumi \alpha_ir_iq^Tx_i +
    n\lambda\|q\|^2\right\}\\
  & = -\sumi \ell^*(\alpha_i) + \inf_q\left\{\sumi \alpha_ir_iq^Tx_i +n\lambda\|q\|^2\right\}.\label{deriv1} 
\end{align}
Where $\ell^*$ is the convex conjugate of the loss function and is defined by
\begin{equation}
  \ell(\alpha_i) = \sup_{\xi_i}\left\{\alpha_i\xi_i - \ell(\xi_i)\right\}.
\end{equation}
Note that the identity
\begin{equation}
  \label{lemma1}
  f(\xi_1,\dots,\xi_n) = \sumi \ell(\xi_i) \Longrightarrow f^*(\xi_1,\dots,\xi_n) = \sumi \ell^*(\xi_i)
\end{equation}
was used. Consider now the second part of \eqref{deriv1}. Since
the expression is differentiable, we can analytically solve for $q$:
\begin{equation}
  \nabla_q \left\{\sumi \alpha_ir_iq^Tx_i + n\lambda\|q\|^2\right\} = 0
\end{equation}
implies that
\begin{equation}
 q = -\frac{1}{2n\lambda}\sumi \alpha_ir_ix_i \label{qdef}
\end{equation}
at the infimum.

Using \eqref{qdef}, we can eliminate $q$ from \eqref{deriv1}, so that
\begin{align}
  g(\alpha) &= -\sumi \ell^*(\alpha_i) - \frac{1}{2n\lambda}\sumij \alpha_i\alpha_jr_ir_jx_i^Tx_j + \frac{1}{4n\lambda}
         \sumij \alpha_i\alpha_jr_ir_jx_i^Tx_j\\
       &= -\sumi \ell^*(\alpha_i) - \frac{1}{4n\lambda}(\alpha \circ r)^TK(\alpha \circ r). 
\end{align}
Therefore, in its dual form, the problem \eqref{primal} is equivalent to solving
\begin{equation}
  \minimizeEquation{\sumi \ell^*(\alpha_i) + \frac{1}{4n\lambda}(\alpha \circ r)^TK(\alpha \circ r).}
\end{equation}

\section{Alternate problem}

We now consider a new problem, slightly different from \eqref{primal} where a
regularization based on the sum of the square of the investment sizes $q^Tx_i$ is applied:
\begin{equation}
  \label{primal2000}
  \minimizeEquation{\sumi \ell(r_i\,q^Tx_i) + \gamma\sumi(q^Tx_i)^2 + n\lambda\|q\|^2}.
\end{equation}
Again, this problem can be respecified using slack vector $\xi \in \real^n$ as
\begin{align}
  \minimizeEquationSt{\label{primal2001}\sumi \ell(\xi_i) + \gamma\sumi(\xi_i/r_i)^2 + n\lambda\|q\|^2}[\label{constraints2}\xi_i = r_i\,q^Tx_i].
\end{align}
The constraints \eqref{constraints2} are again affine, so that Slater's theorem apply.

The lagrangian of \eqref{primal2001} is
\begin{equation}
  \lag(q,\xi,\alpha) = \sumi \ell(\xi_i) + \gamma\sumi(\xi_i/r_i)^2 + n\lambda\|q\|^2 + \sumi \alpha_i(r_iq^Tx_i - \xi_i),
\end{equation}
and we seek its infimum over $(q,\xi)$.
\begin{align}
  &\inf_{q,\xi} \left\{\sumi \ell(\xi_i) + \gamma\sumi (\xi_i/r_i)^2 + n\lambda\|q\|^2 + \sumi \alpha_i(r_iq^Tx_i -
    \xi_i)\right\}\\
  &\quad =\inf_\xi\left\{\sumi \ell(\xi_i) + \gamma\sumi(\xi_i/r_i)^2 - \alpha^T\xi\right\} + \inf_q\left\{\sumi
    \alpha_ir_iq^Tx_i - n\lambda\|q\|^2\right\}\\
  &\quad = - \sup_\xi\left\{a^T\xi - \left(\sumi \ell(\xi_i) + \gamma\sumi(\xi_i/r_i)^2\right)\right\}
    - \frac{1}{4n\lambda} (\alpha \circ r)^TK(\alpha \circ r)\label{this}.
\end{align}

Let $f_i(\xi_i) := h_1(\xi_i)+ h_2(\xi_i) = \ell(\xi_i) + \gamma(\xi_i/r_i)^2$. Then, using \eqref{lemma1},
the first expression of \eqref{this} can be restated as
\begin{equation}
  \label{firstconj}
  -\sup_\xi\left\{\alpha^T \xi - \sumi f_i(\xi_i)\right\} = -\sumi f_i^*(\alpha_i).
\end{equation}
Let us introduce another identity:
\begin{equation}
  \label{lemma2}
  (h_1+h_2)^*(\alpha_i) = \inf_{\alpha_i'+\alpha_i''=\alpha_i} \{h_1^*(\alpha_i') + h_2^*(\alpha_i'')\}.
\end{equation}
Using \eqref{lemma2}, \eqref{firstconj} can be written as
\begin{align}
  -\sumi f_i^*(\alpha_i) &= -\sumi (h_1+h_2)^*(\xi_i)\\
                    &= -\sumi \inf_{\alpha_i'+\alpha_i''=\alpha_i} \{h_1^*(\alpha_i') + h_2^*(\alpha_i'')\}.
\end{align}
The first conjugate function $h_1^*$ is simply $\ell^*$. The second conjugate function can be
derived analytically:
\begin{align}
  h_2^*(\alpha_i'') &= \sup_{\xi_i}\,\{\alpha_i''\xi_i - h_2(\xi_i)\}\\
               &= \sup_{\xi_i}\,\{\alpha_i''\xi_i - \gamma(\xi_i/r_i)^2\}\label{lastsup}.
\end{align}
The supremum occurs when
\begin{equation}
  \label{supconj}
  \xi_i = \frac{r_i^2}{2\gamma}\alpha_i''.
\end{equation}
Therefore, \eqref{lastsup} simplifies to
\begin{equation}
  h_2^*(\alpha_i'') = \frac{r_i^2}{4\gamma}(\alpha_i'')^2.
\end{equation}

Putting it all back together, the dual of \eqref{primal2000} is
\begin{equation}
  -\sumi \inf_{\alpha_i'+\alpha_i''=\alpha_i}\,\left\{\ell^*(\alpha_i') + \frac{r_i^2}{4\gamma}(\alpha_i'')^2\right\}  - \frac{1}{4n\lambda} (\alpha \circ r)^TK(\alpha \circ r),
\end{equation}
which is equivalent to
\begin{equation}
  -\sumi \ell^*(\alpha_i) - \frac{1}{4\gamma}\sumi (r_i\,\beta_i)^2 - \frac{1}{4n\lambda}(r \circ (\alpha+\beta))^TK(r \circ (\alpha+\beta)),
\end{equation}
with new optimization variables $\alpha=\alpha',\beta=\alpha''\in\real^n$. The dual optimization problem is
therefore
\begin{equation}
  \minimizeEquation{\sumi \ell^*(\alpha_i) + \frac{1}{4\gamma}\|r\circ\beta\|^2 + \frac{1}{4n\lambda}(r \circ (\alpha+\beta))^TK(r \circ (\alpha+\beta))}.
\end{equation}
% The second expression is the same as in \eqref{deriv1}





% \section{Old}


% We recall that the basic problem of chosing a linear investment decision
% $q^\star\in\real^p$ based on $p$ features can be described using the following
% formulation:
% \begin{equation}
%   q^\star = \argmax_q\{ n^{-1} \sum_{i=1}^n u(r_i\,q^Tx_i) - \lambda\|q\|^2_2\}.
% \end{equation}

% However, this solution has the disadvantage of being linear in the feature
% space. Therefore, unless `good' and `bad' investments (as perceived by the investor) can
% be neatly separated by halfspaces in the feature space, which is unlikely, the decision is
% at risk of incurring poor performance.

% Using kernel methods we can resolve this problem by augmenting the feature space using a
% mapping $\phi:\real^p\to\mathscr{H}$ where $\mathscr{H}$ is expected to be of higher
% dimension than $p$ (possibly of an infinite dimension). \todo{explain polynomial
%   kernels as a concrete example}. In this new space, a new linear decision can be found
% which, when mapped back to $\real^p$, is nonlinear. 

% However, optimizing the problem \eqref{primal} in such a space can quickly become hard as
% $\binom{p+2}{2}$ grows quite fast for a quadratic kernel. In what follows, we will see
% that we can solve the problem in dual space where only the inner product of observations
% $\langle \phi(x_i),\phi(x_j)\rangle$ in $\mathscr{H}$ will be needed.


% \section{Dual formulation}
% We start with the following problem:
% \begin{equation*}
%   \minimizeEquation{-\sum_{i=1}^nu(r_i\,q^Tx_i) + n\lambda\|q\|^2}.
% \end{equation*}
% Or if we introduce the slack variable $\xi$, we obtain
% \begin{align*}
%   \minimizeEquationSt{-\sum_{i=1}^nu(\xi_i)+n\lambda\|q\|^2}[\xi_i=r_i\,q^Tx_i].
% \end{align*}
% The lagrangian $\lag$ for this problem is then defined for
% $\xi,q\in\real^p$ and $\alpha\in\real^n$ as:
% \begin{equation}
%   \label{lag}
%   \lag = -\sumi u(\xi_i)+\sumi \alpha_i(\xi_i-r_i\,q^Tx_i)+n\lambda \|q\|^2.
% \end{equation}
% KKT conditions apply \todo{explain and understand why...}. We must first have
% $\dd_q\lag=0$ and $\dd_{\xi_i}\lag=0$. These two conditions yield:
% \begin{equation}
%   \label{kkt1}
%   q = \frac{1}{2n\lambda}\sumi \alpha_ir_ix_i
% \end{equation}
% and
% \begin{equation}
%   \label{kkt2}
%   \alpha_i = u'(\xi_i).
% \end{equation}
% Now, denote $u^*$ the convex conjugate of $u$. Then, by standard properties of the
% conjugate function [Boyd p.~95] and from \eqref{kkt2}, we obtain
% \begin{equation}
%   \label{ustar}
%   u^*(\alpha_i) = \xi_iu'(\xi_i) - u(\xi_i) = \xi_i\alpha_i - u(\xi_i).
% \end{equation}
% Therefore, using \eqref{ustar} and \eqref{kkt1} to eliminate successively $\xi$ and $q$
% from \eqref{lag}, we find
% \begin{align}
%   \lag &= \sumi(u^*(\alpha_i)-\xi_i\alpha_i) + \sumi \xi_i\alpha_i 
%          - \sumi r_i\,q^Tx_i + n\lambda\|q\|^2 \nonumber\\
%        &= \sumi u^*(\alpha_i)
%          - \frac{1}{2n\lambda}\sumij \alpha_i\alpha_jr_ir_jx_i^Tx_j
%          + \frac{1}{4n\lambda}\sumij \alpha_i\alpha_jr_ir_jx_i^Tx_j \nonumber\\
%        &=  \sumi u^*(\alpha_i)
%          - \frac{1}{4n\lambda}\sumij \alpha_i\alpha_jr_ir_jx_i^Tx_j\nonumber\\
%        &= \sumi u^*(\alpha_i) - \mu(\alpha \circ r)^TK(\alpha \circ r),\label{gen}
% \end{align}
% with $\mu$ the new regularization parameter and $\circ$ the Hadamard (component-wise)
% product and $K$ is the matrix of inner products in $\mathscr{H}$.

% The new optimization problem becomes
% \begin{align*}
%   \maximizeEquation{\sumi u^*(\alpha_i) - \mu(\alpha\circ r)^TK(\alpha\circ r)}
% \end{align*}
% over $\alpha\in\real^n$.

% Once a solution $\alpha^\star$ for the dual problem has been found, then the investment
% decision based on a set of features $x_0$ can be expressed as follows:
% \begin{align*}
%   p &= q^Tx_0\\
%     &= \sumi\alpha_ir_iK(x_i,x_0)\\
%     &= (\alpha \circ r)^T\phi(x_0),
% \end{align*}
% where
% \[
%   \phi(x) =
%   \begin{pmatrix}
%     K(x_i,x)\\
%     \vdots\\
%     K(x_n,x)
%   \end{pmatrix}.
% \]


% \subsection{Exponential utility}
% Consider for instance $u(r) = e^{\beta r}$.
% The expression \eqref{gen} for $\lag$ can be simplified if we consider an exponential
% utility of the form $u(r) = -\exp(-\beta r)$. Then we have $w(r) = \exp(-\beta r)$ which
% in turn gives $w'(r) = -\beta\exp(-\beta r)$, so that $v(r) =
% -\log(-r/\beta)$. Therefore
% \[
%   w(v(-\alpha_i)) = \left(\frac{\alpha_i}{\beta}\right)^\beta
% \]
% and
% \[
%   \alpha_iv(-\alpha_i) = -\alpha_i\log(\alpha_i/\beta).
% \]
% In what follows we let $\beta=1$ for simplicity. $\lag$ becomes
% \begin{align*}
%   \lag &= \sumi\alpha_i(1-\log \alpha_i) - \frac{1}{4\mu}\sumij
%          \alpha_i\alpha_jr_ir_jx_i^Tx_j.\\
%        &= \alpha^T(1-\log\alpha) - \tilde\lambda(\alpha\circ r)^TK(\alpha \circ r)
% \end{align*}
% using a new regularization parameter $\tilde\lambda$ and $K$ the Gram matrix of inner
% products.

% The prescribed investment scalar based on feature $x_0$ is given by:
% where $\phi(x)_i = K(x_i,x)$. \todo{introduce representer theorem?}.

% \subsection{Risk neutral utility}

% The expression of \eqref{gen} is further simplified if we instead consider a risk neutral
% utility $u(r) = r$. \todo{Degenerate case? We have $\alpha=1$ from \eqref{kkt2}}.


% % \section{Old}


% % Using kernel theory, we can extend this algorithm to include non-linear decisions. For
% % example, consider the sigmoid kernel $K_\sigma$:
% % \[
% %   K_\sigma(q,x) = \frac{\exp(q^T x)}{1+\exp(q^Tx)}.
% % \]
% % Other kernels include \eg polynomial kernel or gaussian distance kernel. In any case, let
% % $K$ be a positive definite symmetric (PDS) kernel. Then the associated non linear learning
% % problem therefore becomes:
% % \[
% %   q^\star = \argmin_q\{ n^{-1}\sum_{i=1}^n u(r_i\,K(q,x_i)) + \lambda\|q\|^2_K\},
% % \]
% % with $\|\cdot\|_K$ the associated kernel space norm. 

% % The problem of this representation is that the investment decision $q$ is confined within
% % the kernel space, and previous guarantees offered for scalar investment $q^Tx_i$ might not
% % hold. Therefore, we are interested in an intermediate representation using a feature
% % mapping $\phi$ of the following form:
% % \[
% %   \phi(x) =
% %   \begin{pmatrix}
% %     K(x_1,x)\\
% %     \vdots\\
% %     K(x_n,x)
% %   \end{pmatrix}
% % \]

% % Then, provided that $K$ is normalized, the representer theorem states that
% % \[
% %   q^\star = \argmin_q\{n^{-1}\sum_{i=1}^n u(r_i\,q^T\phi(x_i)) + \lambda\|q\|^2_2\}.
% % \]

\end{document}

%%% Local Variables:
%%% mode: latex
%%% TeX-master: t
%%% End:
