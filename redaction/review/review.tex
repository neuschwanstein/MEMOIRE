\section{Optimisation moderne de portefeuille}


Dans ce document, nous allons tenter de classifier et de répertorier la plupart des
méthodes ayant rapport, de près ou de loin, à l'intersection des méthodes statistiques
avancées et de l'apprentissage machine avec la théorie du portefeuille, en présentant pour
chacune d'elle leurs avantages et leurs inconvénients.

\subsection{Théorie classique du portefeuille}

Une revue de littérature sur la théorie du portefeuille serait fondamentalement incomplète
sans l'article fondateur de Markowitz, publié en 1952 \cite{markowitz1952portfolio}.

Nous allons montrer que le cadre théorique développé par Markowitz peut être considéré
comme un cas particulier de notre algorithme, pour autant que l'on considère un portefeuille
à un seul actif.

Soit $w \in \Re^k$ le vecteur représentant la répartition du portefeuille de Markowitz à
$k$ actifs à optimiser. Alors un investisseur \textit{markowitzien} souhaite résoudre le
problème suivant:
\begin{equation}
  \minimizeEquationSt{w^T\Sigma w}[\mu^Tw = \mu_0,]
\end{equation}
où $\Sigma \in \Re^{k \times k}$ est la covariance du rendement des actifs et $\mu \in \Re^k$ le vecteur
d'espérance.  \todo{Montrer formellement.} Par la théorie de l'optimisation convexe, il
existe une constante $\gamma\in\Re$ telle que le problème énoncé est équivalent à
\begin{equation}
  \maximizeEquation{\mu^Tw + \gamma\,w^T\Sigma w.}
\end{equation}
Dans le cas où on considère un portefeuille à un seul actif, alors ce problème se réduit
alors à
\begin{equation}
  \label{markov}
  \maximizeEquation{\mu q - \gamma\sigma^2 q^2,}
\end{equation}
où on a posé $\mu \coloneqq \E R$ et $\sigma^2 \coloneqq \Var R$.

Supposons qu'un investisseur soit doté d'une utilité quadratique paramétrée par
\begin{equation}
  \label{quadu}
  u(r) = r - \frac{\gamma}{\sigma^2+\mu^2}\sigma^2r^2,
\end{equation}
et que l'information factorielle intégrée à l'algorithme ne consiste uniquement qu'en les
rendements eux mêmes; autrement dit, le vecteur d'information $X$ se réduirait tout
simplement à un terme constant fixé à 1, \ie, $X\sim 1$. \todo{expliquer}. 

Avec une utilité \eqref{quadu} et l'absence d'information supplémentaire, l'objectif de
\eqref{probva} devient aussitôt
\begin{equation}
  \label{euquad}
  \EU(qR) = q \E R - \frac{\gamma}{\sigma^2+\mu^2}\sigma^2 q^2\E R^2.
\end{equation}
Or, puisque $\Var R = \E  R^2 - (\E R)^2$, on a que $\E R^2 = \sigma^2 + \mu^2$, ce qui entraîne
donc que \eqref{quadu} s'exprime par
\begin{equation}
  \maximizeEquation{\EU(qR) = \mu q - \gamma\sigma^2 q^2,}
\end{equation}
ce qui est tout à fait identique à \eqref{markov}.

Nous suggérons au lecteur intéressé par l'équivalence des diverses formulations
d'optimi\-sation de portefeuille dans un univers de Markowitz \cite{bodnar2013equivalence}
et \cite{markowitz2014mean}, tous deux publiés à l'occasion du soixantième anniversaire de
\cite{markowitz1952portfolio}.

\subsection{Portefeuille universel / Papiers d'Elad Hazan}

Ce mémoire sera également consacré aux garanties statistiques de performance des
estimateurs $q^\star$.

Bien que le modèle soit différent et de nature itérative, le \textit{portefeuille
  universel} de \cite{cover1991universal} est à notre connaissance un des premiers modèles
de gestion de portefeuille à exploiter une distribution arbitraire tout en proposant des
garanties statistiques de convergence.

Voir \cite{cover1991universal,hazan2015online}.


\subsection{Théorie de portefeuille régularisé}

\cite{ban2016machine}

\subsection{Fama and French et suivants?}

\cite{fama1993common}


\subsection{Articles du NIPS}

\subsection{Papiers de Ben Van Roy}

\subsection{Conclusions: Notre problème par rapport à ces deux disciplines}




%%% Local Variables:
%%% mode: latex
%%% TeX-master: "main_review"
%%% End:
