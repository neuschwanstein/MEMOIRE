\section{Optimisation moderne de portefeuille}
\label{sec:review}

\epigraph{Je connais quelques hommes qui lisent avec le profit maximum, cent pages de
  mathématiques, de philosophie, d’histoire ou d’archéologie en vingt minutes.}{Louis
  Pauwels et Jacques Bergier\\\textsc{Le matin des magiciens}}

L'objet de cette section est de présenter une brève introduction à quelques textes
fondamentaux à l'intersection des statistiques et de la gestion financière de
portefeuille.


\subsection{Approche statistique}

Une revue de littérature sur la théorie du portefeuille serait fondamentalement incomplète
sans l'article fondateur de Markowitz, publié en 1952 \cite{markowitz1952portfolio}. Le
cadre théorique développé par Markowitz peut être considéré comme un cas particulier de
notre algorithme, pour autant que l'on considère un portefeuille à un seul actif.

Soit un portefeuille à $k$ actifs d'espérance de rendement $\mu\in\Re^k$ et de covariance
$\Sigma \in \Re^{k\times k}$. Si $q \in \Re^k$ représente la répartition du portefeuille à optimiser, alors
un investisseur \textit{markowitzien} souhaite
\begin{equation}
  \label{rev:marko}
  \maximizeEquation[q]{\mu^Tq - \gamma\,q^T\Sigma q.}
\end{equation}
Littéralement, il cherche à maximiser le rendement espéré du portefeuille pondéré (premier
terme) tout en minimisant, pour un certain niveau de risque $\gamma>0$, sa covariance
totale. En ne considérant qu'un seul actif dont la variance de rendement est notée $\sigma^2$,
le problème devient alors
\begin{equation}
  \maximizeEquation[q]{\mu^Tq - \gamma\,\sigma^2 q^2.}
\end{equation}
En recalibrant le terme de régularisation pour tenir compte de la variance $\sigma^2$, ce
problème devient un cas particulier du problème exposé dans ce mémoire (utilité risque
neutre, décision linéaire et une seule variable de marché constante à 1).

Mais le problème de Markowitz à un actif peut aussi être considéré comme une maximisation
d'utilité sans régularisation. En effet, en définissant
\begin{equation}
  u(r) = r - \frac{\gamma}{\sigma^2+\mu^2}\sigma^2r^2,
\end{equation}
on obtient
\begin{equation}
  \EU(q\,R) = q \E R - \frac{\gamma}{\sigma^2+\mu^2}\sigma^2 q^2\E R^2,
\end{equation}
et donc, puisque $\sigma^2 + \mu^2 = \E R^2$, le problème de Markowitz s'exprime aussi comme un
problème d'utilité espérée:
\begin{equation}
  \maximizeEquation[q]{\EU(qR).}
\end{equation}

Par contre, les garanties sur le rendement équivalent hors échantillon, qu'on développera
à la Section \ref{sec:bound}, ne s'appliquent qu'à des fonctions d'utilité monotones. Or,
comme l'utilité de Markowitz est quadratique, elle ne peut donc pas bénificer des mêmes
garanties, quand bien même on la régulariserait avec un terme $\lambda\|q\|^2$.

Nous suggérons au lecteur intéressé par l'équivalence des diverses formulations
d'optimisation de portefeuille dans un univers de Markowitz \cite{bodnar2013equivalence}
et \cite{markowitz2014mean}, tous deux publiés à l'occasion du soixantième anniversaire de
\cite{markowitz1952portfolio}.


L'article \cite{brandt2009parametric} se rapproche d'une des contributions de ce mémoire
puisqu'il considère l'optimisation d'un portefeuille à $k$ actifs disposant de $p$
variables de marché. Avec $X_t \in \Re^{k \times p}$ une matrice aléatoire représentant la
réalisation au temps $t$ des diverses variables de marché, la composition $w_t \in \Re^k$ du
portefeuille sera donnée par
\begin{equation}
  w_t = \bar w_t + X_t q.
\end{equation}
La décision $q \in \Re^p$ agit donc linéairement sur les variables de marché et globalement
sur tous les actifs. Le terme de biais $\bar w_t$ représente dans l'article original une
composition de référence, par exemple un index lorsqu'il est question de gestion active de
fonds. L'objectif de l'investisseur sera alors de
\begin{equation}
  \maximizeEquation{\E_t u(w_t^Tr_t),}
\end{equation}
autrement dit de choisir une décision $q$ permettant de
\begin{equation}
  \maximizeEquation[q \in \Re^p]{\E_t u(\bar w_t^Tr_t + r_t^TX_tq).}
\end{equation}
On obtient donc un objectif très proche de ce qui est proposé dans ce mémoire. Cependant,
l'absence de régularisation dans le vecteur de décision ne permet pas de fournir à
l'investisseur des garanties sur la performance du portefeuille. Pire, sous certaines
formes d'utilité (par exemple risque neutre), il est évident que la solution de l'objectif
peut avoir une amplitude non bornée!

Néanmoins, une telle approche demeure simple à implémenter, laisse une grande liberté dans
la forme paramétrique de l'utilité et évite d'avoir à calculer les deux premiers moments
statistiques d'un univers à $k$ actifs.

\subsection{Approche régularisée}

D'une certaine façon, \cite{markowitz1952portfolio} et \cite{brandt2009parametric}
approchent le problème de gestion de portefeuille un peu trop brusquement en ne laissant
pas suffisamment de place aux garanties statistiques. Autrement dit, ils présentent tous
un risque de généralisation élevé. La question est particulièrement bien documentée dans
le cas du portefeuille de Markowitz, voir par exemple
\cite{michaud1989markowitz}. Cependant, à l'instar de la méthode proposée par ce mémoire,
d'autres travaux ont cherché à étudier l'importance de la régularisation des décisions
dans la gestion de portefeuille.

Par exemple \cite{olivares2015robust} étudie l'idée selon laquelle les coûts de
transaction inhérents à la gestion de portefeuille peuvent être modélisés comme une
régularisation dans l'objectif de maximisation de rendement -- minimisation de variance. Le
problème de Markowitz \eqref{rev:marko} devient alors
\begin{equation}
  \maximizeEquation[q]{\mu^Tq - \gamma\,q^T\Sigma q - \kappa\|\Lambda(q-q_0)\|_p^p,}
\end{equation}
le troisième terme représentant les coûts de transaction comme la $p$-norme du
rebalancement du portefeuille linéarisé par un opérateur symétrique
$\Lambda \in \Re^{k \times k}$ et paramétré par un scalaire $\kappa$. Bien que la régularisation soit ici
appliquée au vecteur de poids et non à une décision linéaire sur des observations, les
auteurs parviennent empiriquement à la même conclusion, c'est-à-dire que les résultats
hors échantillons sont mieux contrôlés. Intuitivement, un tel résultat s'explique par le
fait qu'un gestionnaire de portefeuille soumis à des contraintes de coûts de transaction
évitera une politique d'investissement trop ambitieuse.

Cependant, l'argument demeure empirique et ne bénéficie donc pas comme ici de véritables
garanties théoriques sur les performances hors échantillon. En outre, le modèle suggéré
reste essentiellement markowitzien et se limite donc à maximiser une utilité quadratique,
alors que notre modèle permet d'optimiser sur une fonction d'utilité de forme arbitraire.

Plus récemment, \cite{ban2016machine} explore l'importance de la régularisation dans une
gestion de portefeuille où le risque est représenté par une fonction
$\mathcal{R}:\Re^k \to \Re$ agissant sur le vecteur de poids $q \in \Re^k$. Comment définir ce
risque est laissé à la discrétion de l'investisseur, mais pourrait être représenté par
exemple par $\mathcal{R}(q) = q^T\Sigma q$ la variance totale du portefeuille ou par sa
valeur à risque conditionnelle $\mathcal{R}(q) = \mathrm{CVar}(q)$.\footnote{La valeur à
  risque conditionnelle paramétrée par $\beta\in(0,1)$ d'une variable aléatoire $R$ (en
  l'occurrence le rendement pondéré par $q$) est une mesure de l'étalement dans les
  régions défavorables à l'investisseur et est donné par
  $\E (R \mid R \leq \text{$\beta$\ieme quantile de $R$})$.} L'argument offert est essentiellement
le même que celui présenté à la section \ref{sec:intro}: si on sait que le problème
empirique
\begin{equation}
  \maximizeEquation[q]{\hat \mu^Tq - \lambda_0\widehat{\mathcal{R}}(q)}
\end{equation}
converge asymptotiquement vers la solution optimale régularisée, l'absence de
régularisation implique l'absence de garantie sur la qualité des résultats lorsque
l'optimisation a lieu sur un nombre de points finis. La régularisation est alors exprimée
comme la \textit{variance empirique du risque}, \ie\ l'objectif devient
\begin{equation}
  \maximizeEquation[q]{\hat \mu^Tq - \lambda_0\widehat{\mathcal{R}}(q) - \lambda_1\Var(\widehat{\mathcal{R}}(q)).}
\end{equation}
Dans ces conditions, les auteurs démontrent que le risque encouru par la décision
empirique régularisée $\qh$ converge elle aussi vers le risque optimal. Cependant, aucune
borne finie n'est donnée, contrairement à ce qui est proposé dans ce mémoire.

Plus près de l'objectif de ce mémoire, \cite{rudin2014big} a le mérite d'appliquer
plusieurs théorèmes issus de l'apprentissage statistique dans un contexte de
gestion. L'idée est en fait de coupler le problème classique du vendeur de
journaux\footnote{Le problème du vendeur de journaux (ou \textsl{newsvendor problem})
  cherche à minimiser les coûts de gestion d'inventaire
  $c(p,D) = \alpha_1(D-p)^+ + \alpha_2(p-D)^+$ où $p$ est la quantité de journaux à commander,
  $D \in \Re$ la demande aléatoire et $\alpha_1,\alpha_2 \in \Re_+$ sont respectivement les coûts liés à la
  rupture de stock et au maintien des invendus. Sous sa forme classique, le problème
  revient aussi à une estimation de quantile. Voir par exemple \cite{shapiro2009lectures}
  pour une introduction au problème du vendeur de journaux dans un contexte d'optimisation
  stochastique.} à une situation ou la décision est prise à partir d'un grand nombre de
variables $X \in \Re^p$ liées au problème. Sous un noyau linéaire, l'objectif est alors de
déterminer $q$ de façon à
\begin{equation}
  \minimizeEquation[q]{\E c(q^TX,D).}
\end{equation}
où $c:\Re\times\bm{D}\to\Re$ est la fonction du coût correspondant à l'achat de $q^Tx$ journaux si
une demande aléatoire $D \in \bm{D} \subseteq \Re$ a lieu. La convexité de $c$ permet alors d'obtenir
des garanties hors échantillon lorsqu'un terme de régularisation quadratique est ajouté à
l'objectif:
\begin{equation}
  \minimizeEquation[q]{\E c(q^TX,D) + \lambda\|q\|^2.}
\end{equation}
De plus, ce papier cherche également à établir le rôle de la dimension $p$ de $X$ dans la
qualité des résultats. Ainsi, à bien des égards, son objectif est très semblable au
nôtre. 

Tous les travaux présentés jusqu'à présent font l'hypothèse que les variables aléatoires
d'intérêt (rendement et variables de marché) sont stationnaires dans le temps. À
l'inverse, \cite{cover1991universal} considère un univers ou les rendements sont
distribués selon un processus arbitraire et possiblement adversariel. L'idée est alors de
minimiser le \textit{regret} encouru par une décision scalaire d'investissement
$q_t \in \Re$, où le regret est calculé comme étant la fortune finale obtenue en appliquant
une décision $q_t$ par rapport à la fortune finale qui aurait été obtenue en
appliquant une politique optimale constante $q^\star$. Autrement dit,
\begin{equation}
  \text{Regret}(T) \coloneqq \max_{q^\star}\sum_{t=1}^T \log(q^\star\,r_t) - \sum_{t=1}^T \log(q_t\,r_t).
\end{equation}
En choisissant $q_t$ à partir d'une descente de gradient (voir par exemple
\cite{hazan2015online} pour un contexte plus général d'apprentissage en ligne) on peut
alors garantir un regret dont la progression est de $\bigO(\sqrt{T})$. Un tel algorithme
induisant un regret sous linéaire, \ie\ dominé par $\bigO(T)$, est appelé
\textit{portefeuille universel}.

\subsection{Conclusion}


En conclusion, la nature générale de notre modèle (impliquée par la forme arbitraire de la
fonction d'utilité) permet donc de le positionner comme une version flexible des travaux
présentés ici. Plusieurs avantages sont ainsi regroupés: le terme de risque arbitraire, la
stabilité des résultats consolidée par la présence de régularisation et l'admission de
variables de marché quelconques. Cependant, notre algorithme a le désavantage de ne
considérer ni un portefeuille à plusieurs actifs, ni de prendre en compte l'aspect
dynamique des marchés, ce qui ce fait généralement en gestion classique de portefeuille.


%%% Local Variables:
%%% mode: latex
%%% TeX-master: "memoire"
%%% End:
