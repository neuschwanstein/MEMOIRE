\section{Introduction}

\subsection{Définition du problème}


Ce mémoire vise à établir clairement et rigoureusement de quelle façon l'apprentissage
machine et la théorie de la statistique moderne permettent à un investisseur doté d'une
grande quantité d'information relative au marché (information que nous appelerons sans
discrimination \textit{facteurs de marché}, \textit{caractéristiques de marché},
... \cit).

Mathématiquement, on peut représenter cette situation en modélisant le \textit{marché}
comme étant une distribution statistique à haute dimension, ayant sur une marge une
variable aléatoire de rendement $R \in \R \subseteq \Re$, et ayant $p$ autres marges décrivant
l'information complémentaire $(X_1,\dots,X_m) \in \X \subseteq \Re^m$. La distribution complète de
marché $M \in \R \times \X$ est alors décrite par
\begin{equation}
  M = (R,X_1,\dots,X_m).
\end{equation}
Notons que la dépendance entre ces coordonnées marginales est inconnue.

On a d'autre part l'aspect d'aversion au risque qui est modélisé par une fonction
d'utilité $u:\R\to\U$, où $\R \subseteq \Re$ est le domaine (fermé ou non) des rendements considérés
et $\U \subseteq \Re$ celui des \textit{utilités}.

Bien qu'en pratique il soit plus facile de travailler sur des fonctions possédant des
valeurs dans $\U$, en pratique cet espace est adimensionel\cit, de sorte que nos résultats
seront présentés dans l'espace des rendements $\R$.

Donnés ces éléments de base, le but de ce mémoire sera alors de déterminer une fonction de
décision d'investissement $q: \X \to \P$ maximisant l'utilité espérée de l'investissement. 

Mathématiquement on a donc le problème fondamental suivant:
\begin{equation}
  \maximizeEquation{\E u(R \cdot q(X)).}
\end{equation}
\todo{Expliquer.}

\subsection{Hypothèses restrictives}



\subsection{Raffinements et hypothèses supplémentaires}


Formulé de façon statistique, lorsqu'on dispose de $n$ observations de $M$, le problème
peut se réexprimer comme
\begin{equation}
  \maximizeEquation{n^{-1}\sumi u(r_i\,q(x_i)).}
\end{equation}



%%% Local Variables:
%%% mode: latex
%%% TeX-master: "main_intro"
%%% End:
