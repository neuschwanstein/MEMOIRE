\section{Introduction}

\subsection{Exposition du problème et hypothèses}

Ce mémoire vise à établir clairement et rigoureusement comment un investisseur
\textit{averse au risque} disposant \textit{d'information complémentaire} au
\textit{marché} peut utiliser cette information pour accroître son \textit{utilité
  espérée} ou, de façon équivalente, son \textit{rendement équivalent certain}.

\paragraph{Modélisation du marché}

Nous entendrons ici par \textit{marché} n'importe quel type d'actif financier ou
spéculatif dans lequel on peut investir une partie de sa fortune dans l'espoir de la voir
fructifier au cours d'une période de temps arbitraire. Ainsi, tout au long de l'exposé
théorique qui suivra, il peut être pertinent d'avoir en tête les rendements quotidiens
issus des grands indices boursiers (par exemple les 500 plus grandes capitalisations
américaines). Cependant, le traitement qui sera développé pourrait tout aussi bien
s'appliquer à une action cotée en bourse dont on considère les rendements mensuels.\nec
Mathématiquement, l'idée de marché peut ainsi être réduite à celle d'une variable
aléatoire $R(t)$ décrivant l'évolution du rendement de l'actif en question.

Relativement à l'idée de marché, nous ferons également l'hypothèse que l'univers a une
influence sur ces rendements. Il serait par exemple raisonnable de croire que le prix du
pétrole a une influence sur l'évolution du rendement du marché américain. De la même
façon, l'annonce d'un scandale aura a son tour des répercussions sur la valeur du titre de
la compagnie dont il est l'objet. En outre, il a été montré par Fama et French que le
rendement d'une action pouvait s'expliquer comme une combinaison de quelques facteurs
fondamentaux (la taille de l'entreprise, le risque de marché et le ratio cours/valeur). On
peut alors considérer un vecteur d'information $\vec X(t)$ dont chaque composante
représente une information particulière, par exemple l'absence ou la présence d'un certain
type de scandale, un ratio comptable, le prix d'un certain actif financier\reph. 

D'un point de vue probabiliste, on peut alors décrire le rendement du marché à $t$ comme
étant une variable aléatoire influencée par l'ensemble des réalisations du vecteur
d'infor\-mation antérieures à $t$. Mathématiquement, on peut considérer le marché $M$
comme un étant processus stochastique multivarié ayant à l'une de ses marges le rendement
$R(t)$ et à l'autre l'ensemble d'information $\{\vec X(\tau) \mid \tau < t\}$.




\subsubsection{Hypothèses}

Le cadre théorique de ce travail imposera un certain nombre de restrictions afin d'en
dériver des résultats probants.

\paragraph{Stationarité}
Nous assumerons que la distribution de marché est \textit{stationnaire}. Entre autres
choses, cela nous permettra de supposer que toute observation mercantile est
\textit{identiquement et indépendamment distribué}\cit. Une telle hypothèse est en
pratique fort contraignante, puisqu'elle exclut d'emblée toute notion temporelle, pourtant
intuitivement importante pour un acteur du marché financier. 


Ce mémoire vise à établir clairement et rigoureusement de quelle façon l'apprentissage
machine et la théorie de la statistique moderne permettent à un investisseur doté d'une
grande quantité d'information relative au marché (information que nous appelerons sans
discrimination \textit{facteurs de marché}, \textit{caractéristiques de marché},
... \cit).

Mathématiquement, on peut représenter cette situation en modélisant le \textit{marché}
comme étant une distribution statistique à haute dimension, ayant sur une marge une
variable aléatoire de rendement $R \in \R \subseteq \Re$, et ayant $p$ autres marges décrivant
l'information complémentaire $(X_1,\dots,X_m) \in \X \subseteq \Re^m$. La distribution complète de
marché $M \in \R \times \X$ est alors décrite par
\begin{equation}
  M = (R,X_1,\dots,X_m).
\end{equation}
Notons que la dépendance entre ces coordonnées marginales est inconnue.

On a d'autre part l'aspect d'aversion au risque qui est modélisé par une fonction
d'utilité $u:\R\to\U$, où $\R \subseteq \Re$ est le domaine (fermé ou non) des rendements considérés
et $\U \subseteq \Re$ celui des \textit{utilités}.

Bien qu'en pratique il soit plus facile de travailler sur des fonctions possédant des
valeurs dans $\U$, en pratique cet espace est adimensionel\cit, de sorte que nos résultats
seront présentés dans l'espace des rendements $\R$.

Donnés ces éléments de base, le but de ce mémoire sera alors de déterminer une fonction de
décision d'investissement $q: \X \to \P$ maximisant l'utilité espérée de l'investissement. 

Mathématiquement on a donc le problème fondamental suivant:
\begin{equation}
  \maximizeEquation{\E u(R \cdot q(X)).}
\end{equation}
\todo{Expliquer.}

\subsection{Hypothèses restrictives}

\subsection{Interprétations}





\subsection{Raffinements et hypothèses supplémentaires}


Formulé de façon statistique, lorsqu'on dispose de $n$ observations de $M$, le problème
peut se réexprimer comme
\begin{equation}
  \maximizeEquation{n^{-1}\sumi u(r_i\,q(x_i)).}
\end{equation}


\subsection{Objectifs}





%%% Local Variables:
%%% mode: latex
%%% TeX-master: "main_intro"
%%% End:
