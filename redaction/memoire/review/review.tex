\section{Revue de littérature du portefeuille}


Dans ce document, nous allons tenter de classifier et de répertorier la plupart des
méthodes ayant rapport, de près ou de loin, à l'intersection des méthodes statistiques
avancées et de l'apprentissage machine avec la théorie du portefeuille, en présentant pour
chacune d'elle leurs avantages et leurs inconvénients.

\subsection{Théorie classique du portefeuille}

Une revue de littérature sur la théorie du portefeuille serait fondamentalement incomplète
sans l'article fondateur de Markowitz, publié en 1952 \cite{markowitz1952portfolio}.

Nous allons montrer que le cadre théorique développé par Markowitz peut être considéré
comme un cas particulier de notre algorithme, pour autant qu'on considère un portefeuille
à un seul actif.

Soit $w\in\real^k$ le vecteur représentant la répartition du portefeuille de Markowitz à
$k$ actifs à optimiser. Alors un investisseur \textit{markowitzien} souhaite résoudre le
problème suivant:
\begin{align*}
  \minimizeEquationSt{w^T\Sigma w}[w^T\mu = \mu_0],
\end{align*}
où $\Sigma\in\real^{k\times k}$ est la covariance du rendement des actifs et
$\mu\in\real^k$ le vecteur d'espérance.  \todo{Montrer formellement.} Par la théorie de
l'optimisation convexe, il existe une constante $\gamma$ telle que le problème énoncé est
équivalent à
\begin{equation*}
  \minimizeEquation{w^T\Sigma w + \gamma\,w^T\mu}.
\end{equation*}

Considérons à présent notre problème initial théorique, sans régularisation, lorsqu'un
investisseur est doté d'une utilité quadratique $u(r) = -1 + r -r^2/2$. Bien qu'une telle
utilité soit d'un usage limité en pratique puisqu'elle cherche à décourager tout autant
les rendements faibles que les rendements importants, il n'empêche que $u$ est une
fonction concave et qu'elle se prête donc tout à fait correctement à notre cadre
théorique. Notons toutefois que cette utilité peut aussi être interprétée comme étant les
deux premiers degrés d'une utilité exponentielle
$u(r) = -\exp(-r) = 1 + r - r^2/2 + \cdots$ \todo{Insérer un graphique?}. On cherche alors à
résoudre le problème suivant:
\begin{equation*}
  \maximizeEquation{\E u(R\cdot q^TX)}.
\end{equation*}
Or, 
\begin{equation*}
  \E u(R\cdot q^TX) = -1 + \E(R\cdot q^TX) - \tfrac{1}{2}\E(R\cdot q^TX)^2.
\end{equation*}
Supposons à présent que les facteurs considérés dans le programme d'optimi\-sation ne soit
uniquement formés que d'un terme constant de biais. Dans un tel cas, $X$ est une variable
aléatoire constante qui retourne toujours 1; $q^TX = q\in\real$ peut dès lors être
interprété comme l'allocation qu'on désire porter au portefeuille considéré. Le problème
devient alors simplement:
\begin{equation*}
  \maximizeEquation{q^T\E R - \tfrac{1}{2}q^2\E R^2}.
\end{equation*}
On sait qu'un programme d'optimisation est invariant à l'ajout de constante. Or, puisque
$\Var R = \E  R^2 - (\E R)^2$ et en posant $\mu = \E R$, on obtient alors
\begin{equation*}
  \maximizeEquation{q^T\mu - \tfrac{1}{2}
\end{equation*}



\subsection{Théorie de portefeuille régularisé}

\cite{ban2016machine}

\subsection{Fama and French}

\cite{fama1993common}

\subsection{Portefeuille universel}

Voir \cite{cover1991universal,hazan2015online}.

\subsection{Articles du NIPS}

\subsection{Papiers de Ben Van Roy}

\subsection{Notre problème par rapport à ces deux disciplines}









%%% Local Variables:
%%% mode: latex
%%% TeX-master: "main"
%%% End:
