\documentclass{article}
\newcommand{\ts}{\textsuperscript}
\newcommand{\figref}[1]{Fig.~\ref{#1}}
\usepackage{amsmath}
\usepackage{amsthm}
\usepackage{graphicx}
\usepackage{geometry}
\usepackage{subcaption}
\usepackage{bm}
\usepackage{hyperref}
\usepackage[retainorgcmds]{IEEEtrantools}
\usepackage{mathtools}
\usepackage{color}
\usepackage{marginnote}
\usepackage[utf8]{inputenc}


\DeclareMathOperator*{\argmax}{arg\,max}
\DeclareMathOperator*{\argmin}{arg\,min}
\DeclareMathOperator{\st}{s.t.}
\DeclareMathOperator{\epi}{epi}
\DeclareMathOperator{\diag}{diag}
\DeclareMathOperator{\dom}{dom}
\DeclareMathOperator{\tr}{tr}
\DeclareMathOperator*{\minimize}{minimize}
\DeclarePairedDelimiter\floor{\lfloor}{\rfloor}

\newcommand{\iso}{\simeq}
\newcommand{\dd}{\partial}
\newcommand{\real}{\bm R}
\newcommand{\trueRisk}{R_{\mathrm{true}}}

\newcommand{\starsection}{\vspace{1em}\begin{center}$\star\quad\star\quad\star$\end{center}\vspace{1em}}


\newcommand{\hilight}[1]{\colorbox{yellow}{#1}}
\let\oldmarginnote\marginnote
\renewcommand{\marginnote}[1]{\oldmarginnote{\footnotesize\emph{#1}}[0cm]}

\theoremstyle{plain}
\newtheorem{prop}{Proposition}
\newtheorem{thm}{Theorem}

\theoremstyle{definition}
\newtheorem*{deff}{Definition}
\newtheorem*{rem}{Remark}


\geometry{letterpaper}
\IEEEeqnarraydefcolsep{0}{\leftmargini}


\title{SVM Formulation of the Portfolio Optimization Problem}
\author{Thierry Bazier-Matte}

\begin{document}
\maketitle

\section{Risk Neutral and Linearly Separable}


Under an SVM formulation of our problem, we look for a decision $q$ such that $q^Tx$ is
the same sign as its output. Under a classical SVM formulation, where the problem is one
of classification, corresponding to a loss step function, the output is $y=\pm 1$.

Let us suppose for now that returns $r_i$ are linearly separable. Then we can scale the
decision $q$ to obtain $\min_S|q^Tx| = 1$, with $S$ the sample. Then, the distance of any
feature point $x_0\in\real^p$ will be given by
\[
  \frac{|q^Tx_0|}{\|q\|_2}.
\]
Therefore the margin $\rho$ of the SVM will be given by
\[
  \rho = \frac{1}{\|q\|_2}.
\]
Geometrically, we wih to have a margin as large as possible or, equivalently, to minimize
$\|q\|_2$, or more simply $\frac{1}{2}\|q\|^2_2$. 

Under the classical formulation of the SVM, where $y_i=\pm1$, we wish to have $\sign
q^Tx_i = \sign y_i$. This can also be expressed as
\[
  y_i\,q^Tx_i \geq 1,
\]
since $q^Tx_i\geq1$. Under the porfolio formulation, the output $y_i$ is actually the
portfolio return $r_i\in\real$. We would again like to obtain $\sign r_i = \sign
q^Tx_i$. Let us pretend for the moment that we are risk neutral and let's further suppose
that there is no $j$ such that $r_j=0$. Let $\bar r = \min_S r_i$. Then using the
transformation $\tilde r_i = r_i/\bar r$, we obtain $r_i \geq 1$. We therefore obtain the
same formulation as with the SVM, that is,
\begin{align*}
  \minimizeEquationSt{\|q\|^2}{\tilde r_i\,q^Tx_i\geq 1}.
\end{align*}
\end{document}

%%% Local Variables:
%%% mode: latex
%%% TeX-master: t
%%% End:
