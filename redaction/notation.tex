\section*{Notation}
\addcontentsline{toc}{section}{Notation}

% \begin{table}[h!]
\begin{tabular}{@{}lll}
  &\textbf{Objet} & \textbf{Notation}\\
  \toprule
  \multicolumn{3}{l}{\textit{Constantes du problème}}\\[1ex]
  & Facteur de régularisation & $\lambda \in \Re$\\
  & Fonction d'utilité & $u:\R\to\Re$\\
  \midrule
  \multicolumn{3}{l}{\textit{Espaces}\hspace{67ex}\,}\\[1ex]
  & Réels & $\Re$\\
  & Espace des rendements & $\R \subseteq \Re$\\
  & Espace des variables de marché & $\X \subseteq \Re^p$\\
  & Espace de décision & $\Q = \kappa(\X,\cdot)$\\
  \midrule
  \multicolumn{3}{l}{\textit{Variables aléatoires et réalisations}}\\[1ex]
  &Variables de marché & $X \in \X$\\
  &               & $x \sim X$\\[0.8ex]
  &Rendement & $R \in \R$\\
  &               &$r \sim R$\\[0.8ex]
  &Loi de marché & $(X,R) = M \in \M$\\
  &               & $(x,r) \sim M$\\
  
  \midrule
  \multicolumn{3}{l}{\textit{Décisions, transformation $\phi$ ou induite par noyau}}\\[1ex]
  & Décision & $q:\X\to\Re$\\
  & & $q(x) = \langle q|x \rangle$\\[0.8ex]
  & Transformation non linéaire & $\phi:\X \to \phi(\X)$\\[0.8ex]
  & Noyau & $\kappa:\X\times\X \to \Re$\\
  &                & $\kappa(x_i,x_j) = \inp{\phi(x_i),\phi(x_j)}  = \langle x_i|x_j \rangle$\\[0.8ex]
  \bottomrule
\end{tabular}
% \end{table}

\clearpage

% helo
\section*{Notation (2)}

\begin{tabular}{@{}lll}
  &\textbf{Objet} & \textbf{Notation}\\
  \midrule
  \multicolumn{3}{l}{\textit{Algorithmes et mesure de performance}\hspace{40ex}\,}\\[1ex]
  &Ensemble d'entraînement & $\S_n = \{(x_1,r_1),\ldots,(x_n,r_n)\}$\\
  &               &$\S_n\sim M^n$\\[0.8ex]
  & Algoritme d'aprentissage & $\alg:\M^n\to\Q$\\
  & &$\alg(\S_n) = \argmax_{q \in \Q}\hEU_\lambda(\S_n,q)$\\[0.8ex]
  & Utilité espérée  & $\EU(q) = \E u(R\cdot q(X))$\\[0.8ex]
  & Utilité espérée régularisée  & $\EU_\lambda(q) = \EU(q) - \lambda/2\|q\|^2$\\[0.8ex]  
  & Utilité espérée en échantillon & $\hEU(\S_n,q) = n^{-1}\sumi u(r_i\,q(x_i))$\\[0.8ex]
  & Utilité esp. en éch. régularisée & $\hEU_\lambda(\S_n,q) = \hEU(\S_n,q) - \lambda/2\|q\|^2$\\[0.8ex]
  & Rendement espéré & $\EN(q) = \E(R\cdot q(X))$\\[0.8ex]
  & Rendement espéré en échantillon & $\hEN(q) = n^{-1}\sumi r_i\,q(x_i)$\\[0.8ex]
  & Erreur de généralisation (utilité) & $\hat\zeta(\S_n) = \hEU(\S_n,\alg(\S_n)) -
                                         \EU(\alg(\S_n))$\\[0.8ex]
  & Erreur de sous optimalité (utilité) & $\zeta(\S_n) = \EU(\alg(\S_n)) -
                                          \EU(q^\star)$\\[0.8ex]    
  & Équivalent certain en échantillon & $\hCE(\S_n,q) =
                                        u^{-1}\big(\hEU(\S_n,q)\big)$\\[0.8ex]
  & Équivalent certain hors échantillon & $\CE(q) = u^{-1}\big(\EU(q)\big)$\\[0.8ex]
  & Erreur de généralisation (eq. cert.) & $\hat\zeta_e(\S_n) = \hCE(\S_n,\alg(\S_n)) -
                                           \CE(\alg(\S_n))$\\[0.8ex]
  & Erreur de sous optimalité (eq. cert.) & $\zeta_e(\S_n) = \CE(\alg(\S_n)) -
                                            \CE(q^\star)$\\[0.8ex]    
  \bottomrule
\end{tabular}
% \clearpage

%%% Local Variables:
%%% mode: latex
%%% TeX-master: "memoire"
%%% End:
