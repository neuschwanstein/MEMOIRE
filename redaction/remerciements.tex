\section*{Remerciements}
\addcontentsline{toc}{section}{Remerciements}

\epigraph{Nunc te, Bacche, canam!}{Virgile\\\textsc{Géorgiques}}

Je tiens d'abord à exprimer toute ma gratitude pour le professeur Erick Delage. Lorsque
j'ai décidé qu'il était temps pour moi de me remettre aux études, j'avais déjà parcouru la
liste des professeurs de HEC avec les simples mots clé {\small\verb+machine learning hec+}
et à l'époque c'était sa page personnelle qui arrivait en tête des résultats d'un
populaire moteur de recherche. Je n'ai donc pas hésité et dès le début du mois de
septembre je l'ai contacté.

J'ai apprécié toute sa rigueur, son support financier (merci pour l'invitation au NIPS!),
sa vaste expérience mais aussi et surtout son point de vue différent de ce que j'avais pu
connaître jusqu'alors : les problèmes convexes, les inégalités, etc. À ce titre, la
lecture du Convex Optimization, cet excellent livre de Boyd a été une révélation. Écrire
cet article pour NIPS a aussi été une expérience des plus enrichissantes. Un jour peut
être cet article verra-t-il le jour.

Maintenant, pour des remerciements plus personnels: d'abord mes parents, Denise et
Jean-François. De mon père je tiens son esprit scientifique et généraliste. À ce jour, il
demeure un homme dont la culture générale et scientifique ne cessera jamais de
m'étonner. De ma mère je tiens le goût de la rigueur mathématique, des problèmes d'esprit
et de l'élégance d'un programme bien construit. À tous les deux, je veux leur exprimer mon
immense gratitude de m'avoir supporté sans réserve dans mon choix de laisser là ``un bon
salaire'' pour continuer mes études. Véronique et Xavier ont aussi toute mon admiration,
ce sont tous deux d'excellents scientifiques destinés, j'en suis certain, à de brillantes
carrières.

À Ariadne aussi, je veux lui exprimer une reconnaissance bien particulière de m'avoir
encouragé au cours des derniers mois. Il est grand temps d'enfin boire un honnête Medoro
en écoutant un film de quatre heures sans être tracassés.

Et finalement à mes amis. Noé bien sûr, nos discussions inébriées sur des napkins au plan
B au sujet d'espaces vectoriels, à François, nos discussions inébriées sur le travail,
l'argent et la vieillesse. À Manu, nos discussions inébriées de métaphysique. Mais aussi
Pat, Max, Simon, Charles, Raph, Paradise et Alex. Mes amis d'HEC aussi, Edith, Nidhal, JS,
ces bons vieux 5 à 7.

Merci à tous et maintenant au travail. 

%%% Local Variables:
%%% mode: latex
%%% TeX-master: "memoire"
%%% End:
