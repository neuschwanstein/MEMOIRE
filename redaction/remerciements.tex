\section*{Remerciements}
\addcontentsline{toc}{section}{Remerciements}

Je tiens d'abord à exprimer toute ma gratitude pour le professeur Erick Delage. Lorsque
j'ai décidé qu'il était temps pour moi de me remettre aux études, j'avais déjà parcouru la
liste des professeurs de HEC avec le simple mot clé \verb+machine learning hec+ et à
l'époque c'était sa page personnelle qui arrivait en tête des résultats d'un populaire
moteur de recherche. Je n'ai donc pas hésité et dès le début du mois de septembre je l'ai
contacté. 

J'ai apprécié toute sa rigueur, sa vaste expérience mais aussi et surtout son point de vue
différent de ce que j'avais pu connaître jusqu'alors. La lecture du Convex Optimization,
cet excellent livre de Boyd a été une révélation. Écrire cet article pour NIPS
a aussi été une expérience des plus enrichissantes. Un jour peut être cet article
verra-t-il le jour. 

J'aimerais également remercier le professeur Bruno Rémillard. Son cours de statistiques a
été la deuxième révélation pour moi. Plusieurs méthodes d'apprentissage machine me
paraîssent encore un peu \textit{ad hoc}, alors que les statistiques offrent de véritables
garanties probabilistes sur les résultats. À n'en point douter, si doctorat un jour il y
a, ce sera dans le domaine des statistiques.

Maintenant, pour des remerciements plus personnels: d'abord mes parents, Denise et
Jean-François. De mon père je tiens son esprit scientifique et généraliste. À ce jour, il
demeure un homme dont la culture générale et scientifique ne cessera jamais de m'étonner,
du grec ancien à la biologie moderne, des sytèmes électroniques à la guerre
autro-prussienne de 1866. De ma mère je tiens le goût des mathématiques, des problèmes
d'esprit et de la programmation. À tous les deux, je veux leur exprimer mon immense
gratitude de m'avoir supporté sans réserve dans mon choix de laisser là ``un bon salaire''
pour continuer mes études. Véronique et Xavier ont aussi toute mon admiration, ce sont
tous deux d'excellents scientifiques destinés, j'en suis certain, à de brillantes
carrières.

À Ariadne aussi, je veux lui exprimer une reconnaissance toute particulière de m'avoir
encouragé au cours des derniers mois. Espérons qu'on puisse enfin boire un honnête Medoro
en écoutant un film de qutre heures sans être tracassés. Ça y est, on y est on se dirige
vers le \textit{bliss}, les belles randonnées, le jardinage en paix et la lecture de
Proust jusqu'à notre mort à l'âge vénérable de 105 ans.

Et finalement à mes amis. Noé bien sûr, nos discussions inébriées sur des napkins au plan
B au sujet d'espaces de Hilbert, de probabilité et de logique formelle. À François, nos
discussions inébriées sur le travail, l'argent et la vieillesse. À Manu, nos discussions
inébriées de philo cosmisque. À Patricia, nos discussions inébriées sur la vie. À Max, nos
joutes de tennis et les appels à 19h moins $\epsilon$. À Simon, nos discussions inébriées et
enfumées sur la musique et la politique. À Edith nos discussions sur la finance, le
networking et la programmation fonctionelle. À Alex nos discussions inébriées sur la
musique et les films. À Nidhal et JS les discussions inébriées aux 5à7. Et aux autres que
je vois moins souvent, mais dont les discussions n'en demeurent pas moins inébriées.

Merci à tous et maintenant au travail. 

%%% Local Variables:
%%% mode: latex
%%% TeX-master: "memoire"
%%% End:
