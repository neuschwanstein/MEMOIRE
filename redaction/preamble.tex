% \usepackage{nips_2016}

\usepackage{times}
\usepackage{amsmath}
\usepackage{amsthm,amsfonts}
\usepackage{graphicx}
\usepackage{mathtools}
\usepackage{bm}
\usepackage{xparse}
\usepackage[utf8]{inputenc}
\usepackage{mathrsfs}
\usepackage{eufrak}
\usepackage{parskip}
\usepackage{hyperref}

\allowdisplaybreaks


% KERNEL
\newcommand{\lag}{\mathscr{L}}
\newcommand{\sumij}{\sum_{i,j=1}^n}

% À classer.

\DeclareMathOperator*{\argmax}{arg\,max}
\DeclareMathOperator*{\argmin}{arg\,min}
\DeclareMathOperator*{\maximize}{maximize}
\DeclareMathOperator*{\minimize}{minimize}
\DeclareMathOperator{\epi}{epi}
\DeclareMathOperator{\diag}{diag}
\DeclareMathOperator{\dom}{dom}
\DeclareMathOperator{\tr}{tr}

\DeclareMathOperator{\sign}{sign}
\DeclareMathOperator{\Pdim}{Pdim}

\DeclareMathOperator{\E}{\bm E}
\DeclareMathOperator{\Var}{\bm{Var}}
\DeclareMathOperator{\CE}{CE}
\DeclareMathOperator{\Cov}{Cov}
\DeclareMathOperator{\EU}{\bm{EU}}
\DeclareMathOperator{\pp}{\bm P} 
% \DeclarePairedDelimiter\floor{\lfloor}{\rfloor}

\newcommand{\ie}{\textit{i.e.}}
\newcommand{\eg}{\textit{e.g.~}}
\newcommand{\iid}{i.i.d.~}

\newcommand{\ts}{\textsuperscript}
\newcommand{\figref}[1]{Fig.~\ref{#1}}

\newcommand{\qs}{q^\star}
\newcommand{\qsl}{q^\star_\lambda}
\newcommand{\qh}{\hat q}

\newcommand{\X}{\mathcal{X}}
\newcommand{\Y}{\mathcal{Y}}
\newcommand{\R}{\mathcal{R}}


\renewcommand{\O}[1]{O\left(#1\right)}

\newcommand{\rad}{\hat{\mathfrak{R}}}
\newcommand{\iso}{\simeq}
\newcommand{\dd}{\partial}
\newcommand{\real}{\mathscr{R}}
\renewcommand{\Re}{\real}
\newcommand{\normal}{\mathscr{N}}
\newcommand{\trueRisk}{R_{\mathrm{true}}}
\newcommand{\uInv}{u^{-1}}
\newcommand{\qHat}{{\hat q}}
\newcommand{\qStar}{{q^\star}}
% \newcommand{\xMax}{X_{\max}}
\newcommand{\xMax}{\xi}
\newcommand{\grad}{\nabla}
\newcommand{\sumi}{\sum_{i=1}^n}
% \newcommand{\dd}{\partial}

% \newcommand{\pp}{P}
% \newcommand{\CE}{\bm{CE}}
\newcommand\subsetsim{\mathrel{%
  \ooalign{\raise0.2ex\hbox{$\subset$}\cr\hidewidth\raise-0.8ex\hbox{\scalebox{0.9}{$\sim$}}\hidewidth\cr}}}

\def\rcurs{{\mbox{$\resizebox{.09in}{.08in}{\includegraphics[trim= 1em 0 14em
        0,clip]{ScriptR}}$}}}
\newcommand{\rf}{\rcurs_f}

\renewcommand{\rf}{r_0}

\newcommand{\starsection}{\vspace{1em}\begin{center}$\star\quad\star\quad\star$\end{center}\vspace{1em}}

% http://tex.stackexchange.com/questions/111551/recursive-multiple-subscript-and-superscript-with-xparse
\NewDocumentCommand{\minimizeEquationSt}{m >{\SplitList{;}}O{}}
{\begin{array}{rl}\mbox{minimiser} & \displaystyle{#1}\\\mbox{tel que}
    \ProcessList{#2}{\stCommand}\end{array}}

\NewDocumentCommand{\maximizeEquationSt}{m >{\SplitList{;}}O{}}
{\begin{array}{rl}\mbox{maximiser} & \displaystyle{#1}\\\mbox{tel que}
    \ProcessList{#2}{\stCommand}\end{array}}

\NewDocumentCommand{\stCommand}{m}{& #1}

\newcommand{\minimizeEquation}[1]{\begin{array}{ll}\mbox{minimiser} & \displaystyle{#1}\end{array}}
\newcommand{\maximizeEquation}[1]{\begin{array}{ll}\mbox{maximiser} & \displaystyle{#1}\end{array}}

\newcommand{\comment}[1]{\textbf{[#1]}}
\newcommand{\Erick}[1]{\textbf{[Erick says: #1]}}
\newcommand{\Thierry}[1]{\textbf{[Thierry says: #1]}}
\newcommand{\todo}[1]{\textbf{[Todo: #1]}}

\theoremstyle{plain}
\newtheorem{prop}{Proposition}
\newtheorem{thm}{Theorem}
\newtheorem{coro}{Corollary}
\newtheorem*{thm*}{Theorem}
\newtheorem{claim}{Claim}
\newtheorem{lemma}{Lemma}
\newtheorem*{lemma*}{Lemma}
\newtheorem{assumption}{Assumption}

\theoremstyle{definition}
\newtheorem*{definition}{Definition}
\newtheorem*{rem}{Remark}
\newtheorem*{ex}{Example}

%% http://tex.stackexchange.com/a/43971/4233
\makeatletter
\def\th@plain{%
  \thm@notefont{}% same as heading font
  \itshape % body font
}
\def\th@definition{%
  \thm@notefont{}% same as heading font
  \normalfont % body font
}
\makeatother
